\section{Scopo dell'incontro} \label{sec:argomenti}
Alla luce dell'incontro con il proponente ed in seguito ad un mirato studio individuale da parte dei membri del gruppo, viene svolta una ulteriore riunione interna per selezionare in maniera definitiva le tecnologie da implementare nel PoC. 

\section{Tecnologie scelte}
\subsection{Lato front-end}
Tra le tecnologie selezionate, è stato scelto di integrare nel progetto l'utilizzo della libreria \textbf{React}. I principali vantaggi sono: la facilità di apprendimento e 
di utilizzo pratico, la possibilità di riutilizzo delle componenti e la facile integrazione con Three.js tramite \textbf{React Three Fiber}. In particolare, quest'ultimo 
permette anche di ottimizzare la gestione della parte grafica, in quanto si occupa in autonomia di alcune operazioni che nativamente sono demandate allo sviluppatore.
\subsection{Lato back-end}
Oltre all'utilizzo di \textbf{Node.js}, il gruppo ha scelto di sviluppare tramite \textbf{Next.js}, un framework ``full-stack" che facilita la gestione delle 
operazioni lato server e ottimizza alcune operazioni lato client, facilitandone quindi anche lo sviluppo front-end. Si sono inoltre individuati i seguenti principali
vantaggi: la semplicità di definizione della struttura delle pagine della web-app e delle API per la comunicazione server-client, il supporto a \textbf{React} e 
l'implementazione di alcune interessanti feature ausiliarie, come ad esempio l'\textit{hot reloading}.
\subsection{Deploy}
Dopo aver approfondito e testato alcune tecnologie per il deploy del prodotto, il gruppo ha deciso di appoggiarsi a \textbf{Vercel}, una piattaforma che permette il 
deploy dell'applicativo collegandosi al repository Github, che risulta inoltre ottimizzato per tecnologie come React e Next.js. Dalle simulazioni effettuate è inoltre 
emersa l'estrema facilità ed intuitività di tale tecnologia che risulta quindi ideale per le esigenze di progetto.