\section{Scopo dell'incontro} \label{sec:argomenti}
La riunione si è svolta per selezionare in maniera definitiva le tecnologie e i framework da implementare nel PoC, alla luce dell'incontro con l'azienda e dello studio individuale dei membri del gruppo.

\section{Tecnologie scelte}
\subsection{Lato front-end}
Tra le tecnologie selezionate, la scelta ritenuta ottimale è stata \textbf{React}. I vantaggi sono: la facilità di apprendimento, di utilizzo pratico, la possibilità di riutilizzo delle componenti e la facile integrazione con Three.js tramite \textbf{React three fiber}. In particolare, quest'ultimo permette anche di gestire la parte grafica in maniera più ottimizzata, perché si occupa in automaticamente di alcune operazioni che nativamente sono demandate allo sviluppatore.
\subsection{Lato back-end}
Oltre all'utilizzo di \textbf{Node.js}, il gruppo ha scelto di applicare anche \textbf{Next.js}, un framework ``full-stack", che quindi facilita la gestione delle operazioni lato server, ma che permette anche di ottimizzare alcune operazioni lato client, facilitando anche il front-end. Oltre a questo, i vantaggi sono la semplicità di definizione della struttura delle pagine della webapp e delle api per la comunicazione server-client, il supporto a \textbf{React} e alcune feature ausiliarie, come l'\textit{hot reloading}.
\subsection{Deploy}
Il gruppo ha scelto di testare anche alcune tecnologie per il deploy del prodotto, ed in partiolare ha selezionato \textbf{Vercel}, che permette il deploy dell'applicativo collegandosi al repository Github ed è ottimizzato per tecnologie come React e Next.js. Dalle prove fatte, inoltre, l'utilizzo dello strumento risulta facile ed intuitivo.