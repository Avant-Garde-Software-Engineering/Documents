\section{Argomenti} \label{sec:argomenti}
La riunione si è tenuta per effettuare il cambio dei ruoli e stabilire gli obiettivi da raggiungere entro la fine della settimana.
    \begin{itemize}
        \item \textbf{Retrospettiva su stato di avanzamento} \\
        Il gruppo ha stabilito di aver raggiunto uno stato di avanzamento tale per cui alcune risorse possono essere indirizzate verso la progettazione e l'ideazione del PoC, mentre le risorse rimanenti saranno dedicate al prosieguo della documentazione. Per questa ragione, due componenti saranno assegnati alla progettazione, mentre gli altri ruoli saranno distribuiti tra i componenti rimanenti.
        \item \textbf{Assegnazione dei ruoli} \\
        I nuovi ruoli sono stati assegnati come segue:
        \begin{itemize}
            \item \textbf{Responsabile}: Lorenzo Pasqualotto
            \item \textbf{Amministratore}: Giulio Biscontin
            \item \textbf{Analista}: Luca Securo, aiutato da Lorenzo Pasqualotto se necessario
            \item \textbf{Progettisti}: Andrea Mangolini, Jessica Carretta
            \item \textbf{Verificatore}: Zaccaria Marangon
        \end{itemize}
        \item \textbf{Obiettivi per la settimana}\\
        Gli obiettivi principali si dividono su tre linee di lavoro:
        \begin{itemize}
            \item \textbf{Continuazione della documentazione}: in particolare dei documenti di analisi dei requisiti, piano di progetto e piano di qualifica secondo istruzioni riportate sulle Norme di Progetto.
            \item \textbf{Studio delle tecnologie}: da parte di ciascun membro del gruppo, individualmente.
            \item \textbf{Inizio della progettazione}: in particolare per il PoC, ad opera dei progettisti.
        \end{itemize}
        \item \textbf{Discussione sul contenuto del PoC} \\
        Si è deciso che il PoC deve avere come obiettivo quello di realizzare tutti i requisiti minimi di progetto, seppure in forma base. Gli obiettivi massimi, al contrario, non sono da considerarsi essenziali in questa prima istanza. Infatti, il PoC serve per dimostrare la fattibilità del capitolato tramite le tecnologie individuate dal gruppo e dal proponente. Pertanto, è imperativo realizzarlo nel minor tempo possibile durante il corso del progetto, in modo tale che il gruppo e l'azienda siano sicuri di poter proseguire nella realizzazione di un prodotto più completo, tramite le tecnologie individuate. Questo approccio eviterà di trovarsi in difficoltà quando il progetto sarà in uno stadio più avanzato.
    \end{itemize}