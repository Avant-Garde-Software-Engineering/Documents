\section{Argomenti}
    \begin{itemize}
        \item Discussione sui tipi di documento prodotti durante il corso del progetto e sull'utilizzo dei template. Il gruppo ha individuato due tipi di template:
        \begin{itemize}
            \item Template generico: da applicare a tutti i documenti esclusi i verbali, che deve contenere:
            \begin{itemize}
                \item \textbf{Frontespizio} con logo, titolo e informazioni sulla versione attuale;
                \item \textbf{Changelog};
                \item \textbf{Indice dei contenuti};
                \item \textbf{Contenuti}, divisi per sezioni.
            \end{itemize}
            \item Template verbale: da applicare ai verbali, oltre agli elementi del verbale generico prevede:
            \begin{itemize}
                \item Sezione per la \textbf{firma di approvazione esterna} nel caso di un verbale esterno;
                \item \textbf{Informazioni} sui partecipanti esterni ed interni e canale di comunicazione.
            \end{itemize}
        \end{itemize}
        Per la lettera di candidatura si è scelto di mantenere un formato in stile lettera postale, senza changelog perché, per sua natura, non prevede versionamento.
        \item Confronto tra i membri del gruppo sull'analisi dei requisiti
            \begin{itemize}
                \item Brainstorming collettivo riguardo i diversi use case del progetto da sviluppare;
                \item Creazione collettiva di una bozza di use case per la creazione del magazzino e per la creazione delle scaffalature;
                \item Riflessione su come realizzare lo use case sulla creazione di un prodotto.
            \end{itemize}
        \item Durante l'incontro sono sorti dubbi sulle possibili situazioni che si potrebbero verificare nel programma. Inoltre, alcune frasi presenti all'interno della presentazione del capitolato non sono chiare. Le domande sono:
            \begin{itemize}
                \item Come indicare la posizione di un prodotto in un scaffale, cosa si intende per coordinata;
                \item Dubbio riguardante la forma dei prodotti in 3D e la loro visualizzione;
                \item Chiaramenti su come dovremmo affrontare lo spostamento di un prodotto tra i vari scaffali;
                \item Necessità di una sezione dedicata agli spostamenti in corso;
                \item Dubbio in merito al significato delle sessioni volatili;
                \item Chiarimenti sul significato della sezione ``simulare il livello di servizio" presente nel pdf a pagina 14.
            \end{itemize}
        Si è deciso che tutte le domande e i dubbi che abbiamo riscontrato saranno poste all'azienda tramite incontro da remoto, per ottenere delle delucidazioni o risposte a riguardo una volta che l'incontro con essa verrà effettuato.\\
        \item Discussione e successiva conferma di come si eseguirà la stesura, modifica e chiusura dell'analisi dei requisiti. In particolare, il documento deve contenere:
        \begin{itemize}
            \item Sezione dedicata ai casi d'uso in forma grafica e descrittiva;
            \item Sezione dedicata alla scrittura in forma tabellare dei requisiti.
        \end{itemize}
    \end{itemize}