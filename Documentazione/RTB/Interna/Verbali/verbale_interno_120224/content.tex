\section{Argomenti} \label{sec:argomenti}
La riunione si è svolta per fare un punto della situazione su cosa è stato fatto rispetto all'ultimo incontro, per decidere come procedere nelle prossime settimane (verso la RTB) e per fare un cambio dei ruoli.
\\ \noindent Si riportano di seguito i principali punti discussi e le decisioni prese in merito.
    \begin{itemize}
        \item \textbf{Retrospettiva su stato di avanzamento} \\
        Il gruppo ha stabilito che la documentazione finora prodotta si trova ad un buon livello di avanzamento, in particolare:
        \begin{itemize}
            \item L'\textbf{Analisi dei Requisiti} v1.0.0 è stata approvata internamente ed è stata inviata all'azienda per una conferma di quanto prodotto;
            \item Le \textbf{Norme di Progetto} v2.0.0 sono state approvate internamente e sono da ritenersi complete per la RTB;
            \item Il \textbf{Piano di progetto} è stato in buona parte verificato e, a meno di qualche inserimento finale, è da considerarsi a buon punto;
            \item Il \textbf{Piano di Qualifica} è stato verificato e si trova alla versione v0.1.0, pronto per l'approvazione;
            \item Il \textbf{Glossario} è stato prodotto in concomitanza con gli altri documenti ed è quindi aggiornato, a meno di un controllo finale.
        \end{itemize}
        Per il PoC sono state implementate la maggior parte delle funzionalità previste. Tuttavia, allo stato attuale, mancano:
        \begin{itemize}
            \item L'inserimento dei prodotti nel magazzino;
            \item La richiesta di spostamento dei prodotti tra scaffalature.
        \end{itemize}        
        \item \textbf{Assegnazione dei ruoli} \\
        Visto il buon avanzamento della documentazione, buona parte delle risorse è stata dedicata alla programmazione in conclusione del PoC, mantenendo comunque attivi i ruoli essenziali di responsabile, amministratore e verificatore. I nuovi ruoli sono stati assegnati come segue:
        \begin{itemize}
            \item \textbf{Responsabile}: Zaccaria Marangon;
            \item \textbf{Amministratore}: Andrea Mangolini;
            \item \textbf{Programmatori}: Giulio Biscontin, Luca Securo, Lorenzo Pasqualotto e, se necessario, Zaccaria Marangon;
            \item \textbf{Verificatore}: Jessica Carretta.
        \end{itemize}
        \item \textbf{Obiettivi per le prossime settimane}\\
        Gli obiettivi principali si dividono su due linee di lavoro:
        \begin{itemize}
            \item \textbf{Conclusione della documentazione}: In previsione della RTB, tutti i documenti devono essere approvati e trovarsi ad uno stato che si può considerare ``concluso" per la RTB.
            \item \textbf{Avanzamento del PoC}: I programmatori devono implementare le funzionalità mancanti il prima possibile.
        \end{itemize}
        \item \textbf{Contatti con l'azienda} \\
        L'Analisi dei requisiti è stata inviata all'azienda per assicurarsi che rispetti quanto richiesto dal proponente. Il gruppo ha deciso di richiedere anche un meeting con l'azienda, al fine di:
        \begin{itemize}
            \item Mostrare il PoC allo stato attuale ed assicurarsi che stia procedendo nella giusta direzione rispetto a quanto richiesto;
            \item Eventualmente discutere dell'Analisi dei requisiti inviata, se l'azienda lo ritiene necessario.
        \end{itemize}
    \end{itemize}