\section{Argomenti} \label{sec:argomenti}
La riunione si è tenuta per discutere delle modalità con cui affrontare la revisione dell'avanzamento (RTB) con il professor Riccardo Cardin e per ultimare le parti mancanti della documentazione.

Si è scelto di suddividere la presentazione nel seguente modo:

\begin{itemize}
    \item \textbf{Prima parte:} La prima parte concerne la spiegazione del capitolato e le modalità attraverso cui abbiamo ricavato i requisiti. Verrà presentata da \textbf{Jessica Carretta} e \textbf{Luca Securo}.
    \item \textbf{Seconda parte:} La seconda parte riguarda la spiegazione del codice realizzato e le motivazioni dietro alle scelte tecnologiche intraprese. Verrà presentata da \textbf{Giulio Biscontin} e \textbf{Lorenzo Pasqualotto}.
\end{itemize}

Per quanto concerne la documentazione è stata richiesta una partecipazione ai membri del gruppo per rendere ottimale l'acquisizione di alcune metriche di qualità riguardanti la documentazione. Ogni documento infatti dovrà essere valutato 1. ortograficamente e 2. secondo l'indice di Gulpease. Di seguito gli assegnamenti:
\begin{itemize}
    \item Analisi dei Requisiti: Zaccaria Marangon
    \item Norme di Progetto: Andrea Mangolini
    \item Piano di Progetto: Luca Securo
    \item Piano di qualifica: Jessica Carretta
    \item Glossario: Lorenzo Pasqualotto
\end{itemize}

Il controllo ortografico dovrà essere effettuato manualmente attraverso l'usilio di \textcolor{blue}{Microsoft Word}, mentre per il controllo dell'indice di Gulpease si consiglia l'utilizzo dello strumento presente al seguente \href{https://www.dogmadynamics.com/calcolo-indice-lettura-facile.html}{link}.