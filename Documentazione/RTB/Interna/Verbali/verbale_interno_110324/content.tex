\section{Argomenti} \label{sec:argomenti}
La riunione si è tenuta principalmente per discutere delle modalità con cui affrontare la revisione di avanzamento RTB con il professor Riccardo Cardin e organizzarne la presentazione. Si sono inoltre discussi alcuni punti utili per ultimare le parti mancanti della documentazione.

\subsection{Organizzazione della presentazione per l'RTB}
Si è scelto di suddividere la presentazione nel seguente modo:
\begin{itemize}
    \item \textbf{Prima parte:} Il focus è sull'Analisi dei requisiti, in particolare verrà introdotto il capitolato e verranno spiegate le modalità attraverso cui i requisiti sono stati ricavati. Verrà presentata da \textbf{Jessica Carretta} e \textbf{Luca Securo}.
    \item \textbf{Seconda parte:} Il focus è sul PoC, verrà infatti mostrato quest'ultimo, spiegandone le motivazioni dietro le scelte tecnologiche intraprese. Verrà presentata da \textbf{Giulio Biscontin} e \textbf{Lorenzo Pasqualotto}.
\end{itemize}

\subsection{Ultimazione dei documenti}
Per l'aggiornamento del ``Piano di Qualifica'', viene richiesta la partecipazione dell'intero gruppo per la misurazione di alcune metriche. In particolare, per quanto riguarda la documentazione, si richiede che ogni documento debba essere valutato ortograficamente e secondo l'indice di Gulpease. Il suddetto controllo dei documenti viene dunque delegato ai singoli membri del gruppo come segue:
\begin{itemize}
    \item \textbf{Analisi dei Requisiti:} Zaccaria Marangon;
    \item \textbf{Norme di Progetto:} Andrea Mangolini;
    \item \textbf{Piano di Progetto:} Luca Securo;
    \item \textbf{Piano di qualifica:} Jessica Carretta;
    \item \textbf{Glossario:} Lorenzo Pasqualotto.
\end{itemize}
Si precisa che il controllo ortografico dovrà essere effettuato manualmente attraverso l'ausilio di \textit{Microsoft Word}, mentre per il controllo dell'indice di Gulpease si consiglia l'utilizzo dello strumento presente al seguente \href{https://www.dogmadynamics.com/calcolo-indice-lettura-facile.html}{link}.