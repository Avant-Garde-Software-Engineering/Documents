\section{Scopo dell'incontro} \label{sec:scopo}
La riunione ha il duplice scopo di discutere con il proponente il documento di Analisi dei requisiti v.1.0.0, precedentemente inviato, e di condividere il progresso relativo al PoC, in modo tale da garantire la coerenza di entrambi i prodotti con gli obiettivi di \nomeAzienda. 

\section{Argomenti} \label{sec:argomenti}
\noindent Si riportano di seguito gli argomenti trattati in dettaglio.

\subsection{Discussione riguardante l'Analisi dei requisiti} \label{sec:argomenti:analisi}
    Il proponente comunica al team che l'Analisi dei requisiti è conforme alle sue aspettative. 
    Tuttavia, esprime anche alcuni dubbi sorti durante la lettura del documento, i quali vengono chiariti durante la riunione. Questi dubbi riguardano:
    \begin{itemize}
        \item \textbf{Modalità assistita e visualizzazione 3D} \\
        Il proponente manifesta la sua perplessità sul significato di ``modalità assistita'', termine introdotto alla sezione \textit{2.2 - Descrizione del prodotto} del documento in questione. Con questa modalità si indica la possibilità, alternativa e parallela al render 3D, di gestire e visualizzare gli elementi del magazzino attraverso bottoni, pannelli informativi e menù contestuali.
        Non si fa invece riferimento a nessuna divisione di responsabilità tra gli utenti.
        Dopo una breve spiegazione in merito, il proponente conferma la scelta, mostrandosi d'accordo sull'inserimento di questa modalità.
    \end{itemize}
    Inoltre, il proponente dispensa alcuni consigli relativi a:
    \begin{itemize}
        \item \textbf{Dimensione degli oggetti}\\
        Con riferimento al caso d'uso \textit{UC13 - Creazione prodotto}, il proponente suggerisce di semplificare la creazione del prodotto eliminando il sottocaso \textit{UC13.2 - Inserimento dimensioni}, in altre parole di rendere il prodotto di dimensioni fisse (uguali alla dimensione del bin) e non modificabili dall'utente.
    \end{itemize}

\subsection{Discussione riguardante il PoC} \label{sec:argomenti:discussione_poc}
Viene mostrato al proponente il PoC realizzato finora, il quale procede conformemente alle aspettative. 
A tal proposito, egli ha inoltre fornito alcuni consigli e richieste da implementare, ovvero:
\begin{itemize}
    \item \textbf{Separazione visuale dei bin di una scaffalatura}\\
    Nello stato attuale, una volta creata una scaffalatura, non è presente un modo di distinguere visivamente i diversi bin vuoti che la compongono. Per questo motivo, il proponente consiglia di aggiungere una separazione grafica in modo da poter capire facilmente quanti bin sono presenti in una scaffalatura.
    \item \textbf{Aggiunta della voce ``Dimensione bin'' nella creazione di uno scaffale}\\
    Il proponente richiede la possibilità di creare una scaffalatura contenente bin di dimensioni variabili a seconda dell'input fornito dall'utente, perciò si rende necessario aggiungere una voce apposita nel form di creazione dello scaffale.
    \item \textbf{Possibile rimozione dell'altezza del magazzino}\\
    Il proponente, notando che gli utenti possono scegliere l'altezza del magazzino, ha informato il team sulla possibilità di rimuovere questa opzione, in quanto non è strettamente necessaria per la creazione del magazzino. Lo scopo principale è infatti quello di gestire le dimensioni orizzontali più che verticali. Se mantenuta, consiglia di modificarla leggermente per renderla più intuitiva per l'utente dal punto di vista grafico.
\end{itemize}

\subsection{Altri argomenti generici} \label{sec:argomenti:altri}
    Al termine della riunione, sono stati posti dal proponente alcuni quesiti riguardanti il progetto più in generale, quali:
    \begin{itemize}
        \item \textbf{Tempistiche previste e prossimi incontri}\\
        Si prevede di completare il PoC entro il 23/02/24 e di fissare una riunione immediatamente dopo per assicurarsi che anche le ultime funzionalità aggiunte rispettino i desideri del proponente.
        \item \textbf{Differenze con progetti effettuati in precedenza ed eventuali difficoltà incontrate}\\
        Il team rileva delle differenze sul livello di formalità e dettaglio del progetto. Infatti, mentre i progetti precedenti si concentravano sulla produzione di codice, nel progetto corrente assumono una parte altrettanto importante documenti e metodicità nel lavoro.
        \item \textbf{Scelte progettuali effettuate per il PoC}\\
        Il team si è concentrato sull'implementazione dei requisiti minimi che risultavano essere di maggiore importanza per lo sviluppo dell'applicativo. Le tecnologie scelte per lo sviluppo dell'interfaccia sono state HTML, CSS e JavaScript (con l'uso della libreria Three.js per la gestione della parte 3D), mentre per quanto riguarda il back-end si è scelto di utilizzare Node.js. Scelte riguardanti efficienza, scalabilità e sicurezza non sono state prese in considerazione per il PoC, vista la sua natura ``usa-e-getta" atta a verificare la fattibilità del capitolato.
    \end{itemize}