\section{Introduzione}\label{sec:introduzione}

\subsection{Scopo del Documento}\label{sec:scopo_del_documento}
Il presente documento ha lo scopo di fornire una descrizione dettagliata dei requisiti di progetto. Il documento é quindi frutto di un'attenta analisi del capitolato, proposto dal proponente, al fine di individuare ed esaminare in modo esaustivo i requisiti minimi, massimi e facoltativi presenti nel progetto. 
Un ulteriore obbiettivo del documento é quello di analizzare e descrivere le possibili attività (use cases) permesse all'utente in fase di esecuzione del programma. 

\subsection{Glossario}\label{sec:glossario}

Al fine di evitare incomprensioni relative alla terminologia usata all’interno del presente documento, viene 
fornito un Glossario di progetto. Ogni terminologia specifica di progetto verrà chiarita e definita in modo tale 
da renderla maggiornamente comprensibile ed evitare errate interpretazioni. Pertanto ogni termine marchiato
con una lettera G come apice, sarà specificato e presente nel glossario sopra citato.

\subsection{Riferimenti}\label{sec:riferimenti}

\subsubsection{Riferimenti normativi}\label{sec:riferimenti_normativi}
\begin{itemize}
    \item Norme di Progetto v1.0.0 \\
    ++++Inserire link alle norme di progetto++++
    \item Capitolato C5 - WMS3: wharehouse management 3D \\
    https://www.math.unipd.it/~tullio/IS-1/2023/Progetto/C5.pdf
\end{itemize}

\subsubsection{Riferimenti informativi}\label{sec:riferimenti_informativi}
\begin{itemize}
    \item Analisi e descrizione delle funzionalita: Use case e relativi diagrammi (UML) - materiale didattico corso
    di Ingegneria del Software\\
    https://www.math.unipd.it/rcardin/swea/2022/Diagrammi%20Use%20Case.pdf
    \item Analisi dei requisiti (T5) - materiale didattico corso di Ingegneria del Software\\
    https://www.math.unipd.it/~tullio/IS-1/2023/Dispense/T5.pdf
\end{itemize}

\newpage

\section{Descrizione}\label{sec:descrizione}

\subsection{Scopo del progetto}\label{sec:scopo_del_progetto}
Lo scopo del progetto é quello di ottimizzare i processi produttivi attraverso l'utilizzo di 
un applicativo 3D che rappresenta lo stato di fatto di un magazzino di stoccaggio.\\
L'applicativo, pensato per un'utilizzo da ufficio, deve permettere all'utente di navigare\textsuperscript{G} all'interno 
del magazzino e poter gestire le seguenti operazioni:
\begin{itemize}
    \item Creazione magazzino:
        \begin{itemize}
            \item Creare un nuovo magazzino
            \item Importare un layout esistente
        \end{itemize}
    \item Modifiche alla struttura del magazzino:
        \begin{itemize}
            \item Modificare scaffalature
            \item Aggiungere nuove scaffalature
            \item Rimuovere scaffalature esistenti
        \end{itemize}
    \item Modifiche ai prodotti presenti in magazzino:
        \begin{itemize}
            \item Eliminare un prodotto
            \item Aggiungere un nuovo prodotto
            \item Spostare un prodotto esistente 
        \end{itemize}
\end{itemize}
Sfruttare quindi un'esperienza tridimensionale per aiutare la comprensione e facilitare la gestione delle 
operazioni di logistica necessarie. 


\subsection{Descrizione del prodotto}\label{sec:descrizione_del_prodotto}

+++++ ancora da definire correttamente
L'applicativo, pensato per soddisfare lo scopo e i requisiti di progetto, si compone principalmente di due 
sezioni distinte. La prima sezione é composta dal terminale, attraverso il quale l'utente potrà interagire 
in modalità assistita\textsuperscript{G}. All'utente sarà infatti permesso, attraverso tale terminale, un certo numero di 
operazioni 
La seconda parte invece è composta dalla visualizzazione vera e proprio del magazzino 3D. In questa area 
l'utente avrà modo di "navigare" e 

\newpage

\section{Use Cases}\label{sec:use_cases}

\Large\textbf{}\\
\Large\textbf{Use Case 1 - Creazione nuovo layout} \\
\vspace{0.5cm}
%\begin{figure}[h]
%  \centering
%  \includegraphics[width=0.8\textwidth]{UseCasesImages/ObjCr.png}
%\end{figure}
\large\textbf{} \\
\textbf{Attori:} User\\
\textbf{Pre-condizione:} Avvio dell'applicazione da parte dell'utente\\
\textbf{Post-condizione: } Creazione del layout magazzino\\
\textbf{Scenario Principale:}  All'utente viene richiesto se caricare un layout esistente o se creare 
un nuovo layout.\\
\vspace{0.5cm}

\Large\textbf{}\\
\Large\textbf{Use Case 1.1 - Creazione Nuovo magazzino} \\
\vspace{0.5cm}
%\begin{figure}[h]
%  \centering
%  \includegraphics[width=0.8\textwidth]{UseCasesImages/ObjCr.png}
%\end{figure}
\large\textbf{} \\
\textbf{Attori:} User\\
\textbf{Pre-condizione:} E' stata selezionata la modalità: "Creazione manuale magazzino" \\
\textbf{Post-condizione: } Nuovo layout magazzino creato correttamente\\
\textbf{Scenario Principale:}  L'utente viene invitato ad indicare le dimensioni del magazzino in modo da poterne creare il layout.\\
\textbf{Estensioni: } UC1.1.1 - Errore Creazione Layout magazzino\\
\vspace{0.5cm}

\Large\textbf{}\\
\Large\textbf{Use Case 1.2 Importazione layout magazzino} \\
\vspace{0.5cm}
%\begin{figure}[h]
%  \centering
%  \includegraphics[width=0.8\textwidth]{UseCasesImages/ObjCr.png}
%\end{figure}
\large\textbf{} \\
\textbf{Attori:} User\\
\textbf{Pre-condizione:} E' stato selezionata la modalità: "importazione layout magazzino" \\
\textbf{Post-condizione: } Importazione Layout Magazzino\\
\textbf{Scenario Principale:}  Il magazzino viene correttamente importato da layout esistente.\\
\textbf{Estensioni: } UC1.2.1 - Errore Importazione Layout\\
\vspace{0.5cm}


\Large\textbf{}\\
\Large\textbf{Use Case 2 - Creazione scaffalatura} \\
\vspace{0.5cm}
%\begin{figure}[h]
%  \centering
%  \includegraphics[width=0.8\textwidth]{UseCasesImages/ObjCr.png}
%\end{figure}
\large\textbf{} \\
\textbf{Attori:} User\\
\textbf{Pre-condizione:} Richiesta nuovi inserimento scaffalatura \\
\textbf{Post-condizione: } Creazione corretta della scaffalatura\\
\textbf{Scenario Principale:}  L'utente crea una nuova scaffalatura inserendone le \textit{specifiche caratteristiche}. Se i dati inseriti non rispettano il \textit{criterio prestabilito} viene visualizzato un messaggio di errore.\\
\textbf{Estensioni: } UC2.1 - Errore Creazione scaffalatura\\
\vspace{0.5cm}

\Large\textbf{}\\
\Large\textbf{Use Case 3 - Spostamento scaffalatura} \\
\vspace{0.5cm}
%\begin{figure}[h]
%  \centering
%  \includegraphics[width=0.8\textwidth]{UseCasesImages/ObjCr.png}
%\end{figure}
\large\textbf{} \\
\textbf{Attori:} User\\
\textbf{Pre-condizione:} Richiesta spostamento scaffalatura \\
\textbf{Post-condizione: } Scaffallatura spostata \\
\textbf{Scenario Principale:}  L'utente selezione la scaffalatura da spostare e il nuovo spazio dove posizionarla.\\
\textbf{Estensioni: } UC3.1 - Errore Spostamento Scaffallatura\\
\vspace{0.5cm}

\Large\textbf{}\\
\Large\textbf{Use Case 4 - Modifica scaffalatura} \\
\vspace{0.5cm}
%\begin{figure}[h]
%  \centering
%  \includegraphics[width=0.8\textwidth]{UseCasesImages/ObjCr.png}
%\end{figure}
\large\textbf{} \\
\textbf{Attori:} User\\
\textbf{Pre-condizione:} Scaffallatura correttamente presente \\
\textbf{Post-condizione: } Corretta modifica scaffalatura\\
\textbf{Scenario Principale:}  L'utente, dopo aver selezionato la scaffalatura, seleziona il comando "Modifica scaffalatura". \\
\textbf{Estensioni: } UC4.1 - Errore modifica scaffalatura\\
\vspace{0.5cm}

\Large\textbf{}\\
\Large\textbf{Use Case 5 - Selezione scaffalatura} \\
\vspace{0.5cm}
%\begin{figure}[h]
%  \centering
%  \includegraphics[width=0.8\textwidth]{UseCasesImages/ObjCr.png}
%\end{figure}
\large\textbf{} \\
\textbf{Attori:} User\\
\textbf{Pre-condizione:} Scaffallatura correttamente presente \\
\textbf{Post-condizione: } Visualizzazione dati scaffalatura\\
\textbf{Scenario Principale:}  L'utente sta navigando\textsuperscript{G} all'interno del magazzino e prova a selezionare una scaffalatura esistente. \\
\vspace{0.5cm}

\Large\textbf{}\\
\Large\textbf{Use Case 6 Creazione Oggetto} \\
\vspace{0.5cm}
\begin{figure}[h]
 \centering
 \includegraphics[width=0.8\textwidth]{UseCasesImages/ObjCreation.png}
\end{figure}

\large\textbf{} \\
\textbf{Attori:} User\\
\textbf{Pre-condizione:} Magazzino correttamente istanziato \\
\textbf{Post-condizione: } Creazione di un oggetto\\
\textbf{Scenario Principale:}\\
\begin{itemize}
  \item L'utente inserisce un codice identificativo dell'oggetto
  \item L'oggetto viene creato
\end{itemize}.\\
\textbf{Estensioni: } UC6.1 Errore Creazione Oggetto\\

\vspace{0.5cm}

{\color{red}{\textbf{Domanda:} La creazione di un oggetto puó avvenire anche senza inserimento?}}

\vspace{0.5cm}

\Large\textbf{}\\
\Large\textbf{Use Case 6.1 Errore Creazione Oggetto} \\
\large\textbf{} \\
\textbf{Attori:} User\\
\textbf{Pre-condizione:} L'utente ha inserito dati scorretti nella schermata di creazione di un oggetto\\
\textbf{Post-condizione: } Visualizzazione Errore\\
\textbf{Scenario Principale:} Il sistema mostra a schermo un messaggio contenente le specifiche dell'errore di inserimento\\

\vspace{0.5cm}

\Large\textbf{}\\
\Large\textbf{Use Case 7 Ricerca Oggetto tramite Id} \\
\vspace{0.5cm}
\begin{figure}[h]
  \centering
  \includegraphics[width=0.8\textwidth]{UseCasesImages/ObjResearchId.drawio.png}
\end{figure}

\large\textbf{} \\
\textbf{Attori:} User\\
\textbf{Pre-condizione:}  Magazzino correttamente istanziato\\
\textbf{Post-condizione: } Visualizzazione risultato ricerca\\
\textbf{Scenario Principale:} \\
\begin{itemize}
  \item L'utente immette un codice identificativo
  \item Il sistema visualizza l'oggetto corrispondente, se trovato, oppure comunica l'assenza di un risultato
\end{itemize}.

\vspace{0.5cm}

\Large\textbf{}\\
\Large\textbf{Use Case 8 Selezione Oggetto} \\
\begin{figure}[h]
  \centering
 \includegraphics[width=0.8\textwidth]{UseCasesImages/ObjSelection.drawio.png}
\end{figure}

\vspace{0.5cm}

\large\textbf{} \\
\textbf{Attori:} User\\
\textbf{Pre-condizione:} Sistema correttamente istanziato \\
\textbf{Post-condizione: } Selezione di un oggetto\\
\textbf{Scenario Principale:}  
 L'utente seleziona un oggetto attraverso [UC9] o [UC10]
\textbf{Generalizzazioni:} 
\begin{itemize}
    \item UC9 Selezione Statica Oggetto
    \item UC10 Selezione 3D Oggetto
\end{itemize}

\Large\textbf{}\\
\Large\textbf{Use Case 8.1 Selezione Statica Oggetto} \\
%\begin{figure}[h]
%  \centering
%  \includegraphics[width=0.8\textwidth]{UseCasesImages/StaticSel.png}
%\end{figure}

\vspace{0.5cm}

\large\textbf{} \\
\textbf{Attori:} User\\
\textbf{Pre-condizione:} L'utente ha effettuato [UC7] con risultato non nullo \\
\textbf{Post-condizione: }  Selezione di un oggetto con conseguente evidenziazione delle correnti posizioni nell'ambiente tridimensionale\\
\textbf{Scenario Principale:}
\begin{itemize}
    \item L'utente seleziona un oggetto
    \item L'oggetto viene evidenziato nell'ambiente 3D ove presente
\end{itemize}

\vspace{0.5cm}

\Large\textbf{}\\
\Large\textbf{Use Case 8.2 Selezione 3D Oggetto} \\
%\begin{figure}[h]
%  \centering
%  \includegraphics[width=0.8\textwidth]{UseCasesImages/3DSel.png}
%\end{figure}

\large\textbf{} \\
\textbf{Attori:} User\\
\textbf{Pre-condizione:} Sistema correttamente istanziato \\
\textbf{Post-condizione: } Selezione di un oggetto\\
\textbf{Scenario Principale:}
Attraverso la navigazione nello spazio tridimensionale l'utente seleziona un oggetto

\vspace{0.5cm}


\Large\textbf{}\\
\Large\textbf{Use Case 9 Inserimento Oggetto} \\
\begin{figure}[h]
 \centering
  \includegraphics[width=0.8\textwidth]{UseCasesImages/ObjInsert.drawio.png}
\end{figure}

\vspace{0.5cm}

\large\textbf{} \\
\textbf{Attori:} User\\
\textbf{Pre-condizione:} L'utente ha creato ([UC6]) o selezionato ([UC7] un oggetto non ancora inserito  \\
\textbf{Post-condizione: } L'oggetto é stato correttamente inserito o viene visualizzato un messaggio di errore\\
\textbf{Scenario Principale:} 
\begin{itemize}
    \item L'utente seleziona la scaffalatura dove vuole inserire l'oggetto 
    \item Il sistema verifica la disponibilitá della scaffalatura
    \item L'oggetto viene inserito o l'inserimento viene rifiutato con conseguente schermata di errore
\end{itemize}
\textbf{Estensione} UC11 Errore Movimento Oggetto


\large\textbf{} \\
\textbf{Attori:} User\\
\textbf{Pre-condizione:} L'utente sta creando un oggetto \\
\textbf{Post-condizione: } L'utente ha inserito l'id di un oggetto\\
\textbf{Scenario Principale:}\\
\begin{itemize}
  \item L'utente inserisce un codice identificativo dell'oggetto
  \item L'oggetto viene creato
\end{itemize}.\\
\textbf{Estensioni: } UC6.1 Errore Creazione Oggetto\\

\vspace{0.5cm}


\vspace{0.5cm}

\Large\textbf{}\\
\Large\textbf{Use Case 10 Spostamento Oggetto} \\
\begin{figure}[h]
 \centering
 \includegraphics[width=0.8\textwidth]{UseCasesImages/ObjMovement.drawio.png}
\end{figure}

\vspace{0.5cm}

\large\textbf{} \\
\textbf{Attori:} User\\
\textbf{Pre-condizione:} L'utente ha selezionato un oggetto \\
\textbf{Post-condizione: } L'oggetto é stato correttamente spostato o viene visualizzato un messaggio di errore\\
\textbf{Scenario Principale:} 
\begin{itemize}
    \item L'utente sposta l'oggetto mediante [UC10.1] o [UC10.2]
    \item Il sistema verifica la disponibilitá della scaffalatura
    \item L'oggetto viene spostato o il movimento viene rifiutato con conseguente schermata di errore
\end{itemize}
\textbf{Generalizzazioni:}
\begin{itemize}
    \item UC10.1 Spostamento Oggetto tramite coordinate
    \item UC10.2 Spostamento Oggetto tramite cursore/touch
\end{itemize}
\textbf{Estensione} UC11 Errore Movimento Oggetto

\vspace{0.5cm}

\Large\textbf{}\\
\Large\textbf{Use Case 10.1 Spostamento Oggetto tramite coordinate} \\
%\begin{figure}[h]
%  \centering
%  \includegraphics[width=0.8\textwidth]{UseCasesImages/Spostamento.png}
%\end{figure}

\vspace{0.5cm}

\large\textbf{} \\
\textbf{Attori:} User\\
\textbf{Pre-condizione:} L'utente ha selezionato un oggetto \\
\textbf{Post-condizione: } L'oggetto é stato correttamente spostato o viene visualizzato un messaggio di errore\\
\textbf{Scenario Principale:} 
\begin{itemize}
    \item L'utente inserisce le coordinate della scaffalatura in cui vuole spostare l'oggetto
    \item Il sistema verifica la disponibilitá della scaffalatura
    \item L'oggetto viene spostato o il movimento viene rifiutato con conseguente schermata di errore
\end{itemize}

\vspace{0.5cm}

\Large\textbf{}\\
\Large\textbf{Use Case 10.2 Spostamento Oggetto tramite cursore/touch} \\
%\begin{figure}[h]
%  \centering
%  \includegraphics[width=0.8\textwidth]{UseCasesImages/Spostamento.png}
%\end{figure}

\vspace{0.5cm}

\large\textbf{} \\
\textbf{Attori:} User\\
\textbf{Pre-condizione:} L'utente ha selezionato un oggetto \\
\textbf{Post-condizione: } L'oggetto é stato correttamente spostato o viene visualizzato un messaggio di errore\\
\textbf{Scenario Principale:} 
\begin{itemize}
    \item L'utente trascina tramite cursore/touch l'oggetto verso la scaffalatua in cui lo vuole spostare
    \item Il sistema verifica la disponibilitá della scaffalatura
    \item L'oggetto viene spostato o il movimento viene rifiutato con conseguente schermata di errore
\end{itemize}

\vspace{0.5cm}

\Large\textbf{}\\
\Large\textbf{Use Case 11 Errore Movimento Oggetto} \\

\vspace{0.5cm}

\large\textbf{} \\
\textbf{Attori:} Sistema\\
\textbf{Pre-condizione:} L'utente ha richiesto un movimento non possibile di un oggetto  \\
\textbf{Post-condizione: } Viene visualizzato un messaggio di errore\\
\textbf{Scenario Principale:} 
Il sistema visualizza un errore conseguente alla richiesta di un movimento non possibile di un oggetto

\vspace{0.5cm}

\Large\textbf{}\\
\Large\textbf{Use Case 12 Eliminazione Oggetto} \\
\begin{figure}[h]
  \centering
  \includegraphics[width=0.8\textwidth]{UseCasesImages/ObjDelete.drawio.png}
\end{figure}
\vspace{0.5cm}

\large\textbf{} \\
\textbf{Attori:} User\\
\textbf{Pre-condizione:} L'utente ha selezionato un oggetto  \\
\textbf{Post-condizione: } L'oggetto viene eliminato da ogni scaffalatura in cui è presente\\
\textbf{Scenario Principale:} 
\begin{itemize}
    \item L'utente elimina l' oggetto selezionato
    \item Il sistema si occupa di rimuoverlo da tutte le scaffalature in cui è presente 
\end{itemize}


{\color{red}{\textbf{Domanda:} Esistono diverse istanze di un oggetto oppure ogni oggeto è univoco?}}

\vspace{0.5cm}


% Struttura dello Use Case - copialo per crearne un altro

%\Large\textbf{}\\
%\Large\textbf{Use Case 1 : Inizializzazione Ambiente} \\
%\vspace{0.5cm}
%\begin{figure}[h]
%  \centering
%  \includegraphics[width=0.8\textwidth]{UseCasesImages/UC1.png}
%\end{figure}

%\large\textbf{} \\
%\textbf{Attori:} \\
%\textbf{Pre-condizione:} \\
%\textbf{Post-condizione:} \\
%\textbf{Scenario Principale:} \\
%\textbf{Generalizzazioni:} \\

%\vspace{0.5cm}

% Fine use case

\section{REQUISITI}\label{sec:requisiti}
La seguente sezione si occupa di \textbf{Descrivere}, \textbf{Classificare}, \textbf{Tracciare} e \textbf{Codificare}
i requisiti individuati dal gruppo. \\
Una prima macrodivisione classificativa sarà effettuata tra requisiti:
\begin{itemize}
    \item \textbf{Funzionali: }indicano le funzionalità offerta all'utente
    \item \textbf{di Interfaccia: }indicano i requisiti offerti dall'interfaccia
    \item \textbf{di Database: }indicano i requisiti di interazione con la base di dati
    \item \textbf{Di Qualità: } garantiscono la qualità del prodotto
    \item \textbf{Di Sistema: }indicano le specifiche tecniche a supporto dell'utiluzzo del prodotto
    \item \textbf{Di Vincolo: } indicano i vincoli logici del sistema
\end{itemize}
i quali meritano una suddivisione in sottosezioni apposite.\\
Ci si occuperà inoltre di distinguerà tra requisiti \textbf{Obbligatori}, \textbf{Desiderabili} ed \textbf{Opzionali}.\\ 
Di particolare importanza è inoltre il tracciamento dell'origine dei requisiti, in cui si esamina il perimetro di origine dei tali: una prima suddivisione sará effettuata tra requisiti \textbf{espliciti} ed \textbf{impliciti}, a cui sarà aggiunta una citazione del documento o la modalità dalla quale lo si è ricavatot, con eventuali riferimenti ai Casi d'uso individuati nella prima parte di questo docuemnto.\\
Sulla base delle specifiche appena discusse, ad ogni documento verrà assegnato un codice che segue la seguente regola:

\begin{figure}[h]
 \centering
  \includegraphics[width=0.5\textwidth]{UseCasesImages/requirementsCod.png}
\end{figure}

\subsection{Requisiti funzionali}

\setlength{\arrayrulewidth}{0.5mm}
\renewcommand{\arraystretch}{2.5}

\begin{table}[h]
\centering
\rowcolors{3}{gray!10!white!100}{gray!0!white!100}
\begin{tabular}{ |>{\centering\arraybackslash}p{1cm}|>{\centering\arraybackslash}p{9cm}|>{\centering\arraybackslash}p{5cm}| }
\hline
\multicolumn{3}{|c|}{\Large Requisiti funzionali} \\
\hline
Codice & Descrizione & Tracciamento\\
\hline
RFD1 & L'utente può caricare un file per inizializzare l'ambiente& Esplicito, Capitolato\\
RFO1 & L'utente può creare un ambiente da zero & Esplicito, Capitolato \\
RFO2 & L'utente può creare delle scaffalature & Esplicito, Capitolato\\
RFO3 & L'utente può creare un oggetto & Esplicito, Capitolato\\
RFO4 & L'utente può navigare nello spazio tridimensionale attraverso il mouse& Esplicito, Capitolato\\
RFO5 & L'utente può navigare nello spazio tridimensionale attraverso la tastiera& Esplicito, Capitolato\\
RFO6 & L'utente può scegliere la dimensione delle scaffalature & Implicito, Analisi dei requisiti\\
RFO7 & L'utente può modificare le scaffalature create & Esplicito, Capitolato\\
\hline
\end{tabular}
\end{table}

\newpage

\begin{table}[h]
\centering
\rowcolors{3}{gray!10!white!100}{gray!0!white!100}
\begin{tabular}{ |>{\centering\arraybackslash}p{1cm}|>{\centering\arraybackslash}p{9cm}|>{\centering\arraybackslash}p{5cm}| }
\hline
\multicolumn{3}{|c|}{\Large Requisiti funzionali} \\
\hline
Codice & Descrizione & Tracciamento\\
\hline
RFO8 & L'utente può eliminare le scaffalature create & Implicito, Analisi dei requisiti\\
RFO8 & L'utente può eliminare un oggetto creato & Implicito, Analisi dei requisiti\\
RFO9 & L'utente può selezionare una scaffalatura nello spazio tridimensionale & Esplicito, Capitolato\\
RFO10 & L'utente può selezionare un oggetto nello spazio tridimensionale & Esplicito, Capitolato\\
RFD2 & L'utente può ricercare un oggetto tramite id & Esplicito, Capitolato\\
RFO11 & L'utente può inserire un oggetto all'interno di una scaffalatura& Esplicito, Capitolato\\
RFO12 & L'utente può richiedere lo spostamento di un oggetto & Esplicito, Capitolato\\
RFO13 & L'utente può rimuovere un oggetto da una scaffalatura & Implicito, Analisi dei requisiti\\
RFO14 & Il sistema deve verificare la disponibilità della scaffalatura target alla richiesta di uno spostamento & Implicito, Dialogo con il Proponente\\
RFO14 & Il sistema delega ad un meccanismo terzo la decisione finale sull'accettazione di un movimento & Implicito, Dialogo con il Proponente\\
\hline
\end{tabular}
\end{table}



\subsection{Requisiti di Interfaccia}

\setlength{\arrayrulewidth}{0.5mm}
\renewcommand{\arraystretch}{2.5}

\begin{table}[h]
\centering
\rowcolors{3}{gray!10!white!100}{gray!0!white!100}
\begin{tabular}{ |>{\centering\arraybackslash}p{1cm}|>{\centering\arraybackslash}p{9cm}|>{\centering\arraybackslash}p{5cm}| }
\hline
\multicolumn{3}{|c|}{\Large Requisiti di Interfaccia} \\
\hline
Codice & Descrizione & Tracciamento\\
\hline
RID1 & L'utente può scegliere come inizializzare il suo ambiente & Esplicito, Capitolato\\
RIO1 & Sviluppo di un'are gestionale nella quale effettuare le ricerche per Id & Implicito, Analisi dei Requisiti\\
RIX1 & Un oggetto selezionato viene evidenziato nell'ambiente tridimensionale & Esplicito, Capitolato\\
\hline
\end{tabular}
\end{table}

\newpage

\subsection{Requisiti di Database}

\setlength{\arrayrulewidth}{0.5mm}
\renewcommand{\arraystretch}{2.5}

\begin{table}[h]
\centering
\rowcolors{3}{gray!10!white!100}{gray!0!white!100}
\begin{tabular}{ |>{\centering\arraybackslash}p{1cm}|>{\centering\arraybackslash}p{9cm}|>{\centering\arraybackslash}p{5cm}| }
\hline
\multicolumn{3}{|c|}{\Large Requisiti di Database} \\
\hline
Codice & Descrizione & Tracciamento\\
\hline
RDO1 & Il Database deve memorizzare le scaffalatura tramite coordinate&Esplicito, Capitolato\\
RDD1 & Il Database deve memorizzare la posizione degli oggetti ed il loro nome & Implicito, Analisi dei Requisiti\\
\hline
\end{tabular}
\end{table}
\newpage

\subsection{Requisiti di Qualità}

\setlength{\arrayrulewidth}{0.5mm}
\renewcommand{\arraystretch}{2.5}

\begin{table}[h]
\centering
\rowcolors{3}{gray!10!white!100}{gray!0!white!100}
\begin{tabular}{ |>{\centering\arraybackslash}p{1cm}|>{\centering\arraybackslash}p{9cm}|>{\centering\arraybackslash}p{5cm}| }
\hline
\multicolumn{3}{|c|}{\Large Requisiti di Qualità} \\
\hline
Codice & Descrizione & Tracciamento\\
\hline
\hline
\end{tabular}
\end{table}
\newpage

\subsection{Requisiti di Sistema}

\setlength{\arrayrulewidth}{0.5mm}
\renewcommand{\arraystretch}{2.5}

\begin{table}[h]
\centering
\rowcolors{3}{gray!10!white!100}{gray!0!white!100}
\begin{tabular}{ |>{\centering\arraybackslash}p{1cm}|>{\centering\arraybackslash}p{9cm}|>{\centering\arraybackslash}p{5cm}| }
\hline
\multicolumn{3}{|c|}{\Large Requisiti di Sistema} \\
\hline
Codice & Descrizione & Tracciamento\\
\hline
\hline
\end{tabular}
\end{table}

\section{Riepilogo}
Di seguito si propone una suddivisione dei requisiti per rilevanza, conteggiandone il numero:

\begin{center}
\begin{tabular}{ c c c c }
 Obbligatori & Desiderabili & Opzionali & Totale \\ 
 17 & 4 & 1 & \textcolor{red}{23}
\end{tabular}
\end{center}