\section*{\Large Registro delle Modifiche}
    \begin{table}[h]
        \centering
        \rowcolors{2}{gray!30}{white}
        \renewcommand\tabularxcolumn[1]{m{#1}} %per centrare contenuto nelle tabelle
        \renewcommand{\arraystretch}{1.5}
        \begin{tabularx}{0.98\textwidth}
            {c|c|c|c|>{\centering\arraybackslash}X}
            \rowcolor{black}
            \textbf{\color{white} Versione} & \textbf{\color{white} Data} & \textbf{\color{white} Nominativo} & \textbf{\color{white} Ruolo} & \textbf{\color{white} Descrizione} \\ 
            \hline

            %inserire versione, data, nome, ruole e cosa è stato modificato
            %più vecchio
            v0.0.1 & 28-11-2023 & Lorenzo Pasqualotto & Analista & Stesura scheletro del documento, scrittura prima versione Use Cases relativi agli oggetti \\
            v0.0.2 & 03-12-2023 & Luca Securo & Analista & Aggiunte al contenuto del documento. Adeguamento del documento per poterlo rendere leggibile in vista dell'incontro con l'azienda \\
            v0.0.3 & 14-12-2023 & Luca Securo & Analista & Aggiunte al contenuto del documento. Sistemazione indice e completamento scrittura dei punti 1 e 2 dell'indice\\
            v0.0.4 & 17-12-2023 & Lorenzo & Analista & \\
            v0.0.5 & 18-12-2023 & Luca Securo & Analista & Aggiunte al contenuto del documento. Riformulazione ed aggiunta di Use Cases\\

            %più recente
            \hline
        \end{tabularx}
    \end{table}