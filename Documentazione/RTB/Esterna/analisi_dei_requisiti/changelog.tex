\section*{\Large Registro delle Modifiche}
    \begin{table}[H]
        \centering
        \rowcolors{2}{gray!30}{white}
        \renewcommand\tabularxcolumn[1]{m{#1}} %per centrare contenuto nelle tabelle
        \renewcommand{\arraystretch}{1.5}
        \begin{tabularx}{0.98\textwidth}
            {c|c|c|c|>{\centering\arraybackslash}X}
            \rowcolor{black}
            \textbf{\color{white} Versione} & \textbf{\color{white} Data} & \textbf{\color{white} Nominativo} & \textbf{\color{white} Ruolo} & \textbf{\color{white} Descrizione} \\ 
            \hline

            %inserire versione, data, nome, ruole e cosa è stato modificato
            %più vecchio
            v0.0.1 & 28-11-2023 & Lorenzo Pasqualotto & Analista & Stesura scheletro del documento, scrittura prima versione Use Cases relativi agli oggetti. \\
            v0.0.2 & 03-12-2023 & Luca Securo & Analista & Aggiunte al contenuto del documento. Adeguamento del documento per poterlo rendere leggibile in vista dell'incontro con l'azienda. \\
            v0.0.3 & 14-12-2023 & Luca Securo & Analista & Aggiunte al contenuto del documento. Sistemazione indice e completamento scrittura dei punti 1 e 2 dell'indice.\\
            v0.0.4 & 17-12-2023 & Lorenzo Pasqualotto & Analista &  Aggiornamento Use Cases relativi agli oggetti.\\
            v0.0.5 & 18-12-2023 & Luca Securo & Analista & Aggiunte al contenuto del documento. Riformulazione ed aggiunta di Use Cases.\\
            v0.0.6 & 21-12-2023 & Lorenzo Pasqualotto & Analista & Stesura prima versione sezione \ref{sec:requisiti}.\\
            v0.0.7 & 07-01-2024 & Jessica Carretta & Analista & Formalizzazione struttura del documento. Modifica e aggiunta di contenuto alle sezioni \ref{subsec:requisiti_funzionali}, \ref{subsec:riepilogo}. Riscrittura sezione \ref{subsec:elenco_use_cases}. Scrittura delle sotto-sezioni \ref{subsec:attori}, \ref{subsec:requisiti_qualita}, \ref{subsec:requisiti_vincolo}, \ref{subsec:requisiti_prestazionali},\ref{subsec:tracciamento}.\\
            v0.0.8 & 05-02-2024 & Luca Securo & Analista & Correzione delle caption delle immagini presenti nel documento. Modifica di alcune immagini con aggiunta di commenti.\\
            v0.1.0 & 07-02-2024 & Andrea Mangolini & Verificatore & Verifica del documento nei contenuti inseriti fino alla versione v0.0.8.\\
            v1.0.0 & 08-02-2024 & Zaccaria Marangon & Responsabile & Approvazione del documento.\\
            v1.0.1 & 17-02-2024 & Andrea Mangolini & Amministratore & Correzione dei contenuti in seguito a riunione con azienda (verbale\_esterno\_160224, v1.0.0). Sezione \ref{sec:descrizione_del_prodotto}, casi d'uso UC4.2.1, UC6.3, UC13, UC27.\\
            v1.1.0 & 18-02-2024 & Jessica Carretta & Verificatore & Verifica dei contenuti aggiunti e modificati nella versione 1.0.1 del documento.\\
            v2.0.0 & 18-02-2024 & Andrea Mangolini & Responsabile & Approvazione del documento.\\
            \hline
        \end{tabularx}
    \end{table}

    \begin{table}[H]
        \centering
        \rowcolors{2}{gray!30}{white}
        \renewcommand\tabularxcolumn[1]{m{#1}} %per centrare contenuto nelle tabelle
        \renewcommand{\arraystretch}{1.5}
        \begin{tabularx}{0.98\textwidth}
            {c|c|c|c|>{\centering\arraybackslash}X}
            \rowcolor{black}
            \textbf{\color{white} Versione} & \textbf{\color{white} Data} & \textbf{\color{white} Nominativo} & \textbf{\color{white} Ruolo} & \textbf{\color{white} Descrizione} \\ 
            \hline
            %inserire versione, data, nome, ruole e cosa è stato modificato
            %più vecchio
            v2.0.1 & 02-03-2024 & Lorenzo Pasqualotto & Programmatore & Modifica di UC1.2 (e relativi sottocasi) ed inserimento UC28 e sottocasi come da richiesta del proponente.\\
            v2.1.0 & 03-03-2024 & Jessica Carretta & Verificatore & Verifica dei contenuti aggiunti e modificati nella versione 2.0.1 del documento.\\
            v3.0.0 & 04-03-2024 & Zaccaria Marangon & Responsabile & Approvazione del documento.\\
            \hline
        \end{tabularx}
    \end{table}