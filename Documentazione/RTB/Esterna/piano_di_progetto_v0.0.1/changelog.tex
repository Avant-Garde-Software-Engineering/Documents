\section*{\Large Registro delle Modifiche}
    \begin{table}[h]
        \centering
        \rowcolors{2}{gray!30}{white}
        \renewcommand\tabularxcolumn[1]{m{#1}} %per centrare contenuto nelle tabelle
        \renewcommand{\arraystretch}{1.5}
        \begin{tabularx}{0.98\textwidth}
            {c|c|c|c|>{\centering\arraybackslash}X}
            \rowcolor{black}
            \textbf{\color{white} Versione} & \textbf{\color{white} Data} & \textbf{\color{white} Nominativo} & \textbf{\color{white} Ruolo} & \textbf{\color{white} Descrizione} \\ 
            \hline

            %inserire versione, data, nome, ruole e cosa è stato modificato
            %più vecchio
            v0.0.1 & 08-12-23 & Jessica Carretta & Amministratore & Creazione della struttura del documento, scrittura delle sezioni \ref{sec:scopo_del_documento}, \ref{sec:il_progetto}, \ref{sec:analisi_rischi}, \ref{sec:riferimenti_esterni} e della sotto-sezione \ref{sec:preventivo:totale}. \\
            
            v0.0.2 & 12-12-23 & Giulio Biscontin & Amministratore & Scrittura della sezione \ref{sec:modello_sviluppo}. \\
            
            v0.0.3 & 20-12-23 & Giulio Biscontin & Amministratore & Inizio stesura della sezione \ref{sec:pianificazione}. \\
            
            v0.0.4 & 28-12-23 & Giulio Biscontin & Amministratore & Completamento stesura della sezione \ref{sec:pianificazione}. \\
            
            v0.0.5 & 04-01-24 & Giulio Biscontin & Amministratore & Stesura della sezione \ref{sec:preventivo}. \\

            v0.1.0 & 07-01-24 & Andrea Mangolini & Verificatore & Verifica dei contenuti nelle sezioni introdotte dalla versione v0.0.1 alla versione v0.0.5. \\

            v0.1.1 & 18-01-24 & Giulio Biscontin & Amministratore & Aggiunta diagrammi di Gantt e stesura sezione \ref{sec:consuntivo:analisi}. \\

            v0.1.2 & 28-02-24 & Giulio Biscontin & Amministratore & Stesura sezione \ref{sec:consuntivo:progRTB}. \\
            
            v0.1.2 & 23-03-24 & Giulio Biscontin & Amministratore & Revisione della sezione \ref{sec:consuntivo:progRTB} derivata dalla revisione con il professor Cardin. \\

            %più recente
            \hline
        \end{tabularx}
    \end{table}