\section{Scopo dell'incontro} \label{sec:scopo}
La riunione si è svolta con l'obiettivo di aggiornare l'azienda \nomeAzienda \space sull'avanzamento del PoC, mostrando le nuove funzionalità aggiunte e le modifiche apportate in seguito alla riunione precedente, per assicurarsi che lo sviluppo sia coerente con le richieste del proponente. 

\section{Argomenti} \label{sec:argomenti}
\noindent Si riportano di seguito gli argomenti trattati in dettaglio.

\subsection{Dimostrazione del PoC}
L'azienda ha visionato lo stato di avanzamento del PoC tramite dimostrazione visiva del suo funzionamento. In particolare, le nuove funzionalità aggiunte rispetto alla versione precedente sono:
\begin{itemize}
    \item Possibilità di scegliere le dimensioni del bin per ogni scaffalatura;
    \item Inserimento dei prodotti in 3D e libreria;
    \item Inserimento prodotti in una scaffalatura (con scelta casuale del bin);
    \item Cancellazione dei prodotti da libreria;
    \item Rimozione della scaffalatura da libreria (se vuota);
    \item Rotazione delle scaffalature;
    \item Richiesta di spostamento di un prodotto verso una nuova scaffalatura (con specifica del bin) tramite apposito form.    
\end{itemize}
\noindent Inoltre, sono state aggiunte delle texture agli elementi 3D per differenziarli meglio tra di loro e renderli più realistici. In particolare, nella scaffalatura la texture serve per dare un'idea di base su come sono strutturati i bin al suo interno.

\subsection{Discussione sul PoC}
In seguito alla dimostrazione, il proponente ha confermato che le funzionalità introdotte rispettano i requisiti.
\subsubsection{Miglioramento delle funzionalità presenti}
Il proponente ha anche richiesto al team di sviluppare ulteriormente alcune funzionalità già presenti, in particolare:
\begin{itemize}
    \item Possibilità di effettuare la richiesta di spostamento anche tramite visualizzazione 3D;
    \item Possibilità di interagire ed effettuare operazioni sui prodotti già inseriti nei bin;
    \item Migliorare le texture applicate agli elementi per renderle più coerenti con l'aspetto di un magazzino;
    \item Possibilità di inserimento di un prodotto in un bin specifico (piuttosto che casuale).
\end{itemize}
\subsubsection{Aggiunta di nuove funzionalità}
Il proponente ha inoltre proposto di aggiungere delle nuove funzionalità al prodotto finale, in particolare:
\begin{itemize}    
    \item Possibilità di scegliere una planimetria per il magazzino che non sia solo di dimensione rettangolare/quadrata;
    \item Possibilità di inserimento e visualizzazione di aree specifiche del magazzino (entrate, uscite, aree di transito muletti).
\end{itemize}
