\section{Scopo dell'incontro} \label{sec:scopo}
La riunione si è svolta con l'obiettivo di aggiornare l'azienda \nomeAzienda \space sullo stato di avanzamento del progetto in seguito al semaforo rosso dopo il primo incontro alla revisione RTB. Il gruppo ha selezionato alcune possibili tecnologie e framework da implementare nel progetto e ne ha discusso con l'azienda durante l'incontro.

\section{Tecnologie proposte} \label{sec:argomenti}
\noindent Le tecnologie proposte per coprire le mancanze emerse dalla prima revisione RTB sono le seguenti:
\subsection{Lato Front-end}
\begin{itemize}
    \item \textit{React}, od eventualmente \textit{Angular}
    \item \textit{React-three-fiber} per l'eventuale integrazione tra \textit{Three.js} e \textit{React}
    \item \textit{Next.js}
\end{itemize}
\subsection{Lato Back-end}
\begin{itemize}
    \item \textit{Next.js}, od eventualmente \textit{Express.js}
    \item \textit{Node.js}
\end{itemize}

\subsection{Discussione sulle tecnologie}
Sulla scelta delle tecnologie l'azienda non ha posto vincoli e non ha respinto nessuna tra quelle proposte: ha dato al gruppo libera scelta, a seconda di quali ritengono essere le opzioni migliori.
\par Il gruppo ha espresso una preferenza su \textit{React} vista la facile integrazione con \textit{Three.js} attraverso \textit{React-three-fiber}, che semplifica la gestione delle componenti grafiche. Mentre, dal lato backend, si è proposto di provare \textit{Next.js}, che non solo è basato su \textit{React}, ma permette uno sviluppo ``full-stack" dell'applicazione web, perché non solo è in grado di gestire operazioni lato server, ma anche lato client. Inoltre, \textit{Next.js} offre tutta una serie di ottimizzazioni che permettono allo sviluppatore di concentrarsi più sulla scrittura della web app vera e propria e meno sulla gestione delle configurazioni.
\par Eventualmente, se il gruppo dovesse decidere di utilizzare \textit{Angular}, il rappresentante esterno si è reso disponibile per aiutare, essendo una tecnologia da lui utilizzata di frequente.
\par Il gruppo ha valutato inoltre la possibilità di testare alcune tecnologie di deploy: un esempio di tecnologia potrebbe essere \textit{Docker}. Anche su questo argomento l'azienda non pone vincoli sulla scelta, in particolare ritiene sufficiente anche l'utilizzo della piattaforma \textit{Github} per pubblicare il codice sorgente, con inclusi i comandi necessari a far funzionare l'applicazione.