\section{Scopo dell'incontro} \label{sec:scopo}
La riunione si è svolta con l'obiettivo di aggiornare l'azienda \nomeAzienda \space sullo stato di avanzamento del progetto in seguito al semaforo rosso dopo il primo incontro alla revisione RTB. Il gruppo ha selezionato alcune possibili tecnologie e framework da implementare nel progetto e ne ha discusso con l'azienda durante l'incontro.

\section{Tecnologie proposte} \label{sec:argomenti}
\noindent Le tecnologie proposte per coprire le mancanze emerse dalla prima revisione RTB sono le seguenti:
\subsection{Framework e librerie per il front-end}
\begin{itemize}
    \item \textit{React}, in alternativa \textit{Angular}
    \item \textit{React-three-fiber} per l'eventuale integrazione tra \textit{Three.js} e \textit{React}
    \item \textit{Next.js}
\end{itemize}
\subsection{Framework e strumenti per il back-end}
\begin{itemize}
    \item \textit{Next.js}, in alternativa \textit{Express.js}
    \item \textit{Node.js}
\end{itemize}

\subsection{Discussione sulle tecnologie}
Sulla scelta delle tecnologie l'azienda non ha posto vincoli e non ha respinto nessuna tra le proposte presentate. Il proponente ha dato al gruppo libera scelta, 
a seconda di quali ritengono essere le opzioni migliori in relazione al progetto e all'esperienza personale dei componenti del gruppo.
\par Lato front-end, il gruppo ha espresso preferenza sull'utilizzo della libreria \textit{React}, che semplifica la gestione delle componenti grafiche vista la 
facile integrazione con \textit{Three.js} attraverso \textit{React-three-fiber}. \\
 Il gruppo ha inoltre proposto di utilizzare il framework \textit{Next.js} che, oltre ad essere basato su \textit{React}, permette uno sviluppo "full-stack" 
 dell'applicazione web, in quanto riesce a gestire operazioni sia lato server che lato client. Inoltre, \textit{Next.js} offre tutta una serie di ottimizzazioni 
 che permettono allo sviluppatore di concentrarsi più sulla scrittura della web app vera e propria e meno sulla gestione delle configurazioni.
\par Eventualmente, se il gruppo dovesse decidere di utilizzare il framework \textit{Angular}, il rappresentante esterno si è reso disponibile in caso di dubbi o necessità, 
essendo una tecnologia utilizzata di frequente dall'azienda proponente.
\par Il gruppo ha valutato inoltre la possibilità di testare alcune tecnologie di deploy: un esempio di tecnologia potrebbe essere \textit{Docker}. Anche su questo argomento
 il proponente non pone vincoli sulla scelta. In particolare ritiene sufficiente anche l'utilizzo della piattaforma \textit{Github} per pubblicare il codice sorgente, con 
 inclusi i comandi necessari per poter eseguire l'applicazione.
\par Per concludere, sarà quindi onere del gruppo testare le varie tecnologie proposte al fine di verificarne la compatibilità con il progetto da realizzare. 
Non sono stati posti veti o particolari richieste da parte del proponente.