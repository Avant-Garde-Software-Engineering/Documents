\documentclass{article}

\usepackage{multirow}
\usepackage{graphicx}
\usepackage{xcolor}
\usepackage{hyperref}
\usepackage{geometry}
\usepackage{fancyhdr}
\usepackage{tabularx}
\usepackage{colortbl}
\usepackage{array}
\usepackage{makecell}
\usepackage{comment}

\hypersetup{colorlinks=true, linkcolor=blue, filecolor=magenta, urlcolor=blue}
\usepackage{titlesec}
\usepackage{float}

\newcommand\ddfrac[2]{\frac{\displaystyle #1}{\displaystyle #2}}

\setcounter{secnumdepth}{4}
\setcounter{tocdepth}{4}
\titleformat{\paragraph}
{\normalfont\normalsize\bfseries}{\theparagraph}{1em}{}
\titlespacing*{\paragraph}
{0pt}{3.25ex plus 1ex minus .2ex}{1.5ex plus .2ex}

\newenvironment{beginningnote}
  {\clearpage
   \thispagestyle{empty}% no header and footer
   \vspace*{\stretch{1}}% some space at the top
   \section*{Note}
  }
  {\par % end the paragraph
   \vspace{\stretch{3}} % space at bottom is three times that at the top
   \clearpage           % finish off the page
  }

\def \titoloDocumento {PIANO DI QUALIFICA}    %modificare con il titolo del documento

\geometry{a4paper,total={160mm,240mm},top=30mm,}
\begin{document}
    \newgeometry{top=45mm,bottom=45mm,right=45mm,left=45mm}
    \begin{titlepage}
        \begin{center}
            \includegraphics[width=\textwidth]{logo+scritta.png}

            \vspace{0.5cm}
            {\huge \textbf{Il gruppo Avant-Garde}}

            \vspace{0.4cm}
            \large{sweavantgarde@gmail.com}

            \vspace{1.5cm}
            \hrulefill\\
            \vspace{0.2cm}
            
            \textbf{\titoloDocumento}\\
            \vspace{0.1cm}
            \hrulefill

            \vfill
            Informazioni sul documento:\\
            \vspace{0.3cm}
                \begin{tabular}{ r | l }
                    \textbf{Versione} & 0.0.6\\ %inserire la versione più recente prensente nel changelog
                    \textbf{Approvazione} & Nome Cognome\\ %inserire nome e cognome responsabile
                    \textbf{Redazione} & Lorenzo Pasqualotto, Luca Securo\\ %inserire nomi e cognomi redattori  
                    \textbf{Verifica} & Nome Cognome\\ %inserire nomi e cognomi verificatori
                    \textbf{Uso} & Esterno\\ %inserire se interno o esterno
                \end{tabular}
        \end{center}
    \end{titlepage}
\restoregeometry

    \pagestyle{fancy}
    \lhead{\includegraphics[width=2cm]{logo+scritta.png}}   
    \rhead{\titoloDocumento}
    \headsep = 15mm

    \section*{\Large Registro delle Modifiche}
    \begin{table}[h]
        \centering
        \rowcolors{2}{gray!30}{white}
        \renewcommand\tabularxcolumn[1]{m{#1}} %per centrare contenuto nelle tabelle
        \renewcommand{\arraystretch}{1.5}
        \begin{tabularx}{0.98\textwidth}
            {c|c|c|c|>{\centering\arraybackslash}X}
            \rowcolor{black}
            \textbf{\color{white} Versione} & \textbf{\color{white} Data} & \textbf{\color{white} Nominativo} & \textbf{\color{white} Ruolo} & \textbf{\color{white} Descrizione} \\ 
            \hline

            %inserire versione, data, nome, ruole e cosa è stato modificato
            %più vecchio
            v0.0.1 & 08-12-23 & Jessica Carretta & Amministratore & Creazione della struttura del documento, scrittura delle sezioni \ref{sec:scopo_del_documento}, \ref{sec:il_progetto}, \ref{sec:analisi_rischi}, \ref{sec:riferimenti_esterni} e della sotto-sezione \ref{sec:preventivo:totale}. \\
            
            v0.0.2 & 12-12-23 & Giulio Biscontin & Amministratore & Scrittura della sezione \ref{sec:modello_sviluppo}. \\
            
            v0.0.3 & 20-12-23 & Giulio Biscontin & Amministratore & Inizio stesura della sezione \ref{sec:pianificazione}. \\
            
            v0.0.4 & 28-12-23 & Giulio Biscontin & Amministratore & Completamento stesura della sezione \ref{sec:pianificazione}. \\
            
            v0.0.5 & 04-01-24 & Giulio Biscontin & Amministratore & Stesura della sezione \ref{sec:preventivo}. \\

            v0.1.0 & 07-01-24 & Andrea Mangolini & Verificatore & Verifica dei contenuti nelle sezioni introdotte dalla versione v0.0.1 alla versione v0.0.5. \\

            v0.1.1 & 18-01-24 & Giulio Biscontin & Amministratore & Aggiunta diagrammi di Gantt e stesura sezione \ref{sec:consuntivo:analisi}. \\

            v0.1.2 & 28-02-24 & Giulio Biscontin & Amministratore & Stesura sezione \ref{sec:consuntivo:progRTB}. \\
            
            v0.1.2 & 23-03-24 & Giulio Biscontin & Amministratore & Revisione della sezione \ref{sec:consuntivo:progRTB} derivata dalla revisione con il professor Cardin. \\

            %più recente
            \hline
        \end{tabularx}
    \end{table}
    \newpage
    {
    \renewcommand{\contentsname}{Indice}    %rimuovere se non si vuole l'indice
    \hypersetup{linkcolor=black}
    \tableofcontents    %rimuovere se non si vuole l'indice
    \newpage    %rimuovere se non si vuole l'indice
    }

    \section{Argomenti} \label{sec:argomenti}
E' stata indetta una riunione di emergenza a seguito del non superamento della prima parte della revisione RTB con il professor Riccardo Cardin.
Il motivo del fallimento è stata l'inadeguatezza e mancanza di alcune scelte tecnologiche necessarie allo sviluppo del progetto.
\\

\noindent A tal proposito, il gruppo si è impegnato ad esplorare i seguenti framework in modo da poter procedere ad un refactoring del PoC:
\begin{itemize}
    \item \textbf{Node.js};
    \item \textbf{Next.js};
    \item \textbf{React.js};
    \item \textbf{React-three-fiber}.
\end{itemize}
Nella riunione sono inoltre state prese in considerazione e discusse possibili alternative, tra cui \textit{Angular}, \textit{Koa.js} e \textit{Redux}.
\\

\noindent In data 19-04-24 è previsto un nuovo incontro di aggiornamento interno al gruppo. L'obiettivo è quello di decidere in maniera definitiva le tecnologie da adottare e suddividere il lavoro tra i membri.
      
\end{document}