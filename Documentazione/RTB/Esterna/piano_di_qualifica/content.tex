\begin{beginningnote}
    Si tenga presente che alcuni termini utilizzati nel documento riportano la lettera \textbf{G} in apice, allo scopo di evidenziare le parole che assumono uno specifico significato nell'ambito del progetto. Per comprenderle in maniera corretta, si rimanda il lettore al documento ``Glossario", che contiene un elenco completo di tutte le terminologie utilizzate con relative definizioni, allo scopo di costruire un linguaggio uniforme che possa migliorare la comunicazione tra i componenti interni al gruppo e gli stakeholders esterni.   %inserire corsivo per ogni termine del glossario?
\end{beginningnote}

%unire scopo delcoumento e il progetto in un unica sezione  (e.g. introduzione) con due sottosezioni?
%%%%%%%%%%%%%%%%%%%%%%%%%%%%%%%%%%%
% SCOPO DEL DOCUMENTO
%%%%%%%%%%%%%%%%%%%%%%%%%%%%%%%%%%%

\section{Introduzione}

\subsection{Scopo del documento}\label{sec:scopo_del_documento}
\par Questo documento formale definisce come le attività di qualità saranno gestite durante il ciclo di vita del software. Include dettagli sui processi e standard da seguire per assicurare che il prodotto finale rispetti i requisiti di qualità specificati. Questo piano è fondamentale per garantire che il software sviluppato sia di alta qualità e soddisfi le aspettative dei clienti e degli stakeholder. Al suo interno conterrà misurazioni sulla qualità atte a migliorare alcune procedure se giudicate non conformi alle aspettative. Per questo motivo il documento è di natura incrementale.

%%%%%%%%%%%%%%%%%%%%%%%%%%%%%%%%%%%
% IL PROGETTO
%%%%%%%%%%%%%%%%%%%%%%%%%%%%%%%%%%%
\subsection{Il progetto}\label{sec:il_progetto}
\par Il progetto nasce nell'ambito dei \textbf{sistemi gestionali di magazzino}, meglio noti con il termine inglese di \textit{Warehouse Management Systems} (WMS), con l'obiettivo di risolvere una serie di problematiche derivanti dalle soluzioni tradizionali tuttora presenti sul mercato.
\par Il focus principale sarà migliorare la user experience, tramite la realizzazione di un applicativo che proponga all'utente un'interazione con il magazzino in un ambiente di lavoro 3D: questa soluzione, rispetto ai tradizionali sistemi 2D, garantirebbe una maggiore comprensione degli spazi, proponendo una visualizzazione più intuitiva e familiare del magazzino all'utente che, di conseguenza, sarà in grado di prendere decisioni organizzative più informate ed efficienti, ottimizzando i processi di logistica.
\par Per raggiungere questo obiettivo, l'ambiente di lavoro non può essere una semplice visualizzazione del magazzino. L'utente dovrà infatti poter:
\begin{itemize}
    \item Navigare l'ambiente 3D;
    \item Progettare la scaffalatura e modificarla nel tempo;
    \item Inserire, spostare e rimuovere prodotti negli scaffali.
\end{itemize}
Il progetto deve concretizzarsi nella realizzazione di una web app fruibile agli impiegati d'ufficio ed incentrata sulla visualizzazione 3D del magazzino.
\par Per visionare il capitolato\textsuperscript{G} e la documentazione del gruppo, si veda la sezione \hyperref[sec:riferimenti_esterni]{Riferimenti Esterni} del documento.

\newpage

\section{Qualità di processo}
Per assicurare la realizzazione di un software di alta qualità, è essenziale adottare metodologie rigorose e ben definite durante il ciclo di vita dello sviluppo. Queste metodologie sono misurate attraverso metriche specifiche che permettono di valutarne l'efficacia e l'efficienza.

\subsection{Progettazione}
\subsubsection{Descrizione}
La fase di progettazione è cruciale per assicurare che il software soddisfi tutti i requisiti funzionali e non funzionali. Questa fase include la definizione dell'architettura del software, la scelta delle tecnologie e la pianificazione dettagliata del processo di sviluppo.

\subsubsection{Metriche e valutazioni}
\paragraph{Completezza della Progettazione}

\begin{table}[h]
\centering
\begin{tabular}{ |>{\centering\arraybackslash}m{3cm}|>{\centering\arraybackslash}m{10cm}|>{\centering\arraybackslash}m{3cm}| }
\hline
Codice & Descrizione & Valutazione\\
\hline
MPC1CP & Misura la percentuale di elementi di progettazione (moduli, classi, funzioni) definiti rispetto al totale previsto &
\textit{Accettato}: $\geq$ 95\%
\textit{Rifiutato}: $<$ 95\% \\
\hline
\end{tabular}
\end{table}

\subsection{Verifica}
\subsubsection{Descrizione}
La verifica è il processo attraverso il quale si assicura che il software soddisfi tutti i requisiti e funzioni come previsto. Questa fase include test funzionali, test di regressione, e revisioni del codice.

\subsubsection{Metriche e valutazioni}
\paragraph{Copertura dei Test}

\begin{table}[h]
\centering
\begin{tabular}{ |>{\centering\arraybackslash}m{3cm}|>{\centering\arraybackslash}m{10cm}|>{\centering\arraybackslash}m{3cm}| }
\hline
Codice & Descrizione & Valutazione\\
\hline
MPC2CT & Percentuale di codice coperto da test automatizzati &
\textit{Accettato}: $\geq$ 80\%
\textit{Rifiutato}: $<$ 80\%\\
\hline
\end{tabular}
\end{table}

\subsection{Budget}
\subsubsection{Descrizione}
Lo sviluppo del progetto deve tener conto di limiti di costo, definiti tra il team e lo stakeholder in fase di candidatura. E'necessario tenere traccia del consumo del budget pattuito.

\subsubsection{Metriche e valutazioni}
\paragraph{Consumo Budget rispetto a Percentuale Completamento Software}

\begin{table}[h]
\centering
\begin{tabular}{ |>{\centering\arraybackslash}m{3cm}|>{\centering\arraybackslash}m{10cm}|>{\centering\arraybackslash}m{3cm}| }
\hline
Codice & Descrizione & Valutazione\\
\hline
MPC3CBCS & indica quale percentuale di budget è stato consumato in rapporto alla percentuale di completamento del software & Si faccia attenzione ad un valore $>$1 nelle fasi finali\\
\hline
\end{tabular}
\end{table}

\newpage

\paragraph{Consumo Budget rispetto ad Aspettative}

\begin{table}[h]
\centering
\begin{tabular}{ |>{\centering\arraybackslash}m{3cm}|>{\centering\arraybackslash}m{10cm}|>{\centering\arraybackslash}m{3cm}| }
\hline
Codice & Descrizione & Valutazione\\
\hline
MPC4CBA& indica la differenza tra budget che ci si aspettava di consumare e il budget consumato  & si faccia attenzione quando il valore risulta negativo\\
\hline
\end{tabular}
\end{table}

\subsection{Documentazione}
\subsubsection{Descrizione}
La documentazione è fondamentale per garantire la comprensibilità e la trasferibilità delle informazioni relative al software. Una buona documentazione include specifiche tecniche, manuali utente e documenti di design.

\subsubsection{Metriche e valutazioni}
\paragraph{Completezza della Documentazione}

\begin{table}[h]
\centering
\begin{tabular}{ |>{\centering\arraybackslash}m{3cm}|>{\centering\arraybackslash}m{10cm}|>{\centering\arraybackslash}m{3cm}| }
\hline
Codice & Descrizione & Valutazione\\
\hline
MPC4CD & Ogni membro del team si dedicherà a valutare la completezza e la disambiguità della documentazione prodotta, decidendo se promuoverla o meno &
\textit{Accettato}: Tutti i membri del team promuovono la documentazione
\textit{Rifiutato}: Almeno un membro del team boccia la documentazione \\
\hline
\end{tabular}
\end{table}

Queste metriche e valutazioni aiuteranno a garantire che ogni fase del processo di sviluppo del software sia eseguita con i più alti standard di qualità, in conformità con gli standard ISO/IEC.

\newpage
\section{Qualità di prodotto}
La qualità di un prodotto software, secondo gli standard ISO/IEC, è un argomento complesso e multidimensionale che richiede una comprensione approfondita di vari aspetti tecnici, funzionali e di gestione. Questa sezione si occupa di definire, descrivere e valutare tramite l'utilizzo di opportune metriche gli standard rilevanti, fornendo un quadro per valutarne la qualità in termini di un insieme di caratteristiche e sottocaratteristiche.

\subsection{Funzionalità.}
\subsubsection{Descrizione} Assicura che il software fornisca tutte le funzioni necessarie per una gestione efficace ed efficiente del magazzino, migliorando l'interazione e la gestione delle risorse.\\
\subsubsection{Metriche e valutazioni}.

\paragraph{Tasso di Copertura dei Requisiti Funzionali}

\begin{table}[h]
\centering
\begin{tabular}{ |>{\centering\arraybackslash}m{3cm}|>{\centering\arraybackslash}m{10cm}|>{\centering\arraybackslash}m{3cm}| }
\hline
Codice & Descrizione & Valutazione\\
\hline
MPD1TCRF & Misura la percentuale di requisiti funzionali implementati &
\textit{Accettato}: $\geq$ 95\%
\textit{Rifiutato}: $<$ 95\% \\
\hline
\end{tabular}
\end{table}


\subsection{Affidabilità}
\subsubsection{Descrizione}
Garantisce che il software sia stabile e affidabile durante l'uso, riducendo gli errori e migliorando la sicurezza.
\subsubsection{Metriche e valutazioni}
\paragraph{Numero di errori}

\begin{table}[h]
\centering
\begin{tabular}{ |>{\centering\arraybackslash}m{3cm}|>{\centering\arraybackslash}m{6cm}|>{\centering\arraybackslash}m{7cm}| }
\hline
Codice & Descrizione & Valutazione\\
\hline
MPD2INE & Numero di malfunzionamenti o crash in un determinato intervallo di tempo &
\textit{Accettato}:   0 incidenti/spazio temporale
\textit{Rifiutato}:  $\geq$ 1 incidente/spazio temporale\\
\hline
\end{tabular}
\end{table}

\paragraph{Robustezza agli errori}

\begin{table}[h]
\centering
\begin{tabular}{ |>{\centering\arraybackslash}m{3cm}|>{\centering\arraybackslash}m{10cm}|>{\centering\arraybackslash}m{3cm}| }
\hline
Codice & Descrizione & Valutazione\\
\hline
MPD3RE & Quale parte di tutti gli errori critici è stata messa sotto controllo &
\textit{Accettato}:   100\%
\textit{Rifiutato}:  $\leq$ 100\%\\
\hline
\end{tabular}
\end{table}

\subsection{Usabilità}
\subsubsection{Descrizione}
L'usabilità è cruciale e garantisce che l'interfaccia utente sia intuitiva, facile da usare e accessibile, riducendo il tempo di apprendimento e migliorando l'efficienza dell'utente. Per misurarla i membri del gruppo che non fanno parte del team di sviluppo del codice simuleranno il cliente finale. Le stime effettuate verranno poi gonfiate di un paio di minuti. 
\subsubsection{Metriche e valutazioni}
\paragraph{Tempo Medio di Apprendimento}
\begin{table}[h]
\centering
\begin{tabular}{ |>{\centering\arraybackslash}m{3cm}|>{\centering\arraybackslash}m{8cm}|>{\centering\arraybackslash}m{5cm}| }
\hline
Codice & Descrizione & Valutazione\\
\hline
MPD4TMA & Tempo necessario ad un nuovo utente per apprendere le funzionalitá offerte dal prodotto &
\textit{Accettato}:  $\leq$ 5 minuti
\textit{Migliorabile}: tra 5 e 15 minuti
\textit{Rifiutato}:  $>$ 15 minuti\\
\hline
\end{tabular}
\end{table}

\subsection{Efficienza di Performance}
\subsubsection{Descrizione}
L'efficienza di performance mira a ottimizzare l'uso delle risorse di sistema e a garantire che il software funzioni con elevate prestazioni, riducendo il tempo di risposta e migliorando l'esperienza dell'utente.
\subsubsection{Metriche e valutazioni}
\paragraph{Tempo Medio di Risposta}

\begin{table}[h]
\centering
\begin{tabular}{ |>{\centering\arraybackslash}m{3cm}|>{\centering\arraybackslash}m{8cm}|>{\centering\arraybackslash}m{5cm}| }
\hline
Codice & Descrizione & Valutazione\\
\hline
MPD5TMR & Indica il tempo medio di risposta di una funzione offerta dal software prodotto &
\textit{Accettato}:  $\leq$ 2 secondi
\textit{Migliorabile}: tra 2 e 3 secondi
\textit{Rifiutato}: $>$ 3 secondi\\
\hline
\end{tabular}
\end{table}

\subsection{Manutenibilità}
\subsubsection{Descrizione}
La manutenibilità si focalizza sulla facilità con cui il software può essere modificato per correggere difetti, migliorare funzionalità o adattarsi a cambiamenti ambientali.
\subsubsection{Metriche e valutazioni}
\paragraph{ Tempo Medio per la Risoluzione dei Difetti}

\begin{table}[h]
\centering
\begin{tabular}{ |>{\centering\arraybackslash}m{3cm}|>{\centering\arraybackslash}m{8cm}|>{\centering\arraybackslash}m{5cm}| }
\hline
Codice & Descrizione & Valutazione\\
\hline
MPD6TMRD & Indica il tempo medio necessario a risolvere un difetto del prodotto & \textit{Accettato}: $\leq$ 1 giorno
\textit{Migliorabile}: tra 1 e 3 giorni
\textit{Rifiutato}: $>$ 3 giorni\\
\hline
\end{tabular}
\end{table}

\newpage

\paragraph{Accoppiamento di componenti}

\begin{table}[h]
\centering
\begin{tabular}{ |>{\centering\arraybackslash}m{3cm}|>{\centering\arraybackslash}m{8cm}|>{\centering\arraybackslash}m{5cm}| }
\hline
Codice & Descrizione & Valutazione\\
\hline
MPD7AC & indica quanto strettamente sono indipendenti i componenti e quanti componenti sono esenti da impatti da cambiamenti negli altri componenti&
\textit{Accettato}:  $\leq$ 10\%
\textit{Migliorabile}: tra 10 e 20\%
\textit{Rifiutato}:  $>$ 20\% \\
\hline
\end{tabular}
\end{table}

\paragraph{Complessità ciclomatica\textsuperscript{G}}

\begin{table}[h]
\centering
\begin{tabular}{ |>{\centering\arraybackslash}m{3cm}|>{\centering\arraybackslash}m{10cm}|>{\centering\arraybackslash}m{3cm}| }
\hline
Codice & Descrizione & Valutazione\\
\hline
MPD7CC & Indica quanti moduli software hanno una complessità ciclomatica inferiore alla soglia, giudicata accettabile, di 20&
\textit{Accettato}:  $\geq$ 90\%
\textit{Rifiutato}:  $<$ 90\% \\
\hline
\end{tabular}
\end{table}

\subsection{Portabilità}
\subsubsection{Descrizione}
La portabilità assicura che il software possa essere eseguito in diversi ambienti di sistema, consentendo una facile installazione, configurazione e adattabilità.
\subsubsection{Metriche e valutazioni}
\paragraph{Compatibilità browser}

\begin{table}[h]
\centering
\begin{tabular}{ |>{\centering\arraybackslash}m{3cm}|>{\centering\arraybackslash}m{10cm}|>{\centering\arraybackslash}m{3cm}| }
\hline
Codice & Descrizione & Valutazione\\
\hline
MPD8CB & Indica il numero di browser con i quali il sistema risulta compatibile. I browser presi in considerazione sono: Chrome, Safari, Microsoft Edge, Mozilla Firefox e Opera&
\textit{Accettato}:  $\geq$ 4/5
\textit{Rifiutato}:  $<$ 4/5 \\
\hline
\end{tabular}
\end{table}

\newpage
\section{Test}
\par In questa sezione sono descritti dettagliatamente i vari tipi di test che verranno effettuati per assicurare la qualità e l'affidabilità del software.

\subsection{Test di Unità}
\par Test focalizzati sulle singole unità o componenti del software, eseguiti per verificare la loro corretta funzionalità in isolamento.

\subsection{Test di Integrazione}
\par Valutazione delle interazioni tra le diverse unità del software per identificare problemi nelle interfacce e nelle interazioni.

\subsection{Test di Sistema}
\par Verifica del comportamento complessivo del sistema, assicurando che tutti i requisiti siano soddisfatti.

\subsection{Test di Regressione}
\par Eseguiti dopo ogni cambiamento al software per garantire che le nuove modifiche non introducano errori nelle parti già testate.

\subsection{Test di Usabilità}
\par Valutazione dell'esperienza dell'utente con l'interfaccia del software, per garantire intuitività e facilità di uso.

\subsection{Test di Prestazione}
\par Controlli sulle prestazioni del software sotto vari carichi di lavoro, valutando tempo di risposta e scalabilità.


\subsection{Test di Accettazione}
\par Test condotti dal cliente o dall'utente finale per verificare se il software soddisfa i requisiti e se è pronto per il rilascio.

\newpage
\section{Resoconto}

   
