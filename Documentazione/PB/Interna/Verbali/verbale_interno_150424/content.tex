\section{Argomenti}
\subsection{Retrospettiva}
Nella prima parte dell'incontro il gruppo ha discusso ciò che è stato fatto nella settimana precedente, valutando lo stato di avanzamento delle issues: si è visto che non tutti gli obiettivi previsti per la milestone precedente sono stati conclusi, in particolare:
\begin{itemize}
    \item Bisogna scrivere i diagrammi UML delle componenti della view;
    \item La documentazione, seppur modificata, non è stata verificata.
\end{itemize}
Si è poi passati alla discussione dei suggerimenti emersi dall'incontro con il professor Cardin sulla progettazione e sugli UML.
In particolare, il gruppo ha stabilito che, durante il prossimo sprint, uno degli obiettivi sarà sistemare i diagrammi secondo le direttive del professore, rimuovendo i pattern previsti per risolvere problematiche ``future" e utilizzando quelli consigliati dalle tecnologie adottate. Inoltre, il gruppo ha stabilito che, come da suggerimento del professore, la struttura dei diagrammi UML dovrà essere adattata alle tecnologie e ai paradigmi di programmazione utilizzati, specificando nella documentazione le convenzioni adottate.

\subsection{Pianificazione}
Nella seconda parte dell'incontro, sono stati definiti gli obiettivi per la prossima iterazione. Inoltre, si è svolta una rotazione dei ruoli.
\bigskip

\noindent Dunque, per il prossimo sprint (terminante il 21/04/24), si è stabilito che:
\begin{itemize}
    \item I progettisti si occuperanno di adeguare gli UML finora prodotti ai suggerimenti del professor Cardin e completeranno la progettazione, aggiungendo i diagrammi UML per i componenti della view;
    \item Il verificatore si occuperà di proseguire lo studio delle tecnologie, continuare a determinare gli unit test da effettuare e verificare i documenti in sospeso;
    \item L'amministratore dovrà occuparsi di ultimare le correzioni alla documentazione in base al feedback della RTB.
\end{itemize}

\noindent Per non perdere tempo, qualora la progettazione venisse ultimata prima del termine dello sprint, i progettisti assumeranno il ruolo di programmatori per inizare l'attività di codifica. I ruoli sono stati stabiliti come segue:
\begin{itemize}
    \item \textbf{Responsabile:} Luca Securo;
    \item \textbf{Amministratore:} Zaccaria Marangon;
    \item \textbf{Verificatore:} Andrea Mangolini;
    \item \textbf{Progettisti:} Giulio Biscontin, Jessica Carretta, Lorenzo Pasqualotto.
\end{itemize}