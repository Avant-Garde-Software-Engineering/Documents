\section{Argomenti}
\subsection{Retrospettiva}
Nella prima parte dell'incontro il gruppo ha discusso ciò che è stato fatto nella settimana precedente, valutando lo stato di avanzamento delle issues: si è visto che non tutti gli obiettivi previsti per la milestone precedente sono stati terminati essendo stati creati con l'avanzamento del progetto, in particolare:
\begin{itemize}
    \item Bisogna ultimare lavori di codifica nella parte di model, setup e eseguire il testing di essi;
    \item La documentazione, pur avendo progredito richiede comunque modifiche per alcuni documenti strettamente collegati al prodotto.
\end{itemize}
Si è poi passati alla discussione riguardo ai comportamenti da adottare per l'inserimento di file SVG e la successiva creazione del magazzino.
In particolare, il gruppo ha stabilito che, durante il prossimo sprint, implementerà un sistema di caricamento di file SVG per la creazione di un magazzino però con delle limitazioni, oltre alla creazione dei necessari test per la suddetta funzione.

\subsection{Pianificazione}
Nella seconda parte dell'incontro, sono stati definiti gli obiettivi per la prossima iterazione. Inoltre, si è deciso di non routare i ruoli rispetto alla settimana scorsa.
\bigskip

\noindent Dunque, per il prossimo sprint (terminante il 28/04/24), si è stabilito che:
\begin{itemize}
    \item Il Responsabile e l'Amministratore dovranno occuparsi di avanzare con la documentazione, almeno le parti in cui non richiedono l'aver completato la programmazione;
    \item Il verificatore si occuperà di proseguire lo studio delle tecnologie, continuare a determinare gli unit test da effettuare e verificare i documenti in sospeso;
    \item I programmatori dovranno continuare la codifica del prodotto con le funzioni necessarie da implementare per l'MVP e completare le issue che sono state assegnate a loro relativa al codice.
\end{itemize}

\noindent Per non perdere tempo, tutti i ruoli rimangono tali per evitare perdite di tempo dovute al cambiamento dei ruoli, soprattutto per il lavoro dei programmatori che è quello che attualmente richiede più tempo per adeguarsi una volta effettuato il cambio. I ruoli sono stati stabiliti come segue:
\begin{itemize}
    \item \textbf{Responsabile:} Luca Securo;
    \item \textbf{Amministratore:} Zaccaria Marangon;
    \item \textbf{Verificatore:} Andrea Mangolini;
    \item \textbf{Programmatori:} Giulio Biscontin, Jessica Carretta, Lorenzo Pasqualotto.
\end{itemize}