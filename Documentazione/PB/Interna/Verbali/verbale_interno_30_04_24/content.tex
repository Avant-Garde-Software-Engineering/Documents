\section{Argomenti}
\subsection{Retrospettiva}
Nella prima parte dell'incontro il gruppo ha analizzato, in retrospettiva, il lavoro effettuato nella settimana precedente.
Gli obbiettivi stabiliti nella riunione del 22/04/24 sono stati realizzati, in particolare:
\begin{itemize}
    \item Il gruppo dedicato alla codifica è riuscito ad arrivare ad un risultato notevole, vicino allo stato completo del prodotto;
    \item Il gruppo di verifica è riuscito a completare la revisione di buona parte del codice realizzato;
    \item  É stata elaborata una stesura iniziale dei documenti \textit{Manuale dell'utente} e \textit{Specifica Tecnica}.
\end{itemize}



\subsection{Pianificazione}
\noindent Nella seconda parte dell'incontro, il gruppo ha discusso e pianificato il lavoro futuro. In particolare si è rivelato necessario un aumento del numero di partecipanti al gruppo di verifica. I programmatori, inoltre, in questa fase si occuperanno della stesura dei documenti mancanti.
\bigskip

\noindent Dunque, per il prossimo sprint (terminante il 5/05/24), si è che:
\begin{itemize}
    \item Il responsabile contatterà il proponente in caso di necessità di un colloquio;
    \item L'amministratore deve aggiornare il  \textit{Piano di Qualifica};
    \item I progettisti procederanno con la stesura della \textit{Specifica Tecnica};
    \item I programmatori si occuperanno del \textit{Manuale dell'utente}, insieme all' aggiornamento del \textit{Piano di Progetto};
    \item I verificatori devono continuare la verifica del codice e della documentazione;
\end{itemize}
\bigskip

Di seguito si riportano comunque i ruoli correnti:
\begin{itemize}
    \item \textbf{Responsabile:} Lorenzo Pasqualotto;
    \item \textbf{Amministratore:} Lorenzo Pasqualotto;
    \item \textbf{Programmatore:} Luca Securo;
    \item \textbf{Progettista:} Jessica Carretta;
    \item \textbf{Verificatori:} Giulio Biscontin, Andrea Mangolini, Zaccaria Marangon.
\end{itemize}