\section{Argomenti}
\subsection{Retrospettiva}
Nella prima parte dell'incontro il gruppo ha analizzato, in retrospettiva, il lavoro effettuato nella settimana precedente.
In particolare, sono stati completati tutti i documenti per la seconda parte della revisione RTB e sono stati inviati all'azienda per l'approvazione i documenti esterni 
(\textit{Analisi dei Requisiti v4.0.0, Piano di Progetto v1.0.0, Piano di Qualifica v2.0.0}). I tempi sono stati rispettati, tuttavia si è verificato un 
leggero ritardo nella risposta da parte dell'azienda proponente. Pertanto, la richiesta di colloquio con il Professor Vardanega per la revisione è stata 
riprogrammata al momento in cui si avrà l'approvazione dell'azienda. 
Si prevede dunque di sollecitare il proponente dopo le vacanze pasquali per procedere con l'RTB il prima possibile.

\subsection{Pianificazione}
\noindent Nella seconda parte dell'incontro, il gruppo ha discusso e pianificato il lavoro futuro. Innanzitutto, si è deciso di modificare la durata degli sprint da due a una settimana. 
L'obiettivo è quello di aumentare il ritmo per poter concludere il progetto nei tempi previsti. Si è inoltre deciso di modificare l'issue tracking system, passando da Jira alle GitHub 
Issues. Ciò è stato stabilito in quanto Jira non si è rivelato uno strumento efficace per la pianificazione degli sprint del gruppo. È stata infatti pian piano abbandonata dal gruppo 
per la poca praticità e immediatezza. Le GitHub Issues garantirebbero una maggiore comodità, essendo disponibili direttamente dal GitHub del gruppo e consentendo comunque di monitorare 
e pianificare tutto il necessario. 
\bigskip

\noindent Inoltre, per il prossimo sprint (terminante dunque il 07/04/24), si è deciso nello specifico che:
\begin{itemize}
    \item L'amministratore deve modificare \textit{Piano di Progetto} e \textit{Norme di Progetto} con quanto appena descritto;
    \item Il responsabile deve richiedere il colloquio con il Professor Vardanega per la seconda parte dell'RTB (post approvazione azienda proponente);
    \item I progettisti, per evitare rallentamenti dovuti alla conclusione dell'RTB, possono iniziare un'attività di ricerca ed addestramento riguardo le possibili scelte 
    da implementare riguardo la progettazione del prodotto finito.
\end{itemize}
Con questi obiettivi in mente, sono stati cambiati i ruoli come segue:
\begin{itemize}
    \item \textbf{Responsabile:} Lorenzo Pasqualotto;
    \item \textbf{Amministratore:} Jessica Carretta;
    \item \textbf{Verificatore:} Luca Securo;
    \item \textbf{Progettisti:} Giulio Biscontin, Andrea Mangolini, Zaccaria Marangon.
\end{itemize}