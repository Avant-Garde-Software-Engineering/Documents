\section{Argomenti}
\subsection{Retrospettiva}
Nella prima parte dell'incontro il gruppo ha analizzato, in retrospettiva, il lavoro effettuato nella settimana precedente.
In particolare, gli obiettivi previsti nel verbale interno del 29-03-24 sono stati tutti raggiunti:
\begin{itemize}
    \item Sono stati modificati i documenti \textit{Norme di Progetto} e \textit{Piano di Progetto} secondo quanto previsto;
    \item La progettazione di dettaglio del prodotto è iniziata e sono stati creati dei diagrammi UML per quanto riguarda la parte di modello e logica di controllo del prodotto finale;
    \item  I documenti inviati all'azienda, dopo sollecito, sono stati approvati ed è stato possibile richiedere la seconda parte della revisione RTB.
\end{itemize}
Inoltre, grazie al rispetto delle scadenze e ad una migliore organizzazione data dalle nuove norme, il gruppo è riuscito anche ad effettuare la revisione RTB con il professor Vardanega in data 05-04-24. 


\subsection{Pianificazione}
\noindent Nella seconda parte dell'incontro, il gruppo ha discusso e pianificato il lavoro futuro. In particolare, si è discusso dei risultati e commenti ricevuti dal professor Vardanega a proposito dei documenti e di come poter procedere per poter rimediare ai difetti riscontrati.
\bigskip

\noindent Dunque, per il prossimo sprint (terminante il 14/04/24), si è deciso nello specifico che:
\begin{itemize}
    \item L'amministratore deve modificare \textit{Piano di Progetto},  \textit{Norme di Progetto} e \textit{Piano di Qualifica} seguendo le indicazioni del professor Vardanega riportate nel documento di valutazione RTB;
    \item Il responsabile deve:
        \begin{itemize}
            \item Richiedere un incontro con il professor Cardin per poter discutere della progettazione fatta finora, in modo da avere conferma di essere sulla giusta strada,
            \item Richiedere dei chiarimenti al professor Vardanega per quanto riguarda il versionamento e il tipo di modifiche da apportare alle \textit{Norme di Progetto}, in quanto questi punti non risultano chiari per il gruppo;
        \end{itemize}
    \item I progettisti devono continuare la progettazione, in particolare devono iniziare la progettazione delle componenti dell'interfaccia utente;
    \item Il verificatore deve iniziare a definire gli unit test per le componenti;
\end{itemize}
\bigskip

\noindent Essendo le attività non completamente concluse (in particolar modo per quanto riguarda la progettazione), il gruppo ha deciso di non ruotare i ruoli. Lo scopo è quello di garantire maggiore continuità e prevenire rallentamenti causati dall'adattamento a nuove responsabilità.
Di seguito si riportano comunque i ruoli correnti per maggiore chiarezza:
\begin{itemize}
    \item \textbf{Responsabile:} Lorenzo Pasqualotto;
    \item \textbf{Amministratore:} Jessica Carretta;
    \item \textbf{Verificatore:} Luca Securo;
    \item \textbf{Progettisti:} Giulio Biscontin, Andrea Mangolini, Zaccaria Marangon.
\end{itemize}