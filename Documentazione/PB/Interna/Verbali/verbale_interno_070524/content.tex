\section{Argomenti}
\subsection{Retrospettiva}
Nella prima parte dell'incontro, il gruppo ha condotto un'analisi retrospettiva del lavoro svolto nella settimana precedente. In particolare, si è completata con successo la codifica e la verifica del prodotto. 

Inoltre, in data 07-05-24, l'azienda ha validato il prodotto, confermando l'implementazione di tutti i requisiti minimi richiesti e il superamento dei test di accettazione. Il prodotto è dunque confermato essere MVP.

I tempi sono dunque stati rispettati, essendo la data di completamento prevista per il 10-05-24. Tuttavia, si è comunque verificato un ritardo nella risposta dell'azienda proponente. È stato infatti necessario sollecitare il proponente per ottenere suddetto colloquio e per ottenere l'approvazione del verbale esterno del 16-04-24, il quale è stato rinviato e attende ancora la firma.

\subsection{Pianificazione}
\noindent Nella seconda parte dell'incontro, il gruppo ha discusso e pianificato il lavoro futuro. L'obiettivo è di completare entrambe le due parti della revisione PB durante il prossimo sprint (terminante il 17-05-24, data prevista di fine progetto).

A tal fine, le issue più urgenti riguardano il completamento dei documenti e l'ultimazione dei test effettuati.

\subsubsection{Ultimazione dei test}
Per quanto riguarda i test, i verificatori si occuperanno di apportare le ultime modifiche di ritocco necessarie a perfezionarli e migliorarne l'organizzazione.

\subsubsection{Completamento dei documenti}
Per quanto riguarda i documenti, questi saranno completati in due diverse ``ondate'' a seconda della priorità.

\bigskip
\noindent La prima ``ondata'' comprende tutti i documenti che necessitano di approvazione da parte del proponente e che sono necessari alla prima parte della revisione PB con il Professor Cardin. Si prevede di completare tali documenti entro l'08-05-24 in modo da poterli inviare all'azienda per l'approvazione il prima possibile. I documenti richiesti in questa ``ondata'' sono:
\begin{itemize}
    \item La \textit{Specifica tecnica} alla v1.0.0, attualmente in fase di verifica;
    \item Il \textit{Manuale utente} alla v1.0.0, attualmente in fase di redazione;
    \item L'\textit{Analisi dei Requisiti} alla v5.0.0, da approvare;
    \item Il \textit{Verbale esterno 07-05-24} alla v1.0.0, ancora da redigere.
\end{itemize}

\bigskip
\noindent La seconda ``ondata'' riguarda i rimanenti documenti. Questi includono, innanzitutto, quelli che non richiedono l'approvazione del proponente e possono quindi essere completati dal gruppo con più tranquillità, quali:
\begin{itemize}
    \item La \textit{Lettera di Presentazione} per la PB, ancora da redigere;
    \item Le \textit{Norme di Progetto} alla v4.0.0, da verificare ed approvare.
\end{itemize}
Gli altri documenti da completare sono quelli che contengono i resoconti del progetto, necessari esclusivamente per la seconda parte della revisione PB con il Professor Vardanega, ovvero:
\begin{itemize}
    \item Il \textit{Piano di Progetto} alla v2.0.0, ancora da modificare;
    \item Il \textit{Piano di Qualifica} alla v3.0.0, ancora da modificare.
\end{itemize}
Per quest'ultimi si prevede di completarli entro il 12-05-24, in modo da poterli inviare al proponente per l'approvazione il prima possibile.

\subsubsection{Altre issue}
Per il prossimo sprint, si è deciso inoltre che:
\begin{itemize}
    \item Il responsabile deve richiedere (ed eventualmente sollecitare) l'approvazione dei documenti all'azienda proponente;
    \item Il responsabile deve richiedere il colloquio con il professor Cardin e, successivamente, con il Professor Vardanega per la PB (post approvazione dei documenti da parte dell'azienda proponente).
\end{itemize}

\subsubsection{Ruoli}
Con questi obiettivi in mente, i ruoli previsti per questo sprint sono i seguenti:
\begin{itemize}
    \item \textbf{Responsabile:} Lorenzo Pasqualotto;
    \item \textbf{Amministratore:} Luca Securo;
    \item \textbf{Verificatori:} Giulio Biscontin, Andrea Mangolini, Zaccaria Marangon;
    \item \textbf{Programmatore:} Jessica Carretta.
\end{itemize}
Si noti che in questa fase il ruolo di ``Programmatore'' viene inteso come ``Redattore'' in quanto la fase di codifica è stata completata. Inoltre, si fa presente che i nuovi ruoli saranno posti in essere soltanto dopo il completamento delle issue attualmente in corso. Infatti, si vuole concludere eventuali modifiche in sospeso per garantire la continuità operativa.

\bigskip
\noindent Si precisa, inoltre, che in questa fase, data l'urgenza e le esigenze del progetto, i ruoli potrebbero subire delle variazioni per adeguarsi alle necessità del progetto stesso. Tali variazioni verranno concordate tra i membri del gruppo attraverso il gruppo Telegram. Al momento, infatti, l'importante è completare le issue rispettando i tempi stabiliti. Pertanto, potrebbero verificarsi delle modifiche per adeguarsi anche alle disponibilità dei membri, pur comunque mantenendo le regole previste nelle \textit{Norme di Progetto.}