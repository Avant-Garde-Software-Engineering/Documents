\section{Scopo dell'incontro} \label{sec:scopo}
La riunione si è svolta con l'obiettivo di presentare l'MVP al proponente e discutere del soddisfacimento dei requisiti obbligatori.

\section{Argomenti} \label{sec:argomenti}
\subsection{Revisione dell'applicativo prodotto}
L'incontro è iniziato con la dimostrazione, mediante condivisione schermo, del superamento dei test di accettazione presenti nel \textit{Piano di qualifica}.
È seguita un'analisi più dettagliata del soddisfacimento dei requisiti obbligatori espliciti, presenti nel \textit{Capitolato} e di ogni requisito obbligatorio presente nell' \textit{Analisi dei requisiti}. I documenti sono disponibili nella sezione \hyperref[sec:riferimenti_esterni]{Riferimenti Esterni}.
\subsection{Riscontro}
Il riscontro è stato \textcolor{green}{positivo} e lo stato di MVP è stato confermato, pertanto si proseguirà verso l'incontro con il professore Cardin per la prima revisione PB.

\newpage
\section{Riferimenti esterni} \label{sec:riferimenti_esterni}
Per ulteriori chiarimenti sugli argomenti discussi nel documento, si possono consultare i seguenti link esterni:
\begin{itemize}
    \item \textbf{Analisi dei requisiti v5.0.0}:\\
    \url{https://github.com/Avant-Garde-Software-Engineering/WMS3D/blob/main/Documentazione/PB/Esterna/analisi_dei_requisiti_v5.0.0.pdf}
    \item \textbf{Piano di qualifica v3.0.0}:\\
    \url{https://github.com/Avant-Garde-Software-Engineering/WMS3D/blob/main/Documentazione/PB/Esterna/piano_di_qualifica_v3.0.0.pdf}
    \item Capitolato \textbf{Warehouse Management 3D}:\\
    \url{https://www.math.unipd.it/~tullio/IS-1/2023/Progetto/C5.pdf}
\end{itemize}