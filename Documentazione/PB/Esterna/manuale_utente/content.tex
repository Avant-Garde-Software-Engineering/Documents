\begin{beginningnote}
    Si tenga presente che alcuni termini utilizzati nel documento riportano la lettera \textbf{G} in apice, allo scopo di evidenziare le parole che assumono uno specifico 
    significato nell'ambito del progetto. Per comprenderle in maniera corretta, si rimanda il lettore al documento ``Glossario", che contiene un elenco completo di tutte le 
    terminologie utilizzate con relative definizioni, allo scopo di costruire un linguaggio uniforme che possa migliorare la comunicazione tra i componenti interni al gruppo 
    e gli stakeholder\textsuperscript{G} esterni.
\end{beginningnote}

%%%%%%%%%%%%%%%%%%%%%%%%%%%%%%%%%%%
% SCOPO DEL DOCUMENTO
%%%%%%%%%%%%%%%%%%%%%%%%%%%%%%%%%%%
\section{Scopo del documento}\label{sec:scopo_del_documento}
    Lo scopo del seguente documento è quello di illustrare le funzionalità fornite dall'applicazione e fornire agli utenti le istruzioni necessarie per il corretto utilizzo 
    della stessa. Si intende quindi informare ogni utilizzatore sui requisiti minimi necessari per la corretta esecuzione dell'applicativo al fine di fornire un'esperienza 
    utente chiara ed esaustiva.    

%%%%%%%%%%%%%%%%%%%%%%%%%%%%%%%%%%%
% SCOPO DEL PROGETTO
%%%%%%%%%%%%%%%%%%%%%%%%%%%%%%%%%%%
\section{Scopo del progetto}\label{sec:scopo_del_progetto}

    Il progetto nasce nell'ambito dei \textbf{sistemi gestionali di magazzino}, meglio noti con il termine inglese di \textit{Warehouse Management Systems} (WMS), con 
    l'obiettivo di risolvere una serie di problematiche derivanti dalle soluzioni tradizionali tuttora presenti sul mercato.\\
    Il focus principale sarà migliorare la user experience, tramite la realizzazione di un applicativo che proponga all'utente un'interazione con il magazzino in un 
    ambiente di lavoro 3D. \\
    Tale soluzione, rispetto ai tradizionali sistemi 2D, garantirebbe una maggiore comprensione degli spazi, proponendo una visualizzazione più intuitiva e completa 
    degli spazi di magazzino. Permetterebbe quindi all'utente di prendere decisioni in modo più efficace ed efficiente, permettendo così di ottimizzare i processi di logistica.

    Per raggiungere questo obiettivo, l'ambiente di lavoro non può essere una semplice visualizzazione del magazzino. L'utente dovrà infatti poter:
    \begin{itemize}
        \item Spostarsi all'interno dell'ambiente 3D;
        \item Progettare le scaffalature che sono presenti nel magazzino e modificarle nel tempo;
        \item Simulare i flussi di movimento di prodotti.
    \end{itemize}

    Il progetto deve concretizzarsi nella realizzazione di una web app fruibile agli impiegati d'ufficio ed incentrata sulla visualizzazione 3D del magazzino.
    Per visionare il capitolato\textsuperscript{G} completo e la documentazione del gruppo, si veda la sezione \hyperref[sec:riferimenti_esterni]{Riferimenti Esterni} 
    del documento.

\newpage


%%%%%%%%%%%%%%%%%%%%%%%%%%%%%%%%%%%
% REQUISITI E COMPATIBILITA'
%%%%%%%%%%%%%%%%%%%%%%%%%%%%%%%%%%%
\section{Requisiti e compatibilità}\label{sec:requisiti_e_compatibilità}



%%%%%%%%%%%%%%%%%%%%%%%%%%%%%%%%%%%
% INSTALLAZIONE 
%%%%%%%%%%%%%%%%%%%%%%%%%%%%%%%%%%%
\section{Installazione}\label{sec:installazione}


%%%%%%%%%%%%%%%%%%%%%%%%%%%%%%%%%%%
% ISTRUZIONI D'USO 
%%%%%%%%%%%%%%%%%%%%%%%%%%%%%%%%%%%
\section{Istruzioni d'uso}\label{sec:Istruzioni_uso}

\newpage
%%%%%%%%%%%%%%%%%%%%%%%%%%%%%%%%%%%
% RIFERIMENTI ESTERNI
%%%%%%%%%%%%%%%%%%%%%%%%%%%%%%%%%%%
\newpage
\section{Riferimenti esterni}\label{sec:riferimenti_esterni}
Per ulteriori chiarimenti sugli argomenti discussi nel documento, si possono consultare i seguenti link esterni:
\begin{itemize}
    \item Capitolato \textbf{Warehouse Management 3D}:\\
    \url{https://www.math.unipd.it/~tullio/IS-1/2023/Progetto/C5.pdf} 
    \item Link alla \textbf{documentazione del gruppo}:\\
    \url{https://avant-garde-software-engineering.github.io/documentazione.html} \textcolor{gray}{\textit{(ultimo accesso 25-04-24)}}
\end{itemize}