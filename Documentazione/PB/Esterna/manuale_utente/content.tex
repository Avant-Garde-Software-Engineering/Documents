\begin{beginningnote}
    Si tenga presente che alcuni termini utilizzati nel documento riportano la lettera \textbf{G} in apice, allo scopo di evidenziare le parole che assumono uno specifico 
    significato nell'ambito del progetto. Per comprenderle in maniera corretta, si rimanda il lettore al documento ``Glossario", che contiene un elenco completo di tutte le 
    terminologie utilizzate con relative definizioni, allo scopo di costruire un linguaggio uniforme che possa migliorare la comunicazione tra i componenti interni al gruppo 
    e gli stakeholder\textsuperscript{G} esterni.
\end{beginningnote}

%%%%%%%%%%%%%%%%%%%%%%%%%%%%%%%%%%%
% SCOPO DEL DOCUMENTO
%%%%%%%%%%%%%%%%%%%%%%%%%%%%%%%%%%%
\section{Scopo del documento}\label{sec:scopo_del_documento}
    Lo scopo del seguente documento è quello di illustrare le funzionalità fornite dall'applicazione e fornire agli utenti le istruzioni necessarie per il corretto utilizzo 
    della stessa. Si intende quindi informare ogni utilizzatore sui requisiti minimi necessari per la corretta esecuzione dell'applicativo al fine di fornire un'esperienza 
    utente chiara ed esaustiva.    

%%%%%%%%%%%%%%%%%%%%%%%%%%%%%%%%%%%
% SCOPO DEL PROGETTO
%%%%%%%%%%%%%%%%%%%%%%%%%%%%%%%%%%%
\section{Scopo del progetto}\label{sec:scopo_del_progetto}

    Il progetto nasce nell'ambito dei \textbf{sistemi gestionali di magazzino}, meglio noti con il termine inglese di \textit{Warehouse Management Systems} (WMS), con 
    l'obiettivo di risolvere una serie di problematiche derivanti dalle soluzioni tradizionali tuttora presenti sul mercato.\\
    Il focus principale sarà migliorare la user experience, tramite la realizzazione di un applicativo che proponga all'utente un'interazione con il magazzino in un 
    ambiente di lavoro 3D. \\
    Tale soluzione, rispetto ai tradizionali sistemi 2D, garantirebbe una maggiore comprensione degli spazi, proponendo una visualizzazione più intuitiva e completa 
    degli spazi di magazzino. Permetterebbe quindi all'utente di prendere decisioni in modo più efficace ed efficiente, permettendo così di ottimizzare i processi di logistica.

    Per raggiungere questo obiettivo, l'ambiente di lavoro non può essere una semplice visualizzazione del magazzino. L'utente dovrà infatti poter:
    \begin{itemize}
        \item Spostarsi all'interno dell'ambiente 3D;
        \item Progettare le scaffalature che sono presenti nel magazzino e modificarle nel tempo;
        \item Simulare i flussi di movimento di prodotti.
    \end{itemize}

    Il progetto deve concretizzarsi nella realizzazione di una web app fruibile agli impiegati d'ufficio ed incentrata sulla visualizzazione 3D del magazzino.
    Per visionare il capitolato\textsuperscript{G} completo e la documentazione del gruppo, si veda la sezione \hyperref[sec:riferimenti_esterni]{Riferimenti Esterni} 
    del documento.

\newpage


%%%%%%%%%%%%%%%%%%%%%%%%%%%%%%%%%%%
% REQUISITI E COMPATIBILITA'
%%%%%%%%%%%%%%%%%%%%%%%%%%%%%%%%%%%
\section{Requisiti e compatibilità}\label{sec:requisiti_e_compatibilità}
    In questa sezione sono illustrati i requisiti minimi neccessari per una corretta esecuzione dell'applicativo realizzato. Saranno quindi evidenziate le caratteristiche 
    che ogni terminale deve soddisfare per configurare correttamente l'ambiente di esecuzione.

    \subsection{Requisiti software}\label{sec:requisiti_e_compatibilità:software}
    L'applicativo sarà reso disponibile all'utente attraverso due modalità: la \textit{modalità utente}\textsuperscript{G} e la \textit{modalità programmatore}\textsuperscript{G}.\\
    La \textit{modalità programmatore}\textsuperscript{G} è consigliata solamente ad utilizzatori esperti. \\
    Per questa esecuzione saranno necessari: l'installazione del software Node.js, alla versione 20.0 o superiore, e un browser stabile. \\
    Per una corretta installazione del software si rimanda alla pagina dedicata, presente nella sezione \hyperref[sec:riferimenti_esterni]{Riferimenti Esterni} del documento. 
    L'applicativo è stato testato per il corretto funzionamento con i principali motori di ricerca che si riportano qui sotto:
    \begin{xltabular}{\textwidth}{ X | X}

        \rowcolor{black}
        \textbf{\color{white} Motore di ricerca} & \textbf{\color{white} Verisone}\\ 
        \hline
        \endhead

        Google Chrome & 124.0 \\
        \hline

        Micorsoft Edge & 124.0 \\
        \hline
        
        Safari & 17.0 \\
        \hline

        Mozilla Firefox & 115.0 \\
        \hline
        
        \caption{Tabella dei requisiti software}
        \label{tab:requisiti:soft}
    \end{xltabular}
    La \textit{modalità programmatore}\textsuperscript{G} è invece accessibile anche da utenti meno esperti. Non necessita di particolari configurazioni ma si raccomanda 
    comunque di utilizzare uno dei browser sopra citati. \\



    \subsection{Requisiti hardware}\label{sec:requisiti_e_compatibilità:hardware}
    L'applicativo realizzato appartiene alla categoria delle web-app. L'esecuzione non avviene quindi in locale sul terminale utilizzato ma viene effettuata direttamente dal 
    browser scelto dall'utente. Non sono quindi richieste dei particolari requisiti hardware minimi per l'esecuzione.\\
    Viene comunque consigliato l'utilizzo di un terminale aggiornato che, a titolo di riferimento, può essere identificato in:
    \begin{xltabular}{\textwidth}{ X | X}

        \rowcolor{black}
        \textbf{\color{white} Componente} & \textbf{\color{white} Requisito minimo consigliato}\\ 
        \hline
        \endhead
        
        Connessione Internet & Connessione Internet stabile e veloce \\
        \hline

        Processore & Quad-Core 1,80 GHz \\
        \hline
        
        Memoria RAM & 8GB DDR3 \\
        \hline

        \caption{Tabella dei requisiti hardware}
        \label{tab:requisiti:hard}
    \end{xltabular}

    \newpage

    

%%%%%%%%%%%%%%%%%%%%%%%%%%%%%%%%%%%
% INSTALLAZIONE ED ESECUZIONE
%%%%%%%%%%%%%%%%%%%%%%%%%%%%%%%%%%%
\section{Installazione ed esecuzione}\label{sec:install_run}

    L'esecuzione dell'applicazione si differisce in base alla modalità di esecuzione scelte. Pertanto si riportano separatamente le due modalità disponibili. \\
    \textbf{In caso si scelga di eseguire il programma in \textit{modalità utente}\textsuperscript{G} è possibile trascurare la sezione installazione. Si rimanda quindi alla 
    sezione \hyperref[sec:install_run:user]{esecuzione della modalità utente}.} \\ 

    \subsection{Modalità programmatore}\label{sec:install_run:esperto}
    Questa sezione intende definire le operazioni di installazione preliminari necessarie per l'esecuzione dell'applicativo in \textit{modalità programmatore}\textsuperscript{G}.\\
    Di seguito saranno quindi elencati i passaggi necessari per la clonazione della repository e l'avvio dell'applicazione. \\
    
    \subsubsection{Clonazione della repository}\label{sec:install_run:esperto:clone}
    Per scaricare tutti i file neccessari all'esecuzione è possibile: 
    \begin{enumerate}
        \item Scaricare il codice (archivio in formato \textit{.zip}) direttamente dal seguente link: \\  
        \url{INSERIRE LINK CORRETTO} \textcolor{gray}{\textit{(ultimo accesso 04-05-24)}}
        \item Utilizzare i servizi messi a disposizione dal software Git, che deve essere installato sulla macchina di esecuzione
        \begin{itemize}
            \item Posizionarsi sulla repository di interesse.
            \item Utilizzare il comando:\\
            \textbf{git clone \url{INSERIRE LINK CORRETTO} }
        \end{itemize}
    \end{enumerate}

    \subsubsection{Avvio dell'applicativo}\label{sec:install_run:esperto:avvio}
    Una volta creata una copia del codice in una repository locale del terminale posizionarsi all'interno di essa in corrispondenza della repository \textit{Diamociunnome}.\\
    Sarà quindi neccessaio aprire un terminale in corrispondenza di tale repository ed eseguire i seguenti passaggi: 
    \begin{enumerate}
        \item Installare le dipendenze (necessario solamente al primo avvio): \\  
        \textbf{npm install}
        \item Buildare l'applicazione con il comando: \\  
        \textbf{npm run dev}
        \item Una volta terminata la fase di built dell'applicativo, aprire un browser e ricercare la seguente pagina web: \\  
        \url{http://localhost:3000}
    \end{enumerate}

    \subsection{Modalità utente}\label{sec:install_run:user}
    Come già spiegato in \textit{modalità utente}\textsuperscript{G} non sarà necessaria alcuna procedura preliminare. L'utente dovrà solamente collegarsi al seguente link per poter eseguire 
    direttamente la web-app dal proprio browser predefinito: 
    \begin{itemize}
        \item\url{inserire il link corretto}
    \end{itemize}

    \newpage



    
%%%%%%%%%%%%%%%%%%%%%%%%%%%%%%%%%%%
% ISTRUZIONI D'USO 
%%%%%%%%%%%%%%%%%%%%%%%%%%%%%%%%%%%
\section{Istruzioni d'uso}\label{sec:Istruzioni_uso}

    \subsection{Schermata iniziale utente}\label{sec:inizio}






\newpage
%%%%%%%%%%%%%%%%%%%%%%%%%%%%%%%%%%%
% RIFERIMENTI ESTERNI
%%%%%%%%%%%%%%%%%%%%%%%%%%%%%%%%%%%

\section{Riferimenti esterni}\label{sec:riferimenti_esterni}
Per ulteriori chiarimenti sugli argomenti discussi nel documento, si possono consultare i seguenti link esterni:
\begin{itemize}
    \item Capitolato \textbf{Warehouse Management 3D}:\\
    \url{https://www.math.unipd.it/~tullio/IS-1/2023/Progetto/C5.pdf} \textcolor{gray}{\textit{(ultimo accesso 04-05-24)}}
    \item Link alla \textbf{documentazione del gruppo}:\\
    \url{https://avant-garde-software-engineering.github.io/documentazione.html} \textcolor{gray}{\textit{(ultimo accesso 25-04-24)}}
    \item Pagina di installazione al software \textbf{Node.js}:\\
    \url{https://nodejs.org/en/} \textcolor{gray}{\textit{(ultimo accesso 04-05-24)}}
    \item \textbf{Glossario} di progetto: \\
    \url{https://github.com/Avant-Garde-Software-Engineering/WMS3D/blob/main/Documentazione/PB/Esterna/glossario.pdf} \textcolor{gray}{\textit{(ultimo accesso 04-05-24)}}
\end{itemize}