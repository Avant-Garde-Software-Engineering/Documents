\section*{\Large Registro delle Modifiche}
    \begin{table}[h]
        \centering
        \rowcolors{2}{gray!30}{white}
        \renewcommand\tabularxcolumn[1]{m{#1}} %per centrare contenuto nelle tabelle
        \renewcommand{\arraystretch}{1.5}
        \begin{tabularx}{0.98\textwidth}
            {c|c|c|c|>{\centering\arraybackslash}X}
            \rowcolor{black}
            \textbf{\color{white} Versione} & \textbf{\color{white} Data} & \textbf{\color{white} Nominativo} & \textbf{\color{white} Ruolo} & \textbf{\color{white} Descrizione} \\ 
            \hline

            %inserire versione, data, nome, ruole e cosa è stato modificato
            %più vecchio
            
            
            v0.0.1 & 25-04-24 & Luca Securo & Responsabile & Creazione della struttura del documento, scrittura delle sezioni \hyperref[sec:scopo_del_documento]{Scopo del documento}, 
                                                             \hyperref[sec:scopo_del_progetto]{Scopo del progetto}\\
            v0.1.0 & 29-04-24 & Zaccaria Marangon & Amministratore & Revisione delle sezioni \hyperref[sec:scopo_del_documento]{Scopo del documento}, 
                                                                     \hyperref[sec:scopo_del_progetto]{Scopo del progetto} e controllo generale del documento.\\
            v0.1.1 & 04-05-24 & Luca Securo & Programmatore & Scrittura dei contenuti delle sezioni \hyperref[sec:requisiti_e_compatibilità]{Requisiti
                                                            e compatibilità}, \hyperref[sec:install_run]{Installazione ed esecuzione}\\
            
            
            %  \ref{sec:analisi_rischi}, \ref{sec:riferimenti_esterni} e della sotto-sezione \ref{sec:preventivo:totale}. \\
               
            %più recente
            
            \hline
        \end{tabularx}
    \end{table}