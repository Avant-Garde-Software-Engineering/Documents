\begin{beginningnote}
    Si tenga presente che alcuni termini utilizzati nel documento riportano la lettera \textbf{G} in apice, allo scopo di evidenziare le parole che assumono uno specifico significato nell'ambito del progetto. 
    Per comprenderle in maniera corretta, si rimanda il lettore al documento ``Glossario", che contiene un elenco completo di tutte le terminologie utilizzate con relative definizioni, 
    allo scopo di costruire un linguaggio uniforme che possa migliorare la comunicazione tra i componenti interni al gruppo e gli stakeholders\textsuperscript{G} esterni.   
\end{beginningnote}

%%%%%%%%%%%%%%%%%%%%%%%%%%%%%%%%%%%
% INTRODUZIONE
%%%%%%%%%%%%%%%%%%%%%%%%%%%%%%%%%%%
\section{Introduzione}\label{sec:introduzione}
\subsection{Scopo del documento}\label{sec:introduzione:scopo_del_documento}
    Il presente documento ha lo scopo di descrivere nel dettaglio i compononeti utilizzati e le scelte progettuali adottate per la realizzazione completa del progetto.
    Verranno quindi trattati in dettaglio gli aspetti fondamentali riguardanti i requisiti necessari per il prodotto, le tecnologie adottatte, i design pattern applicati, l'architettura logica e quella di deployment del prodotto.
    Si intende fornire il ragionamento e le disposizioni per lo sviluppo del progetto, in modo da garantire coerenza con i requisiti specificati nel documento \hyperref[sec:ref_esterni]{Analisi dei requisiti}. 

\subsection{Scopo del progetto}\label{sec:introduzione:scopo_del_progetto}
    Il progetto nasce nell'ambito dei \textbf{sistemi gestionali di magazzino}, meglio noti con il termine inglese di \textit{Warehouse Management Systems} (WMS), con 
    l'obiettivo di risolvere una serie di problematiche derivanti dalle soluzioni tradizionali tuttora presenti sul mercato.\\
    Il focus principale sarà migliorare la user experience, tramite la realizzazione di un applicativo che proponga all'utente un'interazione con il magazzino in un 
    ambiente di lavoro 3D. \\
    Tale soluzione, rispetto ai tradizionali sistemi 2D, garantirebbe una maggiore comprensione degli spazi, proponendo una visualizzazione più intuitiva e completa 
    degli spazi di magazzino. Permetterebbe quindi all'utente di prendere decisioni in modo più efficace ed efficiente, permettendo così di ottimizzare i processi di logistica.
    
    Per raggiungere questo obiettivo, l'ambiente di lavoro non può essere una semplice visualizzazione del magazzino. L'utente dovrà infatti poter:
    \begin{itemize}
        \item Spostarsi all'interno dell'ambiente 3D;
        \item Progettare le scaffalature che sono presenti nel magazzino e modificarle nel tempo;
        \item Simulare i flussi di movimento di prodotti.
    \end{itemize}
    
    Il progetto deve concretizzarsi nella realizzazione di una web app fruibile agli impiegati d'ufficio ed incentrata sulla visualizzazione 3D del magazzino.
    Per visionare il capitolato\textsuperscript{G} completo e la documentazione del gruppo, si veda la sezione \hyperref[sec:ref_esterni]{Riferimenti Esterni} 
    del documento.

%%%%%%%%%%%%%%%%%%%%%%%%%%%%%%%%%%%
% REQUISITI
%%%%%%%%%%%%%%%%%%%%%%%%%%%%%%%%%%%
\section{Requisiti}\label{sec:requisiti}

%%%%%%%%%%%%%%%%%%%%%%%%%%%%%%%%%%%
% TECNOLOGIE
%%%%%%%%%%%%%%%%%%%%%%%%%%%%%%%%%%%
\section{Tecnologie}\label{sec:tecnologie}

%%%%%%%%%%%%%%%%%%%%%%%%%%%%%%%%%%%
% ARCHITETTURA LOGICA
%%%%%%%%%%%%%%%%%%%%%%%%%%%%%%%%%%%
\section{Architettura logica}\label{sec:architettura_logica}

%%%%%%%%%%%%%%%%%%%%%%%%%%%%%%%%%%%
% ARCHITETTURA DI DEPLOYMENT
%%%%%%%%%%%%%%%%%%%%%%%%%%%%%%%%%%%
\section{Architettura di deployment}\label{sec:architettura_deployment}

%%%%%%%%%%%%%%%%%%%%%%%%%%%%%%%%%%%
% RIFERIMENTI ESTERNI
%%%%%%%%%%%%%%%%%%%%%%%%%%%%%%%%%%%
\section{Riferimenti esterni}\label{sec:ref_esterni}

\begin{itemize}
    \item Capitolato \textbf{Warehouse Management 3D}:\\
    \url{https://www.math.unipd.it/~tullio/IS-1/2023/Progetto/C5.pdf} 
    \item Link alla \textbf{documentazione del gruppo}:\\
    \url{https://avant-garde-software-engineering.github.io/documentazione.html} \textcolor{gray}{\textit{(ultimo accesso 28-04-24)}}
    %non sono sicuro se abbia senso mettere il link all'analisi dei requisiti visto che c'è già il link alla documentazione del gruppo
\end{itemize}