\section*{\Large Registro delle Modifiche}
    \begin{table}[h]
        \centering
        \rowcolors{2}{gray!30}{white}
        \renewcommand\tabularxcolumn[1]{m{#1}} %per centrare contenuto nelle tabelle
        \renewcommand{\arraystretch}{1.5}
        \begin{tabularx}{0.98\textwidth}
            {c|c|c|c|>{\centering\arraybackslash}X}
            \rowcolor{black}
            \textbf{\color{white} Versione} & \textbf{\color{white} Data} & \textbf{\color{white} Nominativo} & \textbf{\color{white} Ruolo} & \textbf{\color{white} Descrizione} \\ 
            \hline

            %inserire versione, data, nome, ruole e cosa è stato modificato
            %più vecchio
            v0.0.1 & 28-04-24 & Zaccaria Marangon & Amministratore & Prima redazione, scrittura delle sezioni \ref{sec:introduzione:scopo_del_documento},\ref{sec:introduzione:scopo_del_progetto}. \\
            v0.1.0 & 29-04-24 & Luca Securo & Responsabile & Verifica del documento alla versione v0.0.1. \\
            v0.1.1 & 05-05-24 & Jessica Carretta & Progettista & Scrittura delle sezioni \ref{sec:tecnologie}, \ref{sec:architettura_logica}, \ref{sec:architettura_deployment}, \ref{sec:requisiti} e \ref{sec:ref_esterni}.\\
            v0.2.0 & 07-05-24 & Zaccaria Marangon & Verificatore & Revisione e correzione delle sezioni \ref{sec:tecnologie}, \ref{sec:architettura_logica}, \ref{sec:architettura_deployment}, \ref{sec:requisiti} e \ref{sec:ref_esterni}. \\
            v1.0.0 & 08-05-24 & Lorenzo Pasqualotto & Responsabile & Approvazione del documento.\\
            %più recente
            \hline
        \end{tabularx}
    \end{table}