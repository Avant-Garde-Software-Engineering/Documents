\begin{beginningnote}
    Si tenga presente che alcuni termini utilizzati nel documento riportano la lettera \textbf{G} in apice, allo scopo di evidenziare le parole che assumono uno specifico significato nell'ambito del progetto. Per comprenderle in maniera corretta, si rimanda il lettore al documento ``Glossario", che contiene un elenco completo di tutte le terminologie utilizzate con relative definizioni, allo scopo di costruire un linguaggio uniforme che possa migliorare la comunicazione tra i componenti interni al gruppo e gli stakeholder\textsuperscript{G} esterni.
\end{beginningnote}

%%%%%%%%%%%%%%%%%%%%%%%%%%%%%%%%%%%
% SCOPO DEL DOCUMENTO
%%%%%%%%%%%%%%%%%%%%%%%%%%%%%%%%%%%
\section{Scopo del documento}\label{sec:scopo_del_documento}
Questo documento è destinato sia ai membri del gruppo che agli stakeholder in quanto ha come obiettivo quello di indicare tempi, costi e modalità di sviluppo delle varie fasi del progetto\textsuperscript{G}.
In particolare, al suo interno, sono riportati:
\begin{itemize}
    \item Un'analisi dei rischi, comprendente anche delle tecniche di mitigazione implementate per limitarne le problematiche;
    \item Il modello di sviluppo scelto per il progetto;
    \item La pianificazione delle milestones\textsuperscript{G} del progetto, inclusi i relativi costi e tempi di completamento (sia preventivi sia a consuntivo).
\end{itemize}
Vista la natura del documento, è previsto che questo venga redatto in maniera incrementale, aggiornandolo a seconda dei bisogni di gruppo e proponente e in seguito a riflessioni nate dal compimento delle diverse fasi di sviluppo. Per una visione precisa delle modifiche, si rimanda al changelog, che descrive per ciascuna versione le differenze rispetto a quella precedente.

%%%%%%%%%%%%%%%%%%%%%%%%%%%%%%%%%%%
% IL PROGETTO
%%%%%%%%%%%%%%%%%%%%%%%%%%%%%%%%%%%
\section{Il progetto}\label{sec:il_progetto}
\par Il progetto nasce nell'ambito dei \textbf{sistemi gestionali di magazzino}, meglio noti con il termine inglese di \textit{Warehouse Management Systems} (WMS), con l'obiettivo di risolvere una serie di problematiche derivanti dalle soluzioni tradizionali tuttora presenti sul mercato.
\par Il focus principale sarà migliorare la user experience, tramite la realizzazione di un applicativo che proponga all'utente un'interazione con il magazzino in un ambiente di lavoro 3D: questa soluzione, rispetto ai tradizionali sistemi 2D, garantirebbe una maggiore comprensione degli spazi, proponendo una visualizzazione più intuitiva e familiare del magazzino all'utente che, di conseguenza, sarà in grado di prendere decisioni organizzative più informate ed efficienti, ottimizzando i processi di logistica.
\par Per raggiungere questo obiettivo, l'ambiente di lavoro non può essere una semplice visualizzazione del magazzino. L'utente dovrà infatti poter:
\begin{itemize}
    \item Navigare l'ambiente 3D;
    \item Progettare la scaffalatura e modificarla nel tempo;
    \item Simulare i flussi di movimento di mezzi e prodotti.
\end{itemize}
Il progetto deve concretizzarsi nella realizzazione di una web app fruibile agli impiegati d'ufficio ed incentrata sulla visualizzazione 3D del magazzino.
\par Per visionare il capitolato\textsuperscript{G} e la documentazione del gruppo, si veda la sezione \hyperref[sec:riferimenti_esterni]{Riferimenti Esterni} del documento.

\newpage
%%%%%%%%%%%%%%%%%%%%%%%%%%%%%%%%%%%
% ANALISI DEI RISCHI
%%%%%%%%%%%%%%%%%%%%%%%%%%%%%%%%%%%
\section{Analisi dei rischi}\label{sec:analisi_rischi}
Lo scopo di questa sezione è quella di prendere in esame tutte le possibili problematiche che potrebbero verificarsi durante la realizzazione del progetto, al fine di evitare che questi rischi si concretizzino e minaccino così l'avanzamento delle attività di progetto.
Questa analisi sarà organizzata in forma tabellare, in modo da consentire un monitoraggio continuo e più accessibile, e dividendo i rischi a seconda delle seguenti categorie:
\begin{itemize}
    \item Rischi di tipo tecnico-tecnologico;
    \item Rischi organizzativi.
\end{itemize}
In particolare, per ciascun rischio viene fornito:
\begin{itemize}
    \item \textbf{Una breve descrizione}.
    \item \textbf{La probabilità di occorrenza}, indicata attraverso:
        \begin{itemize}
            \item \textbf{A}: per occorrenza alta;
            \item \textbf{M}: per occorrenza media;
            \item \textbf{B}: per occorrenza bassa.
        \end{itemize}
    \item \textbf{Il grado di pericolosità}, indicato attraverso i seguenti colori:
        \begin{itemize}
        \item \colorbox{red!50}{\textbf{Rosso:}} per rischi con pericolosità alta;
        \item \colorbox{orange!50}{\textbf{Arancione:}} per rischi con pericolosità media;
        \item \colorbox{yellow!75}{\textbf{Giallo:}} per rischi con pericolosità bassa.
        \end{itemize}
    \item \textbf{Precauzioni da prendere}.
    \item \textbf{Il piano di contingenza}\textsuperscript{G}.
\end{itemize}

\subsection{Rischi di tipo tecnico-tecnologico}\label{sec:analisi_rischi:tec}
\renewcommand{\arraystretch}{1.5}
\begin{xltabular}{\textwidth}{p{0.00001\textwidth} X | X | X | p{0.05\textwidth}}
    \rowcolor{black}
    & \textbf{\color{white} Rischio} & \textbf{\color{white} Precauzioni} & \textbf{\color{white} Piano di contingenza} & \textbf{\color{white} Occ.}\\ 
    \hline
    \endhead
    
    \cellcolor{red}& \textbf{Tecnologie sconosciute:} il gruppo ha scarsa esperienza con le tecnologie da utilizzare in fase di codifica del prodotto software. 
    & Ogni membro deve comunicare ai colleghi il suo livello di conoscenza relativo alla tecnologia da utilizzare per migliorare l'efficienza del gruppo (e.g. trovando una migliore suddivisione del lavoro).
    & Si cerca di approfondire la tecnologia da utilizzare con lo studio individuale, ed eventualmente collettivo, della documentazione fornita. Nel caso la tecnologia non sia funzionale e/o causi ritardi troppo prolungati si può valutarne la sostituzione o dismissione. 
    & A \\
    \hline
    
    \cellcolor{orange}& \textbf{Strumenti sconosciuti:} il gruppo non ha alcuna esperienza con software di gestione dei progetti. 
    & Prima di utilizzare uno strumento sconosciuto, si valuta congiuntamente l'efficienza/efficacia del suo utilizzo. Ogni membro dovrà poi esercitarsi a comprendere gli aspetti principali dello strumento e segnalare le eventuali difficoltà incontrate.
    & In caso di dubbi non risolti attraverso lo studio individuale, si può ricorrere ad un incontro con altri membri del gruppo per cercare di risolvere il problema più velocemente. Altrimenti, se ciò è causa di ritardi troppo lunghi, si cerca un'alternativa o si valuta di non usarlo. 
    & A \\
    \hline

    \cellcolor{orange}& \textbf{Problemi hardware o software:} lo strumento di lavoro (sia esso software o hardware) di un componente del gruppo potrebbe non permettere, totalmente o parzialmente, lo svolgimento di una qualche attività del progetto. 
    & Il membro che incorrerà in questo rischio (e.g. a causa di un guasto) dovrà farlo presente tempestivamente agli altri membri del gruppo.
    & Si cerca di utilizzare software affidabili ed effettuare backup periodici. Nel caso di malfunzionamento del dispositivo, è necessario cercare di svolgere i compiti assegnati usandone un altro. Se ciò non fosse possibile, si cerca di ridistribuire le attività in modo da limitare il più possibile il risultante rallentamento nello sviluppo del progetto.
    & B \\
    \hline \\

    \caption{Tabella dei rischi di tipo tecnico-tecnologico}
    \label{tab:rischi:tec}
\end{xltabular}

\subsection{Rischi organizzativi}\label{sec:analisi_rischi:org}
\renewcommand{\arraystretch}{1.5}
\begin{xltabular}{\textwidth}{p{0.00001\textwidth} X | X | X | p{0.05\textwidth}}

    \rowcolor{black}
    & \textbf{\color{white} Rischio} & \textbf{\color{white} Precauzioni} & \textbf{\color{white} Piano di contingenza} & \textbf{\color{white} Occ.}\\ 
    \hline
    \endhead
    
    \cellcolor{red}& \textbf{Inesperienza professionale-organizzativa:} la maggior parte del gruppo affronta per la prima volta un progetto così complesso.
    & Ogni membro deve comunicare ai colleghi quelli che potrebbero rivelarsi dei punti di criticità.
    & Si cerca di approfondire quanto possibile con lo studio individuale. Se ciò non dovesse bastare, il membro può richiedere un aiuto agli altri componenti del gruppo. Nel caso la criticità persistesse, si richiedono chiarimenti ai docenti del corso e/o all'azienda proponente.
    & A \\
    \hline

    \cellcolor{red}& \textbf{Prospetti economici e temporali non rispettati: } i costi monetari e temporali per le varie attività potrebbero essere stati stimati incorrettamente a causa dell'inesperienza del gruppo in tal senso.
    & Si cerca di non pianificare in maniera ottimistica ma tenendo presente eventuali ostacoli in cui si potrebbe incorrere durante l'avanzamento del progetto. Per quanto riguarda la gestione delle scadenze, il responsabile si impegna a richiamare l'attenzione del gruppo su quelle imminenti per evitare slittamenti nel progetto.
    & Se un membro del gruppo si accorge che non sta rispettando la pianificazione lo farà presente al responsabile, che valuterà una rilocazione di risorse oppure, in casi estremi, la modifica del preventivo proposto.
    & A \\
    \hline
    
    \cellcolor{orange}& \textbf{Disponibilità oraria varia:} i membri del gruppo hanno impegni diversi che potrebbero generare difficoltà nelle tempistiche di lavoro e/o nell'organizzazione di incontri collettivi.
    & Ogni membro si impegna ad essere il più possibile reperibile e, se ciò non fosse possibile, di comunicare le sue disponibilità agli altri membri in modo da poter organizzare il lavoro in maniera più efficiente.
    & Si comunica agli altri membri del gruppo i propri impegni attraverso gli strumenti predisposti e si cerca di trovare assieme una soluzione che sfrutti al meglio il tempo a disposizione di tutti.
    & M \\
    \hline

    \cellcolor{yellow}& \textbf{Comunicazione esterna:} una parte del gruppo non ha conoscenza pratica per quanto riguarda la comunicazione in ambito professionale.
    & Il gruppo affida al proponente esterno, più esperto, la scelta del canale di comunicazione più appropriato.
    & Il gruppo si impegna a seguire le indicazioni fornite dal proponente in merito. Altrimenti, si cerca, assieme ai membri più esperti, di pensare a soluzioni alternative da proporre.
    & M \\
    \hline

    \cellcolor{red}& \textbf{Modifiche al progetto in corso d'opera:} l'azienda potrebbe richiedere l'aggiunta e/o modifica di alcuni requisiti, tecnologie e funzionalità a progetto già iniziato.
    & Il gruppo aggiorna l'azienda al completamento di ogni obiettivo prestabilito in modo che questa possa valutare attentamente il lavoro svolto.
    & Si mantiene un rapporto continuativo con l'azienda, aggiornandola sull'avanzamento del progetto. Se l'azienda dovesse comunicare al gruppo un cambiamento, le attività saranno ristrutturate dal responsabile di conseguenza.
    & B \\
    \hline

    \cellcolor{orange}& \textbf{Rapporti interni:} si potrebbero verificare delle difficoltà qualora due o più membri del gruppo dovessero trovarsi, per un qualche motivo, in disaccordo sulle decisioni da prendere.
    & Si cerca di evitare queste situazioni e pensare in ottica collettiva, cercando compromessi.
    & Le opinioni di tutti devono essere discusse allo scopo di prendere la decisione migliore. Nel caso di conflitti, è nell'interesse di tutti i membri del gruppo reinstaurare e mantenere un dialogo costruttivo.
    & B \\
    \hline

    \cellcolor{orange}& \textbf{Distribuzione disomogenea:} il carico di lavoro potrebbe essere mal distribuito, e.g. troppo dispendioso per alcuni e/o troppo leggero per altri. 
    & Ciascun membro, in base alle proprie disponibilità e agli impegni presi per il progetto, si impegna a far presente al gruppo le proprie capacità, segnalando eventuali compiti che potrebbero essere non/più adatti alla sua situazione corrente e futura.
    & Sarà compito del responsabile decidere come redistribuire il lavoro in maniera più efficiente.
    & B \\
    \hline

    \cellcolor{orange}& \textbf{Rapporti esterni: } il proponente potrebbe essere poco presente e/o di scarso aiuto.
    & Si cerca di capire fin da subito le disponibilità dell'azienda.
    & Il responsabile si occuperà di gestire la comunicazione esterna cercando di far presente se dovessero esserci delle difficoltà in tal senso.
    & B \\
    \hline \\

    \caption{Tabella dei rischi organizzativi}
    \label{tab:rischi:org}
\end{xltabular}

\newpage
%%%%%%%%%%%%%%%%%%%%%%%%%%%%%%%%%%%
% MODELLO DI SVILUPPO
%%%%%%%%%%%%%%%%%%%%%%%%%%%%%%%%%%%
\section{Modello di sviluppo}\label{sec:modello_sviluppo}
Un modello di ciclo di vita\textsuperscript{G} serve a fornire relazioni temporali e logiche di un processo software\textsuperscript{G}.
Noi di Avant-Garde abbiamo deciso di scegliere un approccio iterativo, più precisamente abbiamo deciso di adottare un modello agile.
% Illustrare e spiegare la scelta del modello di sviluppo da applicare al progetto

\subsection{Il modello agile}\label{sec:modello_sviluppo:agile}
La metodologia Agile è un approccio alla gestione dei progetti che prevede la suddivisione del progetto in fasi e sottolinea l'importanza della collaborazione e del miglioramento continui, tramite frequenti incontri tra i membri del gruppo e con l'azienda.

\noindent I vantaggi di un modello agile comprendono:
\begin{itemize}
    \item Rispondere in modo rapido e flessibile alle richieste dei clienti e ad eventuali problemi emergenti;
    \item Alta comunicazione fra sviluppatori e cliente;
    \item Distinzione di diverse fasi che permettono i vantaggi di un modello incrementale (e.g. permettono di concentrarsi su piccoli obiettivi raggiungibili e di rivedere periodicamente i risultati).
\end{itemize}
Il ciclo di vita di un progetto che utilizza il modello agile si divide nelle seguenti fasi:
\begin{itemize}
    \item Envision;
    \item Per ogni sprint\textsuperscript{G}:
    \begin{itemize}
        \item Speculate,
        \item Explore,
        \item Adapt;
    \end{itemize}
    \item Close.
\end{itemize}
Vediamole in dettaglio.

\subsubsection{Envision}\label{sec:modello_sviluppo:agile:envision}
In questa prima fase si determinano con il cliente gli obiettivi del progetto, si decidono il team e le norme da utilizzare.
Il completamento della fase di Envision produce:
\begin{itemize}
    \item Un \textit{Piano di Progetto} che definisca la pianificazione generale e lo scopo del progetto stesso;
    \item Gli obiettivi complessivi del progetto;
    \item Gli stakeholder del progetto;
    \item Delle \textit{Norme di Progetto} che definiscano metodologie e strumenti di lavoro e collaborazione del gruppo.
\end{itemize}

\subsubsection{Speculate}\label{sec:modello_sviluppo:agile:speculate}
In questa fase si pianificano per ogni sprint dei requisiti da soddisfare, facendo delle stime sul lavoro richiesto e sui tempi previsti, considerando anche i possibili rischi da gestire. %Una feature è un pezzo di funzionalità o outcome che ha valore per il cliente.
A tal fine, viene definita anche una lista di feature\textsuperscript{G}, basate sui requisiti precedentemente individuati, che devono essere completate durante lo sprint.

\subsubsection{Explore}\label{sec:modello_sviluppo:agile:explore}
Durante la fase Explore si sviluppa effettivamente il prodotto. In particolare, per ogni sprint, vengono realizzate le feature individuate durante la fase di Speculate. A tal fine, sono previsti incontri frequenti tra i membri del gruppo e revisioni delle feature non appena queste vengono create.


\subsubsection{Adapt}\label{sec:modello_sviluppo:agile:adapt}
La fase di Adapt termina lo sprint. In particolare, essa consiste in una revisione finale delle feature da parte del cliente e in una riunione documentata dei membri del team per riflettere su ciò che è stato fatto durante l'ultima iterazione. Si condivide dunque l'esperienza accumulata nel corso di progetto e viene rivista la pianificazione per lo sprint successivo.

\subsubsection{Close}\label{sec:modello_sviluppo:agile:close}
Il progetto passa per le fasi di Speculate, Explore e Adapt fino al momento in cui tutti gli sprint sono completati. Una volta finite tutte le iterazioni e implementate le feature, inizia la fase Close.
Durante questa fase ci si assicura che i deliverable\textsuperscript{G} siano completati e che il cliente sia soddisfatto di questi. Il progetto, dunque, termina.
\vspace{20pt}

\noindent \textit{D'ora in poi, per comodità di esposizione, in questo documento le fasi di Speculate, Explore e Adapt saranno raggruppate in un singolo periodo di tempo chiamato ``Sprint'', come lo definisce il modello Agile.}

\newpage
%%%%%%%%%%%%%%%%%%%%%%%%%%%%%%%%%%%
% PIANIFICAZIONE
%%%%%%%%%%%%%%%%%%%%%%%%%%%%%%%%%%%
\section{Pianificazione}\label{sec:pianificazione}
Si è deciso di suddividere il progetto nelle seguenti quattro fasi:
\begin{itemize}
    \item Analisi;
    \item Progettazione e codifica del Proof of Concept\textsuperscript{G};
    \item Progettazione di dettaglio e codifica finale;
    \item Validazione e collaudo.
\end{itemize}
Ognuna di queste fasi sarà formata da attività mostrate nei corrispettivi diagrammi di Gantt\textsuperscript{G}. Ogni attività si ramifica poi in sotto-attività per mostrarne l’esecuzione ad alto livello.

\subsection{Analisi}\label{sec:pianificazione:analisi}
\textbf{Periodo previsto}: dal 06/11/2023 al 06/01/2024\\\\
Questa fase inizia subito dopo l'assegnazione del capitolato d'appalto e termina il 06/01/2024 come concordato dal gruppo. 
Possiamo far corrispondere questa fase all'Envision nel modello agile. Infatti, i suoi obiettivi sono l'analisi in dettaglio del capitolato e l'individuazione dei requisiti, oltre che la stesura dei documenti preliminari alla codifica.

\subsubsection{Attività}\label{sec:pianificazione:analisi:attivita}
\begin{itemize}
    \item \textbf{Norme di progetto:} stesura del documento \textit{Norme di progetto} necessario a descrivere le regole che il gruppo si impone per il corretto svolgimento del progetto. Con la stesura del documento vengono inoltre studiate le tecnologie e gli strumenti che verranno utilizzati nel corso del progetto.
    \item \textbf{Piano di progetto:} stesura del documento \textit{Piano di progetto}, che presenta un'analisi dei rischi, descrive il piano di progetto in dettaglio e calcola il preventivo per la realizzazione del progetto.
    \item \textbf{Piano di qualifica:} stesura del documento \textit{Piano di qualifica}, in cui vengono esposte tecniche e metodi utilizzati per garantire la qualità del prodotto.
    \item \textbf{Analisi dei requisiti:} stesura del documento di \textit{Analisi dei requisiti}, con una descrizione approfondita dei casi d'uso per il prodotto software e dei requisiti di progetto emersi dallo studio del capitolato, dei casi d'uso stessi e attraverso gli incontri con il proponente.
    \item \textbf{Glossario:} stesura del \textit{Glossario}, che dovrà contenere i termini principali utilizzati nell'ambito del progetto.
\end{itemize}
\begin{figure}[H]
    \centering
    \includegraphics[width=0.8\textwidth]{images/gantt_analisi.PNG}
    \caption{Diagramma di Gantt della fase di analisi}
    \label{fig:gantt_analisi}
\end{figure}

\subsubsection{Periodi}\label{sec:pianificazione:analisi:periodi}
La pianificazione di questa fase è stata organizzata nei seguenti tre periodi.

\paragraph{Primo periodo}\label{sec:pianificazione:analisi:periodi:primo}

\begin{xltabular}{\textwidth}{{0.35\textwidth} | X}
        
    \rowcolor{black}
    \textbf{\color{white} Info} & \textbf{\color{white} Primo Periodo}\\ 
    \hline
    \endhead
    
    \textbf{Data inizio prevista} 
    & 12/11/2023 \\
    \hline

    \textbf{Data fine prevista} 
    & 30/11/2023 \\
    \hline

    \textbf{Pre-condizione} 
    & Assegnazione del capitolato. \\
    \hline
    
    \textbf{Post-condizione} 
    & Stesura iniziale dei documenti elencati attraverso l'utilizzo dei nuovi mezzi e del template\textsuperscript{G} creato appositamente. \\
    \hline

    \textbf{Ruoli attivi} 
    &  \begin{itemize}[topsep=0pt]
        \item Responsabile
        \item Amministratore
        \item Verificatore
        \item Analista
    \end{itemize}\\
    \hline

    \textbf{Descrizione} 
    &  Il gruppo effettua un'analisi preliminare del progetto, anche confrontandosi con l'azienda, e comincia il lavoro di stesura dei documenti che verranno presentati alla revisione della Technology Baseline, a partire dallo studio degli strumenti che verranno utilizzati. \\
    \hline
    
    \textbf{Attività} 
    & \begin{itemize}[topsep=0pt]
        \item Individuazione e studio strumenti
        \item Stesura documento \textit{Norme di progetto} con i dettagli sui metodi di svolgimento del progetto e descrizione degli strumenti che verranno utilizzati
        \item Inizio stesura documento \textit{Piano di progetto}
        \item Inizio stesura documento \textit{Piano di qualifica}
        \item Inizio del lavoro di analisi dei requisiti e incontri con il proponente
        \item Stesura del \textit{Glossario}
    \end{itemize} \\
    \hline

\caption{Tabella descrittiva del periodo 1 della fase di analisi}\label{tab:periodo1_1}
\end{xltabular}

\paragraph{Secondo periodo}\label{sec:pianificazione:analisi:periodi:secondo}

\begin{xltabular}{\textwidth}{{0.35\textwidth} | X}
        
    \rowcolor{black}
    \textbf{\color{white} Info} & \textbf{\color{white} Secondo Periodo}\\ 
    \hline
    \endhead
    
    \textbf{Data inizio prevista} 
    & 1/12/2023 \\
    \hline

    \textbf{Data fine prevista} 
    & 02/01/2024 \\
    \hline

    \textbf{Pre-condizione} 
    & Soddisfacimento post-condizioni del periodo precedente. \\
    \hline
    
    \textbf{Post-condizione} 
    & Completamento della stesura dei documenti iniziati durante la fase precedente. \\
    \hline

    \textbf{Ruoli attivi} 
    &  \begin{itemize}[topsep=0pt]
        \item Responsabile
        \item Amministratore
        \item Verificatore
        \item Analista
    \end{itemize}\\
    \hline

    \textbf{Descrizione} 
    &  Il gruppo si impegna a continuare e completare i documenti iniziati durante la fase precedente in vista della verifica e approvazione di questi. Il documento \textit{Norme di progetto} è completato e deve essere perfezionato. \\
    \hline
    
    \textbf{Attività} 
    & \begin{itemize}[topsep=0pt]
        \item Perfezionamento del documento \textit{Norme di progetto}, verifica e approvazione dello stesso
        \item Completamento stesura documento \textit{Piano di progetto}
        \item Completamento stesura documento \textit{Piano di qualifica}
        \item Completamento del lavoro di analisi dei requisiti e stesura dell'omonimo documento
        \item Aggiornamento del \textit{Glossario} con i nuovi termini individuati dalla documentazione
    \end{itemize} \\
    \hline

\caption{Tabella descrittiva del periodo 2 della fase di analisi}\label{tab:periodo1_2}
\end{xltabular}

\paragraph{Terzo periodo}\label{sec:pianificazione:analisi:periodi:terzo}

\begin{xltabular}{\textwidth}{{0.35\textwidth} | X}
        
    \rowcolor{black}
    \textbf{\color{white} Info} & \textbf{\color{white} Terzo Periodo}\\ 
    \hline
    \endhead
    
    \textbf{Data inizio prevista} 
    & 03/01/2024 \\
    \hline

    \textbf{Data fine prevista} 
    & 06/01/2024 \\
    \hline

    \textbf{Pre-condizione} 
    & Soddisfacimento post-condizioni del periodo precedente. \\
    \hline
    
    \textbf{Post-condizione} 
    & Verifica e approvazione di tutti i documenti prodotti nella fase corrente. \\
    \hline

    \textbf{Ruoli attivi} 
    &  \begin{itemize}[topsep=0pt]
        \item Responsabile
        \item Amministratore
        \item Verificatore
        \item Analista
    \end{itemize}\\
    \hline

    \textbf{Descrizione} 
    &  I documenti scritti vengono ora verificati e approvati secondo quanto stabilito nel \textit{Piano di qualifica}, accertandosi che tutti seguano quanto stabilito nelle \textit{Norme di progetto}. \\
    \hline
    
    \textbf{Attività} 
    & \begin{itemize}[topsep=0pt]
        \item Perfezionamento del documento \textit{Norme di progetto}, verifica e approvazione dello stesso
        \item Verifica e approvazione del documento \textit{Piano di progetto}
        \item Verifica e approvazione del documento \textit{Piano di qualifica}
        \item Verifica e approvazione del documento \textit{Analisi dei requisiti}
        \item Verifica e aggiornamento del \textit{Glossario}
    \end{itemize} \\
    \hline

\caption{Tabella descrittiva del periodo 3 della fase di analisi}\label{tab:periodo1_3}
\end{xltabular}

%%%%%%%%%%%%%%%%%%%%%%%%%%%%%%%%%%%   PROGETTAZIONE RTB   %%%%%%%%%%%%%%%%%%%%%%%%%%%%%%%%%%%

\subsection{Progettazione e codifica del Proof of concept}\label{sec:pianificazione:progRTB}

\textbf{Periodo previsto}: dal 07/01/2024 al 23/02/2024\\\\
Questa fase inizia una volta completati tutti i documenti preliminari previsti nella fase di Analisi e termina il 23/02/2024 come concordato dal gruppo, con il completamento della codifica del \textit{Proof of concept} (abbreviato in \textit{PoC}). Al termine di questa fase il gruppo avrà creato un PoC funzionante che verrà presentato alla revisione della Technology Baseline.

\subsubsection{Attività}\label{sec:pianificazione:progRTB:attivita}
\begin{itemize}
    \item \textbf{Studio tecnologie:} studio delle tecnologie nuove al gruppo per l'implementazione del Proof of Concept.
    \item \textbf{Progettazione PoC:} progettazione del Proof of Concept e discussione con l'azienda sui punti da implementare.
    \item \textbf{Codifica PoC:} codifica del Proof of Concept.
    \item \textbf{Aggiornamento documentazione:} ulteriore revisione dei documenti ed eventuale aggiornamento di essi.
\end{itemize}

\begin{figure}[H]
    \centering
    \includegraphics[width=0.8\textwidth]{images/gantt_codRTB.PNG}
    \caption{Diagramma di Gantt della fase di codifica del PoC}
    \label{fig:gantt_codRTB}
\end{figure}

\subsubsection{Periodi}\label{sec:pianificazione:progRTB:periodi}
Questa fase è composta da un periodo di progettazione e da uno di codifica del PoC, quest'ultimo composto da 4 sprint.

\paragraph{Primo periodo}\label{sec:pianificazione:progRTB:periodi:primo}
\begin{xltabular}{\textwidth}{{0.35\textwidth} | X}
        
    \rowcolor{black}
    \textbf{\color{white} Info} & \textbf{\color{white} Primo Periodo}\\ 
    \hline
    \endhead
    
    \textbf{Data inizio prevista} 
    & 07/01/2024 \\
    \hline

    \textbf{Data fine prevista} 
    & 15/01/2024 \\
    \hline

    \textbf{Pre-condizione} 
    & Soddisfacimento post-condizioni del periodo precedente. \\
    \hline
    
    \textbf{Post-condizione} 
    & Lo studio delle tecnologie adottate per la codifica è terminato e la progettazione architetturale minimale del PoC è definita. \\
    \hline

    \textbf{Ruoli attivi} 
    &  \begin{itemize}[topsep=0pt]
        \item Responsabile
        \item Amministratore
        \item Verificatore
        \item Analista
        \item Progettista
    \end{itemize}\\
    \hline

    \textbf{Descrizione} 
    &  Durante questo periodo si lavora sulla progettazione del Proof of Concept, ad un livello minimale e non eccessivamente approfondito vista la natura ``usa e getta" del PoC, si prende familiarità con le tecnologie necessarie e si discute con l'azienda sui requisiti minimi da implementare all'interno del PoC. Si effettuano eventuali modifiche e correzioni dei documenti. \\
    \hline
    
    \textbf{Attività} 
    & \begin{itemize}[topsep=0pt]
        \item Studio tecnologie per la codifica del PoC 
        \item Riunioni con l'azienda per discutere dei requisiti minimi da implementare nel PoC
        \item Aggiornamento dei documenti
    \end{itemize} \\
    \hline

\caption{Tabella descrittiva del periodo 1 della fase di progettazione e codifica del PoC}\label{tab:periodo2_1}
\end{xltabular}

\paragraph{Secondo periodo}\label{sec:pianificazione:progRTB:periodi:secondo}
\begin{xltabular}{\textwidth}{{0.35\textwidth} | X}
        
    \rowcolor{black}
    \textbf{\color{white} Info} & \textbf{\color{white} Secondo Periodo}\\ 
    \hline
    \endhead
    
    \textbf{Data inizio prevista} 
    & 16/01/2024 \\
    \hline

    \textbf{Data fine prevista} 
    & 23/02/2024 \\
    \hline

    \textbf{Pre-condizione} 
    & Soddisfacimento post-condizioni del periodo precedente. \\
    \hline
    
    \textbf{Post-condizione} 
    & Codice PoC funzionante. \\
    \hline

    \textbf{Ruoli attivi} 
    &  \begin{itemize}[topsep=0pt]
        \item Responsabile
        \item Amministratore
        \item Verificatore
        \item Programmatore
    \end{itemize}\\
    \hline

    \textbf{Descrizione} 
    &  Durante questo periodo si passa alla codifica del codice seguendo quanto definito nel primo periodo della fase di progettazione e codifica del PoC. \\
    \hline
    
    \textbf{Attività} 
    & \begin{itemize}[topsep=0pt]
        \item \textbf{Codifica:} Implementazione del PoC
        \item Aggiornamento dei documenti
    \end{itemize} \\
    \hline

\caption{Tabella descrittiva del periodo 2 della fase di progettazione e codifica del PoC}\label{tab:periodo4_2}
\end{xltabular}

\noindent Nello specifico, questo periodo è composto dai seguenti quattro sprint di avanzamento della codifica del Proof of Concept.


\subparagraph{Primo sprint}\label{sec:pianificazione:codificaRTB:periodi:primo}

\begin{xltabular}{\textwidth}{{0.35\textwidth} | X}
        
    \rowcolor{black}
    \textbf{\color{white} Info} & \textbf{\color{white} Primo sprint}\\ 
    \hline
    \endhead
    
    \textbf{Data inizio prevista} 
    & 16/01/2024 \\
    \hline

    \textbf{Data fine prevista} 
    & 21/01/2024 \\
    \hline

    \textbf{Pre-condizione} 
    & Soddisfacimento post-condizioni del periodo precedente. \\
    \hline
    
    \textbf{Post-condizione} 
    & Interfaccia utente che permette di visualizzare un ambiente 3D. \\
    \hline

    \textbf{Ruoli attivi} 
    &  \begin{itemize}[topsep=0pt]
        \item Responsabile
        \item Amministratore
        \item Verificatore
        \item Progettista
        \item Programmatore
    \end{itemize}\\
    \hline
    
    \textbf{Attività} 
    & \begin{itemize}[topsep=0pt]
        \item Approfondimento delle tecnologie studiate finora 
        \item \textbf{Creazione ambiente 3D:} Implementazione di un'interfaccia utente che permetta di muoversi all'interno di un ambiente 3D che visualizza un magazzino con dimensioni inserite dall'utente
    \end{itemize} \\
    \hline

\caption{Tabella descrittiva dello sprint 1 della codifica del PoC}\label{tab:periodo3_1}
\end{xltabular}

\subparagraph{Secondo sprint}\label{sec:pianificazione:codificaRTB:periodi:secondo}

\begin{xltabular}{\textwidth}{{0.35\textwidth} | X}
        
    \rowcolor{black}
    \textbf{\color{white} Info} & \textbf{\color{white} Secondo sprint}\\ 
    \hline
    \endhead
    
    \textbf{Data inizio prevista} 
    & 22/01/2024 \\
    \hline

    \textbf{Data fine prevista} 
    & 04/02/2024 \\
    \hline

    \textbf{Pre-condizione} 
    & Soddisfacimento post-condizioni dello sprint precedente. \\
    \hline
    
    \textbf{Post-condizione} 
    & Possibilità di creare, modificare, piazzare e muovere scaffalature. \\
    \hline

    \textbf{Ruoli attivi} 
    &  \begin{itemize}[topsep=0pt]
        \item Responsabile
        \item Amministratore
        \item Verificatore
        \item Programmatore
    \end{itemize}\\
    \hline
    
    \textbf{Attività} 
    & \begin{itemize}[topsep=0pt]
        \item \textbf{Creazione scaffalature:} Implementazione della funzionalità che permetta all'utente di creare una nuova scaffalatura
        \item \textbf{Piazzamento scaffalature:} Possibilità di poter piazzare la scaffalatura creata all'interno del magazzino 3D
        \item \textbf{Spostamento scaffalature:} Possibilità di spostare, modificare o eliminare una scaffalatura piazzata in precedenza nel magazzino 3D
    \end{itemize} \\
    \hline
    \caption{Tabella descrittiva dello sprint 2 della codifica del PoC}\label{tab:periodo3_2}
\end{xltabular}

\subparagraph{Terzo sprint}\label{sec:pianificazione:codificaRTB:periodi:terzo}
\begin{xltabular}{\textwidth}{{0.35\textwidth} | X}
        
    \rowcolor{black}
    \textbf{\color{white} Info} & \textbf{\color{white} Terzo sprint}\\ 
    \hline
    \endhead
    
    \textbf{Data inizio prevista} 
    & 05/02/2024 \\
    \hline

    \textbf{Data fine prevista} 
    & 15/02/2024 \\
    \hline

    \textbf{Pre-condizione} 
    & Soddisfacimento post-condizioni dello sprint precedente. \\
    \hline
    
    \textbf{Post-condizione} 
    & Aggiunta di prodotti con possibilità di piazzarli nelle scaffalature presenti nel magazzino 3D. \\
    \hline

    \textbf{Ruoli attivi} 
    &  \begin{itemize}[topsep=0pt]
        \item Responsabile
        \item Amministratore
        \item Verificatore
        \item Programmatore
    \end{itemize}\\
    \hline
    
    \textbf{Attività} 
    & \begin{itemize}[topsep=0pt]
        \item \textbf{Creazione prodotto:} Tramite interfaccia utente, possibilità di creare un nuovo prodotto
        \item \textbf{Posizionamento prodotto:} Possibilità di piazzare il prodotto creato all'interno di una scaffalatura libera
    \end{itemize} \\
    \hline

\caption{Tabella descrittiva dello sprint 3 della codifica del PoC}\label{tab:periodo3_3}
\end{xltabular}

\subparagraph{Quarto sprint}\label{sec:pianificazione:codificaRTB:periodi:quarto}

\begin{xltabular}{\textwidth}{{0.35\textwidth} | X}
        
    \rowcolor{black}
    \textbf{\color{white} Info} & \textbf{\color{white} Quarto sprint}\\ 
    \hline
    \endhead
    
    \textbf{Data inizio prevista} 
    & 16/01/2024 \\
    \hline

    \textbf{Data fine prevista} 
    & 23/02/2024 \\
    \hline

    \textbf{Pre-condizione} 
    & Soddisfacimento post-condizioni dello sprint precedente. \\
    \hline
    
    \textbf{Post-condizione} 
    & Possibilità di creare una richiesta di spostamento di un item da un luogo ad un altro. \\
    \hline

    \textbf{Ruoli attivi} 
    &  \begin{itemize}[topsep=0pt]
        \item Responsabile
        \item Amministratore
        \item Verificatore
        \item Programmatore
    \end{itemize}\\
    \hline
    
    \textbf{Attività} 
    & \begin{itemize}[topsep=0pt]
        \item \textbf{Interfaccia spostamento oggetto:} Creazione di un'interfaccia che permetta di creare una richiesta di spostamento di un prodotto da un luogo ad un altro
        \item \textbf{Creazione richiesta spostamento:} Implementare la creazione di una richiesta di spostamento di un prodotto
    \end{itemize} \\
    \hline

\caption{Tabella descrittiva dello sprint 4 della codifica del PoC}\label{tab:periodo3_4}
\end{xltabular}

%%%%%%%%%%%%%%%%%%%%%%%%%%%%%%%%%%%   PROGETTAZIONE E CODIFICA   %%%%%%%%%%%%%%%%%%%%%%%%%%%%%%%%%%%

\subsection{Progettazione di dettaglio e codifica finale}\label{sec:pianificazione:progCodifica}

\textbf{Periodo previsto}: dal 24/02/2024 al 01/04/2024\\\\
Questa fase inizia una volta completata la codifica del \textit{Proof of Concept} e completata la prima revisione (RTB). La fase termina il 01/04/2024 come concordato dal gruppo, con un prodotto funzionante e pronto per la consegna al proponente. Durante questa fase verranno anche creati documenti quali il \textit{Manuale Utente} e la \textit{Specifica Tecnica}.


\subsubsection{Attività}\label{sec:pianificazione:prog_codifica:attivita}
\begin{itemize}
    \item \textbf{Progettazione di dettaglio:} viene progettato nel dettaglio il prodotto finale, in modo da poterlo implementare in modo efficiente, completo e con meno problematiche possibile. 
    \item \textbf{Incremento dei documenti:} aggiornamento e verifica dei documenti.
    \item \textbf{Codifica:} viene implementato il prodotto finale, partendo da quanto emerso tramite la realizzazione del PoC. La codifica avviene in modo incrementale, seguendo gli incrementi individuati nella fase di analisi secondo il modello Agile.
    \item \textbf{Manuale Utente:} stesura del \textit{Manuale Utente}.
    \item \textbf{Specifica Tecnica:} stesura della \textit{Specifica Tecnica}.
    \item \textbf{Verifica:} il codice viene verificato in fase di scrittura per facilitare il lavoro nella fase di validazione.
\end{itemize}
\begin{figure}[H]
    \centering
    \includegraphics[width=0.8\textwidth]{images/gantt_dettaglio.PNG}
    \caption{Diagramma di Gantt della fase di codifica finale}
    \label{fig:gantt_codRTB}
\end{figure}

\subsubsection{Periodi}\label{sec:pianificazione:prog_codifica:periodi}
Questa fase è composta da due periodi, l'ultimo dei quali composto da 2 sprint.

\paragraph{Primo periodo}\label{sec:pianificazione:prog_codifica:periodi:primo}

\begin{xltabular}{\textwidth}{{0.35\textwidth} | X}
        
    \rowcolor{black}
    \textbf{\color{white} Info} & \textbf{\color{white} Primo Periodo}\\ 
    \hline
    \endhead
    
    \textbf{Data inizio prevista} 
    & 25/02/2024 \\
    \hline

    \textbf{Data fine prevista} 
    & 05/03/2024 \\
    \hline

    \textbf{Pre-condizione} 
    & Soddisfacimento post-condizioni del periodo precedente, superamento revisione RTB. \\
    \hline
    
    \textbf{Post-condizione} 
    & Progetto architetturale ad alto livello completato e pronto ad essere implementato.\\
    \hline

    \textbf{Ruoli attivi} 
    &  \begin{itemize}[topsep=0pt]
        \item Responsabile
        \item Amministratore
        \item Verificatore
        \item Progettista
        \item Programmatore
    \end{itemize}\\
    \hline

    \textbf{Descrizione} 
    &  Durante questo periodo si progetta il prodotto finale, revisionando i dettagli e discutendone con l'azienda. Vengono inoltre corretti i documenti dopo le segnalazioni dei committenti alla revisione di avanzamento RTB.\\
    \hline
    
    \textbf{Attività} 
    & \begin{itemize}[topsep=0pt]
        \item Scelta dei design pattern\textsuperscript{G} e dell'architettura logica con la definizione delle componenti e delle unità architetturali\textsuperscript{G}
        \item Preparazione dei diagrammi UML delle classi per il documento di \textit{Specifica Tecnica} 
        \item Aggiornamento dei documenti
    \end{itemize} \\
    \hline

\caption{Tabella descrittiva del periodo 1 progettazione e codifica dettaglio}\label{tab:periodo4_1}
\end{xltabular}

\paragraph{Secondo periodo}\label{sec:pianificazione:prog_codifica:periodi:secondo}

\begin{xltabular}{\textwidth}{{0.35\textwidth} | X}
        
    \rowcolor{black}
    \textbf{\color{white} Info} & \textbf{\color{white} Secondo Periodo}\\ 
    \hline
    \endhead
    
    \textbf{Data inizio prevista} 
    & 05/03/2024 \\
    \hline

    \textbf{Data fine prevista} 
    & 01/04/2024 \\
    \hline

    \textbf{Pre-condizione} 
    & Soddisfacimento post-condizioni del periodo precedente. \\
    \hline
    
    \textbf{Post-condizione} 
    & Programma funzionante. \\
    \hline

    \textbf{Ruoli attivi} 
    &  \begin{itemize}[topsep=0pt]
        \item Responsabile
        \item Amministratore
        \item Verificatore
        \item Programmatore
    \end{itemize}\\
    \hline

    \textbf{Descrizione} 
    &  Durante questo periodo si passa alla codifica del codice seguendo il progetto architetturale, implementando nel dettaglio ogni requisito. \\
    \hline
    
    \textbf{Attività} 
    & \begin{itemize}[topsep=0pt]
        \item \textbf{Manuale utente:} Inizio stesura del \textit{Manuale Utente}
        \item \textbf{Specifica tecnica:} Inizio stesura della Specifica Tecnica
        \item \textbf{Codifica:} Implementazione del prodotto finale
        \item Aggiornamento dei documenti
    \end{itemize} \\
    \hline

\caption{Tabella descrittiva del periodo 2 progettazione e codifica dettaglio}\label{tab:periodo4_2}
\end{xltabular}


\noindent Nello specifico, questo periodo è composto dai seguenti due sprint di avanzamento della codifica del prodotto finale.


\subparagraph{Primo sprint}\label{sec:pianificazione:codificaPB:periodi:primo}

\begin{xltabular}{\textwidth}{{0.35\textwidth} | X}
        
    \rowcolor{black}
    \textbf{\color{white} Info} & \textbf{\color{white} Primo sprint}\\ 
    \hline
    \endhead
    
    \textbf{Data inizio prevista} 
    & 05/03/2024 \\
    \hline

    \textbf{Data fine prevista} 
    & 18/03/2024 \\
    \hline

    \textbf{Pre-condizione} 
    & Soddisfacimento post-condizioni del periodo precedente. \\
    \hline
    
    \textbf{Post-condizione} 
    & Interfaccia utente che permette di visualizzare un ambiente 3D. Possibilità di creare, modificare ed eliminare aree. Possibilità di creare, modificare, piazzare e muovere scaffalature.\\
    \hline

    \textbf{Ruoli attivi} 
    &  \begin{itemize}[topsep=0pt]
        \item Responsabile
        \item Amministratore
        \item Verificatore
        \item Progettista
        \item Programmatore
    \end{itemize}\\
    \hline
    
    \textbf{Attività} 
    & \begin{itemize}[topsep=0pt]
         \item \textbf{Specifica tecnica:} Inizio stesura della Specifica Tecnica
        \item Definizione degli unit test da eseguire
        \item \textbf{Creazione ambiente 3D:} Implementazione di un'interfaccia utente che permetta di muoversi all'interno di un ambiente 3D che visualizza un magazzino con planimetria inserita dall'utente
        \item \textbf{Creazione aree:} Implementazione della funzionalità che permetta all'utente di creare delle aree interne al magazzino
        \item \textbf{Modifica aree:} Implementazione della funzionalità che permetta all'utente di modificare o eliminare un'area del magazzino
        \item \textbf{Creazione scaffalature:} Implementazione della funzionalità che permetta all'utente di creare una nuova scaffalatura
        \item \textbf{Piazzamento scaffalature:} Possibilità di poter piazzare la scaffalatura creata all'interno del magazzino 3D
        \item \textbf{Spostamento scaffalature:} Possibilità di spostare, modificare o eliminare una scaffalatura piazzata in precedenza nel magazzino 3D
    \end{itemize} \\
    \hline

\caption{Tabella descrittiva dello sprint 1 della codifica del prodotto finale}
\end{xltabular}

\subparagraph{Secondo sprint}\label{sec:pianificazione:codificaPB:periodi:secondo}

\begin{xltabular}{\textwidth}{{0.35\textwidth} | X}
        
    \rowcolor{black}
    \textbf{\color{white} Info} & \textbf{\color{white} Secondo sprint}\\ 
    \hline
    \endhead
    
    \textbf{Data inizio prevista} 
    & 19/03/2024 \\
    \hline

    \textbf{Data fine prevista} 
    & 01/04/2024 \\
    \hline

    \textbf{Pre-condizione} 
    & Soddisfacimento post-condizioni dello sprint precedente. \\
    \hline
    
    \textbf{Post-condizione} 
    & Interfaccia utente che permette di visualizzare e ricercare i dati in forma testuale. Possibilità di salvare i dati del magazzino in un file e di utilizzarlo come configuratore dell'ambiente 3D. Aggiunta di prodotti con possibilità di piazzarli nelle scaffalature presenti nel magazzino 3D. \\
    \hline

    \textbf{Ruoli attivi} 
    &  \begin{itemize}[topsep=0pt]
        \item Responsabile
        \item Amministratore
        \item Verificatore
        \item Programmatore
    \end{itemize}\\
    \hline
    
    \textbf{Attività} 
    & \begin{itemize}[topsep=0pt]
        \item \textbf{Manuale utente:} Inizio stesura del \textit{Manuale Utente}
        \item \textbf{Creazione libreria:} Implementazione di un'interfaccia utente che permetta di visualizzare i dati delle scaffalature e dei prodotti in forma testuale e di filtrarli attraverso una ricerca
        \item \textbf{Creazione prodotto:} Tramite interfaccia utente, possibilità di creare un nuovo prodotto
        \item \textbf{Posizionamento prodotto:} Possibilità di piazzare il prodotto creato (ed eventualmente già posizionato) all'interno di una scaffalatura libera
        \item \textbf{Salvataggio dati:} Implementazione della funzionalità che permetta all'utente di salvare i dati del magazzino in un file
        \item \textbf{Configurazione da file:} Implementazione della funzionalità che permetta all'utente di creare l'ambiente 3D attraverso il caricamento di un file apposito
    \end{itemize} \\
    \hline
    \caption{Tabella descrittiva dello sprint 2 della codifica del prodotto finale}
\end{xltabular}



%%%%%%%%%%%%%%%%%%%%%%%%%%%%%%%%%%%   VALIDAZIONE   %%%%%%%%%%%%%%%%%%%%%%%%%%%%%%%%%%%

\subsection{Validazione e collaudo}\label{sec:pianificazione:val_collaudo}

\textbf{Periodo previsto}: dal 02/04/2024 al 26/04/2024\\\\
Questa fase inizia una volta completata la codifica del prodotto finale e termina con la revisione PB. In particolare, essa corrisponde alla Close del modello Agile. Infatti, questa fase è volta a verificare, revisionare e approvare il prodotto finale e la relativa documentazione.

\subsubsection{Attività}\label{sec:pianificazione:val_collaudo:attivita}
Questa fase ha come attività principali la verifica finale ad alto livello del prodotto realizzato nella fase precedente, la validazione rispetto ai requisiti preposti e la presentazione all'azienda proponente, nonchè il completamento e verifica finale della documentazione del progetto.

\begin{figure}[H]
    \centering
    \includegraphics[width=0.8\textwidth]{images/gantt_collaudo.PNG}
    \caption{Diagramma di Gantt del periodo di verifica e validazione}
    \label{fig:gantt_collaudo}
\end{figure}


\subsubsection{Periodi}\label{sec:pianificazione:val_collaudo:periodi}
Questa fase è composta da un unico periodo.

\paragraph{Primo Periodo}\label{sec:pianificazione:val_collaudo:periodi:primo}

\begin{xltabular}{\textwidth}{{0.35\textwidth} | X}
        
    \rowcolor{black}
    \textbf{\color{white} Info} & \textbf{\color{white}Primo Periodo}\\ 
    \hline
    \endhead
    
    \textbf{Data inizio prevista} 
    & 02/04/2024 \\
    \hline

    \textbf{Data fine prevista} 
    & 26/04/2024 \\
    \hline

    \textbf{Pre-condizione} 
    & Soddisfacimento post-condizioni del periodo precedente. \\
    \hline
    
    \textbf{Post-condizione} 
    & Consegna Product Baseline. \\
    \hline

    \textbf{Ruoli attivi} 
    &  \begin{itemize}[topsep=0pt]
        \item Responsabile
        \item Amministratore
        \item Verificatore
    \end{itemize}\\
    \hline

    \textbf{Descrizione} 
    &  Questa fase è adibita alla verifica e validazione del codice e del completamento dei documenti prodotti durante il progetto. \\
    \hline

\caption{Tabella descrittiva del periodo di verifica e validazione}\label{tab:periodo5_1}
\end{xltabular}

\newpage
%%%%%%%%%%%%%%%%%%%%%%%%%%%%%%%%%%%
% PREVENTIVO
%%%%%%%%%%%%%%%%%%%%%%%%%%%%%%%%%%%

\section{Preventivo}\label{sec:preventivo}

Questa sezione descrive un prospetto economico e orario del progetto, suddiviso per fasi e ruoli. Per ogni fase viene riportato ogni membro del gruppo, con le ore da esso ricoperte per ogni ruolo e infine il costo complessivo della fase per ruolo.
Ogni membro del gruppo deve ricoprire ogni ruolo almeno una volta per la durata del progetto, perciò i ruoli vengono cambiati periodicamente accordandosi con gli altri membri in modo da soddisfare questa richiesta.
Per una descrizione più dettagliata della suddivisione dei membri e dello scopo dei vari ruoli, consultare il documento \textit{Norme di Progetto} nella sezione dedicata ai processi organizzativi.\\

\subsection{Dettaglio per periodo}\label{sec:preventivo:periodi}
A seguire sono elencate tutte le fasi del progetto, con i relativi ruoli e le ore che ogni membro del gruppo dovrà ricoprire per ogni fase, seguito dal costo complessivo per fase. Per ottimizzare l'utilizzo dello spazio, i vari ruoli vengono abbreviati nel modo seguente:\\
\begin{itemize}
    \item \textbf{Re:} \textit{Responsabile}
    \item \textbf{Am:} \textit{Amministratore}
    \item \textbf{An:} \textit{Analista}
    \item \textbf{Pt:} \textit{Progettista}
    \item \textbf{Pr:} \textit{Programmatore}
    \item \textbf{Ve:} \textit{Verificatore}
\end{itemize}

%%%%%%%%%%%%%%%%%%%%%%%%%%%%%%%%%%%   ANALISI   %%%%%%%%%%%%%%%%%%%%%%%%%%%%%%%%%%%

%\newpage
\subsubsection{Analisi}\label{sec:preventivo:periodi:analisi}

\begin{center}
\begin{xltabular}{\textwidth}{| 1 | {0.35\textwidth} | {0.35\textwidth} | {0.35\textwidth} | {0.35\textwidth} | {0.35\textwidth} | {0.35\textwidth} | 1 |}
        
    \rowcolor{black}
    \textbf{\color{white} Nominativo} & \textbf{\color{white}Re}& \textbf{\color{white}Am}& \textbf{\color{white}An}& \textbf{\color{white}Pt}& \textbf{\color{white}Pr}& \textbf{\color{white}Ve}& \textbf{\color{white}Totale ore}\\ 
    \hline
    \endhead

    Jessica Carretta & 3 & 6 & 6 & 0 & 0 & 5 & 20 \\
    \hline
    
    Giulio Biscontin & 3 & 5 & 4 & 0 & 0 & 8 & 20 \\
    \hline
    
    Luca Securo & 2	& 0 & 10 & 0 & 0 & 8 & 20 \\
    \hline
    
    Andrea Mangolini & 5 &	6 &	4 &	0 &	0 &	5 &	20 \\
    \hline
    
    Zaccaria Marangon & 2 & 0 & 5 & 0 & 0 & 13 & 20 \\
    \hline
    
    Lorenzo Pasqualotto & 4 & 4 & 9 & 0 & 0 & 3 & 20 \\
    \hline

\caption{Suddivisione dei ruoli nel periodo di Analisi}\label{tab:ruoli_analisi}
\end{xltabular}

\begin{xltabular}{\textwidth}{| 1 | {0.35\textwidth} | 1 |}
            
    \rowcolor{black}
    \textbf{\color{white} Ruolo} & \textbf{\color{white} Ore} & \textbf{\color{white} Costo}\\ 
    \hline
    \endhead

    Responsabile & 19 & €570,00 \\
    \hline
    
    Amministratore & 21 & €420,00 \\
    \hline
    
    Analista & 38 & €950,00 \\
    \hline
    
    Progettista & 0 & €0,00 \\
    \hline
    
    Programmatore & 0 & €0,00 \\
    \hline
    
    Verificatore & 42 & €630,00 \\
    \hline
    
    \textbf{Totale} & \textbf{120} & \textbf{€2.570,00} \\
    \hline
        
    \caption{Costo per ruolo Analisi}\label{tab:costo_analisi}
\end{xltabular}
\end{center}

\begin{figure}[H]
    \centering
    \includegraphics[width=0.75\textwidth]{images/grafico_analisi.png}
    \caption{Grafico a barre suddivisione ruoli Analisi}
    \label{fig:grafico_analisi}
\end{figure}

\begin{figure}[H]
    \centering
    \includegraphics[width=0.5\textwidth]{images/torta_analisi.png}
    \caption{Grafico a torta suddivisione costi per ruolo Analisi}
    \label{fig:torta_analisi}
\end{figure}

%%%%%%%%%%%%%%%%%%%%%%%%%%%%%%%%%%%   PROGETTAZIONE RTB   %%%%%%%%%%%%%%%%%%%%%%%%%%%%%%%%%%%

\newpage
\subsubsection{Progettazione e codifica del Proof of Concept}\label{sec:preventivo:periodi:progRTB}

\begin{center}
\begin{xltabular}{\textwidth}{| 1 | {0.35\textwidth} | {0.35\textwidth} | {0.35\textwidth} | {0.35\textwidth} | {0.35\textwidth} | {0.35\textwidth} | 1 |}
        
    \rowcolor{black}
    \textbf{\color{white} Nominativo} & \textbf{\color{white}Re}& \textbf{\color{white}Am}& \textbf{\color{white}An}& \textbf{\color{white}Pt}& \textbf{\color{white}Pr}& \textbf{\color{white}Ve}& \textbf{\color{white}Totale ore}\\ 
    \hline
    \endhead

    Jessica Carretta & 4 & 3 & 3 & 8 & 7 & 7 & 32 \\
    \hline
    
    Giulio Biscontin & 2 & 0 & 4 & 5 & 13 & 8 & 32 \\
    \hline
    
    Luca Securo & 3	& 6 & 0 & 7 & 10 & 6 & 32 \\
    \hline
    
    Andrea Mangolini & 2 &	1 &	0 &	8 & 11 & 10 & 32 \\
    \hline
    
    Zaccaria Marangon & 2 & 4 & 2 & 6 & 10 & 8 & 32 \\
    \hline
    
    Lorenzo Pasqualotto & 1 & 4 & 0 & 2 & 10 & 15 & 32 \\
    \hline

\caption{Suddivisione dei ruoli nel periodo di Realizzazione Proof of Concept}\label{tab:ruoli_progRTB}
\end{xltabular}

\begin{xltabular}{\textwidth}{| 1 | {0.35\textwidth} | 1 |}
            
    \rowcolor{black}
    \textbf{\color{white} Ruolo} & \textbf{\color{white} Ore} & \textbf{\color{white} Costo}\\ 
    \hline
    \endhead

    Responsabile & 14 & €420,00 \\
    \hline
    
    Amministratore & 18 & €360,00 \\
    \hline
    
    Analista & 9 & €225,00 \\
    \hline
    
    Progettista & 36 & €900,00 \\
    \hline
    
    Programmatore & 61 & €915,00 \\
    \hline
    
    Verificatore & 54 & €810,00 \\
    \hline
    
    \textbf{Totale} & \textbf{192} & \textbf{€3.630,00} \\
    \hline
        
    \caption{Costo per ruolo progettazione e codifica PoC}\label{tab:costo_progRTB}
\end{xltabular}
\end{center}

\begin{figure}[H]
    \centering
    \includegraphics[width=0.75\textwidth]{images/grafico_progPOC.png}
    \caption{Grafico a barre suddivisione ruoli progettazione e codifica PoC}
    \label{fig:grafico_progRTB}
\end{figure}

\begin{figure}[H]
    \centering
    \includegraphics[width=0.5\textwidth]{images/torta_progPOC.png}
    \caption{Grafico a torta suddivisione costi per ruolo progettazione e codifica PoC}
    \label{fig:torta_progRTB}
\end{figure}

%%%%%%%%%%%%%%%%%%%%%%%%%%%%%%%%%%%   CODIFICA DETTAGLIO   %%%%%%%%%%%%%%%%%%%%%%%%%%%%%%%%%%%

%\newpage
\subsubsection{Progettazione di dettaglio e codifica finale}\label{sec:preventivo:periodi:dettaglio}

\begin{center}
    \begin{xltabular}{\textwidth}{| 1 | {0.35\textwidth} | {0.35\textwidth} | {0.35\textwidth} | {0.35\textwidth} | {0.35\textwidth} | {0.35\textwidth} | 1 |}
            
        \rowcolor{black}
        \textbf{\color{white} Nominativo} & \textbf{\color{white}Re}& \textbf{\color{white}Am}& \textbf{\color{white}An}& \textbf{\color{white}Pt}& \textbf{\color{white}Pr}& \textbf{\color{white}Ve}& \textbf{\color{white}Totale ore}\\ 
        \hline
        \endhead
    
        Jessica Carretta & 2 & 0 & 0 & 9 & 18 & 5 & 34 \\
        \hline
        
        Giulio Biscontin & 5 & 1 & 0 & 11 & 11 & 6 & 34 \\
        \hline
        
        Luca Securo & 3	& 0 & 0 & 10 & 13 & 8 & 34 \\
        \hline
        
        Andrea Mangolini & 2 &	0 &	4 &	10 & 13 & 5 & 34 \\
        \hline
        
        Zaccaria Marangon & 2 & 5 & 0 & 12 & 15 & 0 & 34 \\
        \hline
        
        Lorenzo Pasqualotto & 5 & 0 & 0 & 15 & 14 & 0 & 34 \\
        \hline
    
    \caption{Suddivisione dei ruoli nel periodo di progettazione di dettaglio e codifica finale}\label{tab:ruoli_dettaglio}
    \end{xltabular}

\begin{xltabular}{\textwidth}{| 1 | {0.35\textwidth} | 1 |}
            
    \rowcolor{black}
    \textbf{\color{white} Ruolo} & \textbf{\color{white} Ore} & \textbf{\color{white} Costo}\\ 
    \hline
    \endhead

    Responsabile & 13 & €390,00 \\
    \hline
    
    Amministratore & 6 & €120,00 \\
    \hline
    
    Analista & 4 & €100,00 \\
    \hline
    
    Progettista & 67 & €1.675,00 \\
    \hline
    
    Programmatore & 84 & €1.260,00 \\
    \hline
    
    Verificatore & 24 & €360,00 \\
    \hline
    
    \textbf{Totale} & \textbf{198} & \textbf{€3.855,00} \\
    \hline
        
    \caption{Costo per ruolo progettazione di dettaglio e codifica finale}\label{tab:costo_dettaglio}
\end{xltabular}
\end{center}

\begin{figure}[H]
    \centering
    \includegraphics[width=0.75\textwidth]{images/grafico_codPOC.png}
    \caption{Grafico a barre suddivisione ruoli progettazione di dettaglio e codifica finale}
    \label{fig:grafico_dettaglio}
\end{figure}

\begin{figure}[H]
    \centering
    \includegraphics[width=0.5\textwidth]{images/torta_codPOC.png}
    \caption{Grafico a torta suddivisione costi per ruolo progettazione di dettaglio e codifica finale}
    \label{fig:torta_dettaglio}
\end{figure}

%%%%%%%%%%%%%%%%%%%%%%%%%%%%%%%%%%%   VALIDAZIONE   %%%%%%%%%%%%%%%%%%%%%%%%%%%%%%%%%%%

%\newpage
\subsubsection{Validazione e collaudo}\label{sec:preventivo:periodi:collaudo}

\begin{center}
    \begin{xltabular}{\textwidth}{| 1 | {0.35\textwidth} | {0.35\textwidth} | {0.35\textwidth} | {0.35\textwidth} | {0.35\textwidth} | {0.35\textwidth} | 1 |}
            
        \rowcolor{black}
        \textbf{\color{white} Nominativo} & \textbf{\color{white}Re}& \textbf{\color{white}Am}& \textbf{\color{white}An}& \textbf{\color{white}Pt}& \textbf{\color{white}Pr}& \textbf{\color{white}Ve}& \textbf{\color{white}Totale ore}\\ 
        \hline
        \endhead
    
        Jessica Carretta & 0 & 0 & 0 & 0 & 0 & 8 & 8 \\
        \hline
        
        Giulio Biscontin & 0 & 2 & 0 & 0 & 0 & 6 & 8 \\
        \hline
        
        Luca Securo & 0 & 3 & 0 & 0 & 0 & 5 & 8 \\
        \hline
        
        Andrea Mangolini & 2 & 0 & 0 & 0 & 0 & 6 & 8 \\
        \hline
        
        Zaccaria Marangon & 5 & 0 & 0 & 0 & 0 & 3 & 8 \\
        \hline
        
        Lorenzo Pasqualotto & 0 & 0 & 0 & 0 & 0 & 8 & 8 \\
        \hline
    
    \caption{Suddivisione dei ruoli nel periodo di validazione e collaudo}\label{tab:ruoli_collaudo}
    \end{xltabular}

\begin{xltabular}{\textwidth}{| 1 | {0.35\textwidth} | 1 |}
            
    \rowcolor{black}
    \textbf{\color{white} Ruolo} & \textbf{\color{white} Ore} & \textbf{\color{white} Costo}\\ 
    \hline
    \endhead

    Responsabile & 7 & €210,00 \\
    \hline
    
    Amministratore & 5 & €100,00 \\
    \hline
    
    Analista & 0 & €0,00 \\
    \hline
    
    Progettista & 0 & €0,00 \\
    \hline
    
    Programmatore & 0 & €0,00 \\
    \hline
    
    Verificatore & 36 & €540,00 \\
    \hline
    
    \textbf{Totale} & \textbf{48} & \textbf{€850,00} \\
    \hline
        
    \caption{Costo per ruolo validazione e collaudo}\label{tab:costo_collaudo}
\end{xltabular}
\end{center}

\begin{figure}[H]
    \centering
    \includegraphics[width=0.75\textwidth]{images/grafico_collaudo.png}
    \caption{Grafico a barre suddivisione ruoli validazione e collaudo}
    \label{fig:grafico_collaudo}
\end{figure}

\begin{figure}[H]
    \centering
    \includegraphics[width=0.5\textwidth]{images/torta_collaudo.png}
    \caption{Grafico a torta suddivisione costi per ruolo validazione e collaudo}
    \label{fig:torta_collaudo}
\end{figure}

\subsection{Prospetto economico e prospetto orario complessivi}\label{sec:preventivo:totale}
Per queste voci si rimanda al documento \textit{Suddivisione dei ruoli e preventivo dei  costi}, presente alla sezione ``Candidatura'' del \href{https://avant-garde-software-engineering.github.io/documentazione.html}{repository documentale del gruppo}.



\newpage
%%%%%%%%%%%%%%%%%%%%%%%%%%%%%%%%%%%
% CONSUNTIVO DI PERIODO
%%%%%%%%%%%%%%%%%%%%%%%%%%%%%%%%%%%
\section{Consuntivo di periodo}\label{sec:consuntivo}

Di seguito viene presentato un consuntivo per i diversi periodi del progetto, in cui si confrontano le ore preventivate con quelle effettivamente impiegate e i costi preventivati con quelli effettivi, nonchè le difficoltà incontrate e le strategie utilizzate per mitigare i rischi.\\

\subsection{Analisi}\label{sec:consuntivo:analisi}

\begin{center}
    \textbf{Data inizio:} 06/11/2023 \\
    \textbf{Data fine:} 06/01/2024 \\
    \begin{xltabular}{\textwidth}{| 1 | 1 | {0.35\textwidth} | 1 | 1 |}
                
        \rowcolor{black}
        \textbf{\color{white} Ruolo} & \textbf{\color{white} Totale ore} & \textbf{\color{white} Diff. ore} & \textbf{\color{white} Totale costo} & \textbf{\color{white} Diff. costo}\\ 
        \endhead
    
        Responsabile & 19 & 0 & €570,00 & €0,00 \\
        \hline
        
        Amministratore & 25 & +4 & €500,00 & +€80,00 \\
        \hline
        
        Analista & 42 & +4 & €1050,00 & +€100,00 \\
        \hline
        
        Progettista & 0 & 0 & €0,00 & €0,00 \\
        \hline
        
        Programmatore & 0 & 0 & €0,00 & €0,00 \\
        \hline
        
        Verificatore & 36 & -6 & €540,00 & -€90,00 \\
        \hline
        
        \textbf{Totale} & \textbf{122} & \textbf{+2} & \textbf{€2.660,00} & \textbf{+€90,00} \\
        \hline
            
        \caption{Differenza ore e costi previsti con effettivi, Analisi}\label{tab:consuntivo_analisi}
    \end{xltabular}
\end{center}

\subsubsection{Resoconto periodo}\label{sec:consuntivo:analisi:resoconto}

Dal consuntivo di periodo emerge che il gruppo è stato in generale abbastanza in linea con il preventivo,
 con un leggero aumento dei costi dovuto alle ore aggiuntive impiegate dai ruoli di amministratore ed analista.
Questo aumento è stato causato da una sottostima delle ore necessarie per la stesura dei documenti,
 in quanto era per noi la prima esperienza in questo ambito e si è rivelato più dispendioso in termini di tempo di quanto ci aspettassimo.
Inoltre, l'analisi dei requisiti è stata più complessa del previsto, in quanto il capitolato richiedeva una buona comprensione
 del dominio applicativo\textsuperscript{G} e abbiamo dovuto effettuare ricerche approfondite nell'ambito.\\
Nonostante ciò, il gruppo è riuscito a rispettare i tempi di consegna e a mantenere i costi entro i limiti previsti, grazie ad una buona organizzazione
 e ad un lavoro costante e collaborativo, perciò non sono previsti cambiamenti per il prossimo periodo.\\

\newpage
\subsubsection{Mitigazione rischi attuata}\label{sec:consuntivo:analisi:mitigazione}

A seguito vengono riportati alcuni rischi incontrati e la mitigazione effettuata per tali rischi.\\

\paragraph{Rischi organizzativi}

\begin{center}


    \begin{xltabular}{\textwidth}{| X | X |}
                
        \rowcolor{black}
        \textbf{\color{white} Descrizione} & \textbf{\color{white} Mitigazione}\\ 
        \endhead
    
        A causa dei diversi impegni di ogni componente del gruppo, gli orari di disponibilità offerti dagli individui erano spesso molto diversi tra loro &
        Sono stati utilizzati efficientemente i mezzi di comunicazione specificati nelle norme di progetto e il gruppo si è suddiviso i compiti dividendosi in gruppi più piccoli
        con simili disponibilità orarie \\
        \hline
            
        \caption{Tabella descrittiva rischi organizzativi e mitigazioni periodo Analisi}\label{tab:rischi_organizzativi_analisi}
    \end{xltabular}
\end{center}

\paragraph{Rischi tecnologici}

\begin{center}
    \begin{xltabular}{\textwidth}{| X | X |}
                
        \rowcolor{black}
        \textbf{\color{white} Descrizione} & \textbf{\color{white} Mitigazione}\\ 
        \endhead
    
        Essendo molte delle tecnologie utilizzate nuove per il gruppo, era presente il rischio di non riuscire ad utilizzare al meglio tali strumenti &
        Dopo aver osservato tutti i lati positivi e negativi delle opzioni per le tecnologie, il gruppo ha scelto quelle più adatte e comode e ne ha effettuato uno studio per poterle utilizzare al meglio. \\
        \hline
            
        \caption{Tabella descrittiva rischi tecnologici e mitigazioni periodo Analisi}\label{tab:rischi_tecnologici_analisi}
    \end{xltabular}
\end{center}

\subsection{Progettazione e codifica del Proof of Concept}\label{sec:consuntivo:progRTB}
\begin{center}
    \textbf{Data inizio:} 07/01/2024 \\
    \textbf{Data fine:} 25/03/2024 \\
    \begin{xltabular}{\textwidth}{| 1 | 1 | {0.35\textwidth} | 1 | 1 |}
                
        \rowcolor{black}
        \textbf{\color{white} Ruolo} & \textbf{\color{white} Totale ore} & \textbf{\color{white} Diff. ore} & \textbf{\color{white} Totale costo} & \textbf{\color{white} Diff. costo}\\ 
        \endhead
    
        Responsabile & 14 & 0 & €570,00 & €0,00 \\
        \hline
        
        Amministratore & 18 & 0 & €500,00 & €0,00 \\
        \hline
        
        Analista & 12 & +3 & €300,00 & +€75,00 \\
        \hline
        
        Progettista & 24 & -12 & €600,00 & -€300,00 \\
        \hline
        
        Programmatore & 70 & +9 & €1050,00 & +€135,00 \\
        \hline
        
        Verificatore & 54 & 0 & €810,00 & €0,00 \\
        \hline
        
        \textbf{Totale} & \textbf{192} & \textbf{0} & \textbf{€3.540,00} & \textbf{-€90,00} \\
        \hline
            
        \caption{Differenza ore e costi previsti con effettivi, PoC}\label{tab:consuntivo_analisi}
    \end{xltabular}
\end{center}

\subsubsection{Resoconto periodo}\label{sec:consuntivo:analisi:resoconto}

Dal consuntivo emergono alcuni problemi che sono sorti durante questo periodo. 
In particolare, il ruolo di progettista ha impiegato meno ore di quelle preventivate, in quanto il gruppo ha deciso di concentrarsi maggiormente sulla codifica
del PoC, il quale è stato codificato principalmente per testare le tecnologie utilizzate e non per creare un prodotto completo, in modo da avere una base solida
su cui basare la progettazione del prodotto finale. Inoltre, il ruolo di analista ha impiegato più ore di quelle preventivate in quanto alcune richieste presentate
 dal proponente non erano chiare e abbiamo dovuto effettuare delle modifiche al documento dell'analisi dei requisiti.\\
Dal consuntivo emerge inoltre che il gruppo ha sforato molto sul tempo di consegna previsto, in quanto il rallentamento del lavoro causato dagli esami universitari
ha portato ad un ritardo nella consegna del PoC, e una incorretta valutazione delle tecnologie sufficienti per la realizzazione del PoC ha portato ad uno sforo dei
tempi preventivati di consegna del PoC, in quanto implementare tecnologie aggiuntive ha richiesto la ricostruzione della maggior parte del codice.\\
Su queste basi, il gruppo ha deciso di effettuare sostanziali modifiche alla pianificazione dei periodi successivi, in modo da recuperare il ritardo accumulato,
prestando più attenzione a ciò che il gruppo ha deciso, assicurandoci di seguire strettamente le scelte illustrate nei documenti realizzati e coordinando in maniera
più efficiente i vari membri in base alle proprie disponibilità e capacità.\\


\subsubsection{Mitigazione rischi attuata}\label{sec:consuntivo:analisi:mitigazione}

A seguito vengono riportati alcuni rischi incontrati e la mitigazione effettuata per tali rischi.\\

\paragraph{Rischi organizzativi}

\begin{center}


    \begin{xltabular}{\textwidth}{| X | X |}
                
        \rowcolor{black}
        \textbf{\color{white} Descrizione} & \textbf{\color{white} Mitigazione}\\ 
        \endhead
    
        A causa della sessione invernale degli esami universitari, il tempo disponibile offerto dal gruppo per lavorare nel progetto era minimo. &
        Il gruppo si impegna a non arrestare completamente il lavoro che viene svolto durante gli esami e dedica una quantità di tempo ogni giorno
        all'avanzamento sul progetto, in modo da poter ritornare a ritmo normale più facilmente una volta finiti gli esami \\
        \hline

        La revisione con il professor Cardin ha fatto emergere alcune lacune nel nostro codice sotto il punto di vista tecnologico e di implementazione di framework. &
        Il gruppo si è immediatamente impegnato per cercare tecnologie aggiuntive possibilmente utili, anche chiedendo consiglio al proponente. Il gruppo si è inoltre
        rapidamente riorganizzato per minimizzare il ritardo causato da questa richiesta \\
        \hline
            
        \caption{Tabella descrittiva rischi organizzativi e mitigazioni periodo Analisi}\label{tab:rischi_organizzativi_poc}
    \end{xltabular}
\end{center}

\paragraph{Rischi tecnologici}

\begin{center}
    \begin{xltabular}{\textwidth}{| X | X |}
                
        \rowcolor{black}
        \textbf{\color{white} Descrizione} & \textbf{\color{white} Mitigazione}\\ 
        \endhead
    
        Le tecnologie utilizzate nel progetto risultano sconosciute ai membri del gruppo e, talvolta, con limitata documentazione &
        I membri del gruppo si impegnano non solo a studiare singolarmente le tecnologie scelte, ma anche a condividere con il resto del team ciò che imparano
        tramite la scrittura del codice \\
        \hline
            
        \caption{Tabella descrittiva rischi tecnologici e mitigazioni periodo Proof of Concept}\label{tab:rischi_tecnologici_poc}
    \end{xltabular}
\end{center}

\newpage
%%%%%%%%%%%%%%%%%%%%%%%%%%%%%%%%%%%
% MITIGAZIONE DEI RISCHI
%%%%%%%%%%%%%%%%%%%%%%%%%%%%%%%%%%%
%\section{Difficoltà incontrate e mitigazione dei rischi}\label{sec:mitigazione_rischi}
% Scrivere le soluzioni implementate per risolvere i problemi derivanti da rischi effettivamente riscontrati
% DA SCRIVERE UNA VOLTA CHE SI SONO EFFETTIVAMENTE VERIFICATI DEI RISCHI


\newpage
%%%%%%%%%%%%%%%%%%%%%%%%%%%%%%%%%%%
% RIFERIMENTI ESTERNI
%%%%%%%%%%%%%%%%%%%%%%%%%%%%%%%%%%%
\newpage
\section{Riferimenti esterni}\label{sec:riferimenti_esterni}
Per ulteriori chiarimenti sugli argomenti discussi nel documento, si possono consultare i seguenti link esterni:
\begin{itemize}
    \item Capitolato \textbf{Warehouse Management 3D}:\\
    \url{https://www.math.unipd.it/~tullio/IS-1/2023/Progetto/C5.pdf} \textcolor{gray}{\textit{(ultimo accesso 07-04-24)}}
    \item Link alla \textbf{documentazione del gruppo}:\\
    \url{https://avant-garde-software-engineering.github.io/documentazione.html} \textcolor{gray}{\textit{(ultimo accesso 07-04-24)}}
    \item Link alle \textbf{slides sul ciclo di vita del software}:\\
    \url{https://www.math.unipd.it/~tullio/IS-1/2023/Dispense/T3.pdf} \textcolor{gray}{\textit{(ultimo accesso 07-04-24)}}
    \item Link alle \textbf{slides sulla gestione di progetto}:\\
    \url{https://www.math.unipd.it/~tullio/IS-1/2023/Dispense/T4.pdf} \textcolor{gray}{\textit{(ultimo accesso 07-04-24)}} 
\end{itemize}